%===========================================================
% do not change this formatting please :-)
%===========================================================
\documentclass[letter,12pt]{article}
\usepackage[margin=1in]{geometry}
\usepackage{amssymb,amsmath,amsthm,bm,enumitem,mathrsfs,colortbl,fancyhdr,tcolorbox,enumitem}
\def\honorcode{\textit{In accordance with the Hokie Honor Code, I affirm that I have neither given nor received unauthorized assistance on this assignment.}}
\fancyhead{}
\fancyhead[L]{Collin McDevitt } %Replace "NAME: " WITH YOUR NAME
\fancyhead[C]{MATH 3144 HOMEWORK 08}
\fancyhead[R]{PAGE \thepage}
\fancyfoot{}
\renewcommand{\footrulewidth}{0.4pt}
\fancyfoot[C]{\honorcode}
\date{\today}
%===========================================================
% convenient commands -- feel free to add more as you see fit
%===========================================================
\newcommand{\C}{\mathbb{C}}
\newcommand{\K}{\mathbb{K}}
\newcommand{\Lin}{\mathcal{L}}
\newcommand{\Mat}{\mathcal{M}}
\newcommand{\Poly}{\mathcal{P}}
\newcommand{\Bas}{\mathcal{B}}
\newcommand{\Q}{\mathbb{Q}}
\newcommand{\R}{\mathbb{R}}
\newcommand{\dotp}{\boldsymbol{\cdot}}
\newcommand{\Span}{\operatorname{Span}}
\newcommand{\Null}{\operatorname{Null}}



\begin{document}
\pagestyle{fancy}
%===========================================================
%===========================================================


%===========================================================
% PROBLEM 1
%===========================================================
\begin{tcolorbox}
  \textbf{Problem 1.} Let $U$ be the subset of $\Poly_3(\R)$ given by
  $$U = \left\{ p \in \Poly_3(\R) \; : \; p(0)=p(1) \right\}.$$
  Define the function
  \begin{align*}
    & \alpha: U \times U \rightarrow \R \\
    & \alpha(p,q) = \int_0^1 p(x) q'(x)\, dx
  \end{align*}
  \begin{enumerate}
  \item Prove that $\alpha$ is an alternating bilinear form.
  \item Find the matrix $\Mat(\alpha,\Bas)$ for $\alpha$ with respect to the following basis
    $$\Bas = \left\{ x^3 - x, \, x^2 - x, \, 1 \right\}$$
  \end{enumerate}
\end{tcolorbox}
\vskip1em



\begin{proof}
First I will show that $\alpha$ (as defined above) is a bilinear form. 

First fix some $p \in U$ then for any $q_1,q_2 \in U$ and $\lambda \in \R$ we have 

\begin{align*}
    \alpha(p,\lambda q_1 +q_2)&=\int _0 ^1 p(x)(\lambda q_1 +q_2)^\prime(x) dx\\
    &= \int_0^1 p(x)\lambda q_1^\prime(x)+p(x)q_2^\prime(x)dx\\
    &= \lambda \int_0^1 p(x)q_1^\prime(x)dx + \int_0^1 p(x)q_2^\prime(x)dx\\
    &= \lambda \alpha(p,q_1) + \alpha(p,q_2)
\end{align*}

Now fix some $q \in U$ then for any $p_1,p_2 \in U$ and $\lambda \in \R$ we have
\[
    \alpha (\lambda p_1 +p_2,q)= \int _0 ^1 (\lambda p_1 +p_2)(x)q^\prime(x)dx
\]
Using the same logic as above we get $\alpha (\lambda p_1 +p_2,q)=\lambda \alpha (p_1,q)+\alpha(p_2,q)$. 
Hence we have that $\alpha$ is a bilinear form.

Now let $p\in U$ then we have 
\begin{align*}
\alpha(p,p)& = \int_0^1 p(x)p^\prime(x)dx\\
&= \frac{1}{2}p(x)^2 \Big|_0^1\\
&= \frac{1}{2}\big (p(1)^2 - p(0)^2\big )\\   
&= 0 
\end{align*}

Hence we get that it is an alternating bilinear form.



\end{proof}


Now finding the matrix $\Mat (\alpha ,\Bas)$ I will say that $b_1=x^3-x,\;\; b_2 =x^2-x,\;\;b_3=1$. We have immediately that $\alpha(b_i,b_3)=0$ for $i\in \{1,2,3\}$ additionally as this is an alternating form we get $\alpha(b_i,b_i)=0$ and we also get $\alpha(b_i,b_j)=-\alpha(b_j,b_i)$ by Theorem 9.16. Hence we only need to examine the values of $\alpha(b_1,b_2)$

\begin{align*}
    \alpha(b_1,b_2)&= \int_0^1(x^3-x)(x^2-x)^\prime dx\\
    &= \int_0^1(x^3-x)(2x-1)dx\\
    &= \int_0^1 2x^4-x^3-2x^2+xdx\\
    &= \frac{2}{5}x^5-\frac{1}{4}x^4-\frac{2}{3}x^3+\frac{1}{2}x^2\Big|_0^1\\
    &= \frac{2}{5}-\frac{1}{4}-\frac{2}{3}+\frac{1}{2}\\
    &= -\frac{1}{60}
\end{align*}

Hence we get the matrix \[\Mat(\alpha ,\Bas)=\begin{pmatrix}
    0 & -\frac{1}{60} & 0\\
    \frac{1}{60} & 0 & 0\\
    0 & 0 & 0
\end{pmatrix}\]


%===========================================================
% SOLUTION 1
%===========================================================


\newpage


%===========================================================
% PROBLEM 2
%===========================================================
\begin{tcolorbox}
  \textbf{Problem 2.} If $V$ and $W$ are $\K$-vector spaces, observe that the Cartesian $V \times W$ is a $\K$-vector space with the following addition and scalar multiplication operations:
  $$(\mathbf{v_1},\mathbf{w_1})+(\mathbf{v_2},\mathbf{w_2}) = (\mathbf{v_1}+\mathbf{v_2},\mathbf{w_1}+\mathbf{w_2}) \qquad \text{ and } \qquad k(\mathbf{v},\mathbf{w}) = (k \mathbf{v}, k \mathbf{w}).$$
  Show that, in general, a bilinear form $\beta \in V^{(2)}$ is \underline{not} a linear functional, $\Lin(V \times V, \K)$.
\end{tcolorbox}

Let $\beta\in V^{(2)}$ and $(a_1,b_1),(a_2,b_2)\in V\times V$ then if $\beta$ where linear we would have 
\begin{align*}
    \beta((a_1,b_1)+(a_2,b_2))&=\beta(a_1+a_2,b_1+b_2)\\
    &=\beta(a_1+a_2,b_1)+\beta(a_1+a_2,b_2)\\
    &=\beta(a_1,b_1)+\beta(a_2,b_1)+\beta(a_1,b_2)+\beta(a_2,b_2)
\end{align*}
We only get the equality $ \beta((a_1,b_1)+(a_2,b_2))=\beta((a_1,b_1))+\beta((a_2,b_2))$ if and only if $\beta(a_2,b_1)+\beta(a_1,b_2)=0$. 

Now let $(a,b)\in V\times V$ and $\lambda \in \K$ then we have 
\[
    \beta(\lambda(a,b))=\beta(\lambda a,\lambda b)=\lambda \beta(a,\lambda b)=\lambda^2 \beta(a,b)
\]
Hence we only get the equality $\beta(\lambda(a,b))=\lambda \beta ((a,b))$ if and only if $\lambda = \lambda^2$ this doesn't hold for all the scalars hence it is not linear. 


\vskip1em


\newpage


%===========================================================
% PROBLEM 3
%===========================================================
\begin{tcolorbox}
  \textbf{Problem 3.} The notion of a bilinear form can be extended to a \textbf{bilinear map} in the following way: Let $U,V,W$ be $\K$-vector spaces. The function $\Gamma: V \times W \rightarrow U$ is a bilinear map if it satisfies the following: for all scalars $k$ and vectors $\mathbf{v}, \mathbf{w}$:
  $$\Gamma(\mathbf{v_1}+\mathbf{v_2},\mathbf{w}) = \Gamma(\mathbf{v_1},\mathbf{w}) + \Gamma(\mathbf{v_2},\mathbf{w}) \qquad \text{ and } \qquad \Gamma(k \mathbf{v},\mathbf{w}) = k \Gamma(\mathbf{v},\mathbf{w}),$$
  $$\Gamma(\mathbf{v},\mathbf{w_1}+\mathbf{w_2}) = \Gamma(\mathbf{v},\mathbf{w_1}) + \Gamma(\mathbf{v},\mathbf{w_2}) \qquad \text{ and } \qquad \Gamma(\mathbf{v},k \mathbf{w}) = k \Gamma(\mathbf{v},\mathbf{w}).$$
  \begin{enumerate}
  \item Go find your old multivariable calculus textbook and look up the definition of the cross product on $\R^3$.
  \item Prove that $\Gamma: \R^3 \times \R^3 \rightarrow \R^3$ given by
    $$\Gamma(\mathbf{v},\mathbf{w}) = \underbrace{\mathbf{v} \times \mathbf{w}}_{\text{cross product}}$$
    is a bilinear map.
  \item A bilinear map $\Gamma: V \times V \rightarrow U$ is said to be \textbf{alternating} if $\Gamma(\mathbf{v},\mathbf{v}) = \mathbf{0}$ for all $\mathbf{v}$. Prove that the cross product map above is alternating.
  \end{enumerate}
\end{tcolorbox}
\vskip1em

(1) Stewart's Calculus book defines it as 
\begin{tcolorbox}
If $\mathbf{a}=\langle a_1, a_2,a_3\rangle$ and $\mathbf{b}=\langle b_1,b_2,b_3\rangle$, then the \textbf{cross product} of $\mathbf{a}$ and $\mathbf{b}$ is the vector 
\[
    \mathbf{a}\times \mathbf{b}=\langle a_2b_3 - a_3b_2,a_3 b_1 -a_1b_3 ,a_1b_2-a_2b_1 \rangle
\]
\end{tcolorbox}

(2) Let $a=(a_1,a_2,a_3),b=(b_1,b_2,b_3),c=(c_1,c_2,c_3)\in \R^3$ and $\Gamma$ be the cross product map. Then we have 

   \[ \Gamma((a_1,a_2,a_3)+(b_1,b_2,b_3),(c_1,c_2,c_3))=\]

   \[ ((a_2+b_2)c_3-(a_3+b_3)c_2,(a_3+b_3)c_1-(a_1+b_1)c_3,(a_1+b_1)c_2-(a_2+b_2)c_1)\]

   distributing the $c_i$'s we get the equality
   \[
    (a_2c_3-a_3c_2,a_3c_1-a_1c_3,a_1c_2-a_2c_1)+(b_2c_3-b_3c_2,b_3c_1-b_1c_3,b_1c_2-b_2c_1)
   \]
    which is equal to $\Gamma(a,c)+\Gamma(b,c)$ hence we get the first condition.

    Now 
    \[
        \Gamma((a_1,a_2,a_3),(b_1,b_2,b_3)+(c_1,c_2,c_3))=
    \]
    \[
        (a_2(b_3+c_3)-a_3(b_2+c_2),a_3(b_1+c_1)-a_1(b_3+c_3),a_1(b_2+c_2)-a_2(b_1+c_1))
    \]
    distributing each $a_i$ we get the equality
    \[
        (a_2b_3-a_3b_2,a_3b_1-a_1b_3,a_1b_2-a_2b_1)+(a_2c_3-a_3c_2,a_3c_1-a_1c_3,a_1c_2-a_2c_1)
    \]
    which is equal to $\Gamma(a,b)+\Gamma(a,c)$ hence we have the second condition satisfied.
    
    Now let $k\in K$  then we have 
    \begin{align}
        \Gamma(k(a_1,a_2,a_3),(b_1,b_2,b_3))&=  (ka_2b_3-ka_3b_2,ka_3b_1-ka_1b_3,ka_1b_2-ka_2b_1)\\
        &= k(a_2b_3-a_3b_2,a_3b_1-a_1b_3,a_1b_2-a_2b_1)\\
        &= k\Gamma((a_1,a_2,a_3),(b_1,b_2,b_3))
\end{align}
Now as $\Gamma(a,kb)=(ka_2b_3-ka_3b_2,ka_3b_1-ka_1b_3,ka_1b_2-ka_2b_1)$ we get both $\Gamma(ka,b)=k\Gamma(a,b)=\Gamma(a,kb)$ hence we have completed all the conditions and so $\Gamma$ is a bilinear map.
    

(3) 
\begin{proof}
    

Let $(v_1,v_2,v_3)\in \R^3$ then we get 

\begin{align*}
    \Gamma ((v_1,v_2,v_3),(v_1,v_2,v_3))&=(v_2v_3-v_3v_2,v_3v_1-v_1v_3,v_1v_2-v_2v_1)\\
    &= (0,0,0)
\end{align*}

\end{proof}


\end{document}
