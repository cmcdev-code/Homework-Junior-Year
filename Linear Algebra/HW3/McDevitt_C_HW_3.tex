%===========================================================
% do not change this formatting please :-)
%===========================================================
\documentclass[letter,12pt]{article}
\usepackage[margin=1in]{geometry}
\usepackage{amssymb,amsmath,amsthm,bm,mathrsfs,colortbl,fancyhdr,tcolorbox,enumitem}
\def\honorcode{\textit{In accordance with the Hokie Honor Code, I affirm that I have neither given nor received unauthorized assistance on this assignment.}}
\fancyhead{}
\fancyhead[L]{Collin McDevitt} %Replace "NAME: " WITH YOUR NAME
\fancyhead[C]{MATH 3144 HOMEWORK 03}
\fancyhead[R]{PAGE \thepage}
\fancyfoot{}
\renewcommand{\footrulewidth}{0.4pt}
\fancyfoot[C]{\honorcode}
\date{\today}
%===========================================================
% convenient commands -- feel free to add more as you see fit
%===========================================================
\newcommand{\C}{\mathbb{C}}
\newcommand{\K}{\mathbb{K}}
\newcommand{\Poly}{\mathcal{P}}
\newcommand{\Q}{\mathbb{Q}}
\newcommand{\R}{\mathbb{R}}
\newcommand{\dotp}{\boldsymbol{\cdot}}
\newcommand{\Span}{\operatorname{Span}}

\begin{document}
\pagestyle{fancy}
%===========================================================
%===========================================================


%===========================================================
% PROBLEM 1
%===========================================================
\begin{tcolorbox}
  \textbf{Problem 1.}
  Consider $V=\C$, the complex numbers, as a $\C$-vector space. Define a function $\Re: \C \rightarrow \C$ by
  \begin{align*}
    \Re(x+iy) = x
  \end{align*}
  Is $\Re$ a linear map? If so, prove it. If not, explain why not.
\end{tcolorbox}
\vskip1em

%===========================================================
% SOLUTION 1
%===========================================================

No it is not
\begin{proof}
    No as it does not satisfy homogeneity.
    Consider $(1+i)\cdot \Re(1+i)=1+i$ but $\Re((1+i)\cdot (1+i))=\Re(2i)=0$ so it does not satisfy homogeneity.

\end{proof}
\newpage


%===========================================================
% PROBLEM 2
%===========================================================
\begin{tcolorbox}
  \textbf{Problem 2.} \textbf{Extending a linear map.} Let $V$ be a finite-dimensional $\K$-vector space, $W$ a $\K$-vector space, and $U$ asubspace of $V$. Furthermore, let
  \begin{align*}
    \left\{\mathbf{u_1},\ldots,\mathbf{u_m}\right\} & \text{ be a basis for } U \text{ and} \\
    \left\{\mathbf{u_1},\ldots,\mathbf{u_m},\mathbf{v_{m+1}},\ldots,\mathbf{v_n}\right\} & \text{ be the (extended) basis for } V.
  \end{align*}
  Taking $T:U \rightarrow W$ to be any linear map, define a function $f: V \rightarrow W$ by
  $$ f\big(a_1 \mathbf{u_1} + \cdots + a_m \mathbf{u_m} + a_{m+1} \mathbf{v_{m+1}} + \cdots + a_n \mathbf{v_{n}}\big) = T\!\big(a_1 \mathbf{u_1} + \cdots + a_m \mathbf{u_m}\big). $$
  \begin{enumerate}[label=(\alph*)]
  \item Is $f$ a linear map? If so, prove it. If not, explain why not.
  \item What happens if the definition of $f$ is changed to the following?
    $$f(\mathbf{x}) = \begin{cases} T(\mathbf{x}) & \text{ if } \mathbf{x} \in U \\ \mathbf{0} & \text{ otherwise} \end{cases}$$
  \end{enumerate}
\end{tcolorbox}
\vskip1em

%===========================================================
% SOLUTION 2
%===========================================================
\begin{enumerate}[label=(\alph*)]
\item \begin{proof}{Yes it is a linear map}

    Let $\lambda \in \K$ and let $\vec{v}=\mathbf{a_1u_1+...+a_1u_m+a_{m+1}v_{m+1}+...+a_nv_n}\in V$ and $u=\mathbf{b_1w_1+...+b_mw_m+v_{m+1}v_{m+1}+...+b_mv_n}\in V$. Where $u_i,w_i\in \Span(U)$ and $a_i\in \K$
    \begin{align*}
      f(\lambda \vec{v}+\vec{u})&=T(\lambda (a_1u_1+...+a_1u_m)+b_1w_1+...+b_mw_m) \\
      &=T(\lambda (a_1u_1+...+a_1u_m))+T(b_1w_1+...+b_mw_m)   \\
      &= \lambda T(a_1u_1+...+a_1u_m)+T(b_1w_1+...+b_mw_m)\\
      &= \lambda f(\vec v)+f(\vec{u})
    \end{align*}

    
\end{proof}
\item \begin{proof}
  Then it is no longer a linear map. Let $x\in U$ and $y\in V$ where $y\not \in U$. Then $f(x+y)=0$ but $f(x)+f(y)=T(x)+0$ hence it is not linear.
\end{proof}
\end{enumerate}
  

\newpage


%===========================================================
% PROBLEM 3
%===========================================================
\begin{tcolorbox}
  \textbf{Problem 3.} Prove that the following is a subspace of $\mathcal{L}(\K^2)$:
  $$U = \left\{ f \in \mathcal{L}(\K^2) \: : \: \begin{array}{l} a,b,c \in \K \text{ and} \\ f(x,y) = (ax+by,bx+cy) \end{array} \right\}$$
\end{tcolorbox}
\vskip1em

%===========================================================
% SOLUTION 3
%===========================================================
\begin{proof}
  We have the function $f(x,y)=(0,0)\in U$ as we can just let $a=b=c=0$ then $f(x,y)=(0x+0y,0x+0y)=(0,0)$. Now suppose that we have two different vectors $f_1,f_2$ of $U$ where $f_1(x,y)=(a_1x+b_1y,b_1x+c_1y)$ and $f_2(x,y)=(a_2x+b_2y,b_2x+c_2y)$ where $a_i,b_i\in \K$. Then $f_1(x,y)+f_2(x,y)=(a_1x+b_1y,b_1x+c_1y)+(a_2x+b_2y,b_2x+c_2y)=\left((a_1+a_2)x+(b_1+b_2)y, (b_1+b_2)x+(c_1+c_2)y\right)$ now as $(a_1+a_2),(b_1+b_2),(c_1+c_2)\in \K$ we get $f_1+f_2\in U$. Now let $\lambda \in \K$ then $\lambda f_1(x,y)=\lambda(a_1 x+ b_1 y, b_1 x+c_1y)=(\lambda a_1 x+ \lambda b_1 y, \lambda b_1 x+\lambda c_1y)$ and as $\lambda a_1, \lambda b_1, \lambda c_1\in \K$ we have $\lambda f_1 \in U$. Therefore it satisfies the three step subspace test hence it is a subspace. 

\end{proof}


\newpage


%===========================================================
% PROBLEM 4
%===========================================================
\begin{tcolorbox}
  \textbf{Problem 4.}
  For any linear map $T \in \mathcal{L}(V)$, we say that a subspace $U \subseteq V$ is an \textbf{invariant subspace of \bm{$T$}} if and only if $T(U) \subseteq U$.
  \vskip1em
  Let $T:\R^3 \rightarrow \R^3$ be the linear map given by
  $$
  T\!\left(\begin{bmatrix} x \\ y \\ z \end{bmatrix}\right) = \begin{bmatrix} 7x-3y+5z \\ 12x-4y+12z \\ -x + y + z \end{bmatrix}.
  $$
  Show that each of the following subspaces of $\R^3$ are invariant subspaces of $T$.
  $$
  U_1 = \Span\!\left(\begin{bmatrix} -2 \\ -3 \\ 1 \end{bmatrix}\right)
  \qquad \text{and} \qquad
  U_2 = \Span\!\left(\begin{bmatrix} - 1 \\ 0 \\ 1 \end{bmatrix},\begin{bmatrix}1 \\ 2 \\ 0 \end{bmatrix}\right)
  $$
\end{tcolorbox}
\vskip1em

%===========================================================
% SOLUTION 4
%===========================================================
\begin{proof}

  Assume $T,U_1,U_2$ are as defined above. Let $\vec{v}\in U_1$ then by the definition of span we have for some scalar $\lambda$ that $\vec{v}=\lambda \begin{bmatrix} -2 \\ -3 \\ 1 \end{bmatrix}$ likewise as for any vector $\vec w\in U_2$ there exists two scalars $\alpha,\beta$ such that $\vec w= \alpha \begin{bmatrix} - 1 \\ 0 \\ 1 \end{bmatrix} + \beta \begin{bmatrix}1 \\ 2 \\ 0 \end{bmatrix} $. Then 
  \[
    T(\vec{v})= T\left(\lambda \begin{bmatrix} -2 \\ -3 \\ 1 \end{bmatrix}\right)= \lambda T\left(\begin{bmatrix} -2 \\ -3 \\ 1 \end{bmatrix}\right)=\lambda \begin{bmatrix} -14+9+5 \\ -24+12+12 \\ 2 -3 + 1 \end{bmatrix}=\lambda \begin{bmatrix}
      0\\0\\0
    \end{bmatrix}
  \] 
  As this is just the subspace containing the zero vector we have $T(U_1)\subseteq U_2$.

  Now computing $T(\vec w)$.
  \[
    \alpha T\left(\begin{bmatrix} - 1 \\ 0 \\ 1 \end{bmatrix}\right)+ \beta T\left( \begin{bmatrix}1 \\ 2 \\ 0 \end{bmatrix} \right) =\alpha \begin{bmatrix}
      -7+5\\-12+12\\1+1
    \end{bmatrix}+ \beta\begin{bmatrix}
      7-6\\12 - 8\\-1+2
    \end{bmatrix} =2\alpha \begin{bmatrix}1 \\ 2 \\ 0 \end{bmatrix}+2\beta \begin{bmatrix}
      1\\2\\0
    \end{bmatrix}+ \beta \begin{bmatrix}
      1\\0\\1
    \end{bmatrix}
  \]
  As this is just equal to a linear combination of the basis elements of $U_2$ we have  $T(U_2)\subseteq U_2$. 
\end{proof}


\newpage


%===========================================================
% PROBLEM 5
%===========================================================
\begin{tcolorbox}
  \textbf{Problem 5.}
  Let $U_1, U_2$ be the subspaces from Problem 4. Prove that
  $$\R^3 = U_1 \oplus U_2.$$
\end{tcolorbox}
\vskip1em

%===========================================================
% SOLUTION 5
%===========================================================
\begin{proof}

  Let $(a,b,c)\in \mathbb{R}^3$ then we have the linear combination \[(-b+2a+2c)\langle -2,-3,1\rangle+(b-2a+2c)\langle -1,0,1\rangle+(3a-b+3c)\langle 1,2,0 \rangle=\langle a,b,c\rangle\]
  Therefore we have $\mathbb{R}^3\subseteq U_1+ U_2$. Now the other direction we have have the sum of two subspaces of $\R^3$ therefore $U_1+U_2\subseteq \mathbb{R}^3$ which implies $R^3=U_1+U_2$. 
  Now to show the direct sum. Let $(a,b,c)\in U_1\cap U_2$ then we have for some scalars $\alpha, \beta, \lambda$ that 
  \[
  \alpha(-2,-3,1) =\beta(-1,0,1)+\lambda(1,2,0)
  \]
  

  \begin{equation*}
    \left\{
      \begin{aligned}
       -\alpha &=-\beta +\lambda \\
       -3\alpha &=2\lambda \\
      \alpha &=\beta
      \end{aligned}
    \right.
  \end{equation*}
  Substituting $\beta $ for $\alpha$ in the first equation we get $\lambda =0$ which implies $\alpha=0$ which finally implies $\beta=0$.
  
  Therefore the only vector that is in the intersection is the zero vector. This implies that it is a direct sum. 


\end{proof}


\end{document}