%===========================================================
% do not change this formatting please :-)
%===========================================================
\documentclass[letter,12pt]{article}
\usepackage[margin=1in]{geometry}
\usepackage{amsfonts,amssymb,amsmath,amsthm,bm,enumitem,mathrsfs,colortbl,fancyhdr,tcolorbox,enumitem}
\def\honorcode{\textit{In accordance with the Hokie Honor Code, I affirm that I have neither given nor received unauthorized assistance on this assignment.}}
\fancyhead{}
\fancyhead[L]{NAME: } %Replace "NAME: " WITH YOUR NAME
\fancyhead[C]{MATH 3144 HOMEWORK 09}
\fancyhead[R]{PAGE \thepage}
\fancyfoot{}
\renewcommand{\footrulewidth}{0.4pt}
\fancyfoot[C]{\honorcode}
\date{\today}
%===========================================================
% convenient commands -- feel free to add more as you see fit
%===========================================================
\newcommand{\C}{\mathbb{C}}
\newcommand{\K}{\mathbb{K}}
\newcommand{\Lin}{\mathcal{L}}
\newcommand{\Mat}{\mathcal{M}}
\newcommand{\mbf}[1]{\mathbf{#1}}
\newcommand{\Poly}{\mathcal{P}}
\newcommand{\Bas}{\mathcal{B}}
\newcommand{\Q}{\mathbb{Q}}
\newcommand{\R}{\mathbb{R}}
\newcommand{\dotp}{\boldsymbol{\cdot}}
\newcommand{\Span}{\operatorname{Span}}
\newcommand{\Null}{\operatorname{Null}}
\newcommand{\sgn}{\mathrm{sgn}}


\begin{document}
\pagestyle{fancy}
%===========================================================
%===========================================================


%===========================================================
% PROBLEM 1
%===========================================================
\begin{tcolorbox}
  \textbf{Problem 1.}
  Prove or provide a counterexample: For any $\alpha \in V^{(4)}_{alt}$, the set
  $$U = \left\{ (\mbf{v_1},\mbf{v_2},\mbf{v_3},\mbf{v_4}) \in V^4 \; : \; \alpha(\mbf{v_1},\mbf{v_2},\mbf{v_3},\mbf{v_4})=0 \right\}$$
  is a subspace of $V^4$.
\end{tcolorbox}
\vskip1em

This is not true. 
\begin{proof}
    Let $V=\mathbb{R}^4$ be an $\R$ vector space. Define $\alpha \in \mathbb{R}_{\text{alt}}^{(4)}$ by 
    \[
        \alpha(v_1,v_2,v_3,v_4) = \det 
        \begin{pmatrix}
        v_{1,1} & v_{1,2} & v_{1,3} & v_{1,4} \\
        v_{2,1} & v_{2,2} & v_{2,3} & v_{2,4} \\
        v_{3,1} & v_{3,2} & v_{3,3} & v_{3,4} \\
        v_{4,1} & v_{4,2} & v_{4,3} & v_{4,4} \\    
        \end{pmatrix}
    \]
    Which is an alternating $4$-multilinear form by Theorem 9.4.5. 
    
    Now take the two vectors of $V^4$ (I will put them in the form of a matrix for clarity).
    \[
    v_1=\begin{pmatrix}
        1& 1& 0 & 0\\
        1& 1& 0 & 0\\
        1& 1& 1 & 0\\
        0& 0& 0 & 0\\
    \end{pmatrix}
    \quad \text{and} \quad
    v_2=\begin{pmatrix}
        0& -1& 0 & 0\\
        -1& 0& 0 & 0\\
        -1& -1& 0 & 0\\
        0& 0& 0 & 1\\
    \end{pmatrix}
    \]
    The fact that each of these vectors have a zero determinant is based off the fact that each contain a column of all $0$s hence any permutation will have a $0$ in the product. 
    But 
    \[
        \alpha (v_1+v_2) = \det\begin{pmatrix}
        1& 0& 0 & 0\\
        0& 1& 0 & 0\\
        0& 0& 1 & 0\\
        0& 0& 0 & 1\\
        \end{pmatrix}=1
    \]
    We have this is a diagonal matrix hence by problem $3$ we have it's determinant is $1$.
 
    
\end{proof}

\newpage


%===========================================================
% PROBLEM 2
%===========================================================
\begin{tcolorbox}
  \textbf{Problem 2.} Let $A = \begin{pmatrix} A_{1,1} & A_{1,2} & A_{1,3} \\  A_{2,1} & A_{2,2} & A_{2,3} \\  A_{3,1} & A_{3,2} & A_{3,3} \end{pmatrix}.$ Let's compute the determinant from the definition, regarding it as a map $\det: \K^3 \rightarrow \K$ given by
  $$\det(\mbf{v_1},\mbf{v_2},\mbf{v_3}) = \sum_{\substack{\text{perm.} \\ \sigma}} \sgn(\sigma) A_{\sigma(1),1}\cdots A_{\sigma(3),3}.$$
  \begin{enumerate}
  \item Fill out the table to describe all permutations of the triple $(1,2,3)$, and find their signs.
  \item Compute $\det(A)$.
  \end{enumerate}
\end{tcolorbox}
\vskip1em

%===========================================================
% SOLUTION 2
%===========================================================
Using the equations 
\begin{enumerate}
\item Fill out the table below.
  \begin{center}
    \begin{tabular}{|r|p{0.5\textwidth}|c|}
      \hline
      \textbf{Permutation $\bm{\sigma}$} & \textbf{Explicit Description} & \textbf{$\bm{\text{sgn}(\sigma)}$} \\ \hline
      $\sigma_1 = \rm{Id}$               & $(1,2,3) \mapsto (1,2,3)=(1)(2)(3)$     & $1$                                \\ \hline
      $\sigma_2$                         & $(1,2,3) \mapsto (2,1,3)=(12)(3)$             &  $-1$                                  \\ \hline
      $\sigma_3$                         & $(1,2,3) \mapsto (3,2,1)=(13)(2)$             &     $-1$                               \\ \hline
      $\sigma_4$                         & $(1,2,3) \mapsto (1,3,2)=(23)(1)$             &        $-1$                            \\ \hline
      $\sigma_5$                         & $(1,2,3) \mapsto (3,1,2)=(132)$             &             $1$                       \\ \hline
      $\sigma_6$                         & $(1,2,3) \mapsto (2,3,1)=(123)$             &               $1$                     \\ \hline
    \end{tabular}
  \end{center}
\item
{
\[
\det(\bf{v_1,v_2,v_3})=\sum_{i=1}^6 \sgn(\sigma_i) A_{\sigma_i(1),1}A_{\sigma_i(2),2}A_{\sigma_i(3),3}
\]

\[
    =A_{1,1}A_{1,2}A_{1,3}-A_{2,1}A_{1,2}A_{3,3}-A_{3,1}A_{2,2}A_{1,3}-A_{1,1}A_{3,2}A_{2,3}+A_{3,1}A_{1,2}A_{2,3}+A_{2,1}A_{3,2}A_{1,3}
\]

}
\end{enumerate}

\newpage


%===========================================================
% PROBLEM 3
%===========================================================
\begin{tcolorbox}
  \textbf{Problem 3.}
  Suppose $A = \begin{pmatrix} A_{1,1} & * & \cdots & * \\ 0 & A_{2,2} & \cdots & * \\ 0 & 0 & \ddots & \vdots \\ 0 & 0 & 0 & A_{n,n} \end{pmatrix}$. Then
  $$\det(A) = A_{1,1} A_{2,2} \cdots A_{n,n}.$$
  {\footnotesize It may be helpful to acknowledge that $A$ can be described by the fact that $A_{i,j} = 0$ whenever $j < i$.}
\end{tcolorbox}
\vskip1em

\begin{proof}
    Using the fact that $A_{i,j}=0$ whenever $j<i$. It suffices to show for any permutation other then the identity we have some $\sigma(i)<i$. 

    Assume that there exists a permutation of $n$ elements $\sigma\not = \text{Id}$ such that $\sigma(i)\geq i$ for all $i$. Then $\sigma(n)=\sigma(n)$ (because $n$ is the max of the $n$ elements ) based on the fact that this is a bijection we get that $\sigma(n-1)=n-1$. Continuing this process we get that $\sigma(i)=i$ for all $i\in \{1,...,n\}$ which contradicts the fact that $\sigma\not = \text{Id}$. Hence we have that $\sigma(i)<i$ for some $i\in \{1,...,n\}$. 
    
    Then all the terms in the sum are $0$ except for the identity permutation. Hence we get $\det(A)=A_{\text{Id}(1),1},...,A_{\text{Id}(n),n}=A_{1,1},...,A_{n,n}$

\end{proof}

\end{document}
