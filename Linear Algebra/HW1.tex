
%===========================================================
% do not change this formatting please :-)
%===========================================================
\documentclass[letter,12pt]{article}
\usepackage[margin=1in]{geometry}
\usepackage{amssymb,amsmath,amsthm,mathrsfs,colortbl,fancyhdr,tcolorbox,enumitem}
\def\honorcode{\textit{In accordance with the Hokie Honor Code, I affirm that I
have neither given nor received unauthorized assistance on this assignment.}}
\fancyhead{}
\fancyhead[L]{Collin McDevitt} %Replace "NAME: " WITH YOUR NAME
\fancyhead[C]{MATH 3144 HOMEWORK 01}
\fancyhead[R]{PAGE \thepage}
\fancyfoot{}
\renewcommand{\footrulewidth}{0.4pt}
\fancyfoot[C]{\honorcode}
\date{\today}
%===========================================================
% convenient commands -- feel free to add more as you see fit
%===========================================================
\newcommand{\C}{\mathbb{C}}
\newcommand{\K}{\mathbb{K}}
\newcommand{\Poly}{\mathcal{P}}
\newcommand{\Q}{\mathbb{Q}}
\newcommand{\R}{\mathbb{R}}
\newcommand{\dotp}{\boldsymbol{\cdot}}
\begin{document}
\pagestyle{fancy}
%===========================================================
%===========================================================
%===========================================================
% PROBLEM 1
%===========================================================
\begin{tcolorbox}
    \textbf{Problem 1.} Determine (with brief explanation only; no proof required)
    whether or not each of the following subsets is a subspace of $\R^3$.
    \begin{enumerate}[label=\textbf{\alph{*}.}]
        \item $\left\{(x_1,x_2,x_3) \in \R^3 \: : \: x_1 + 2x_2 + 3x_3 = 0\right\}$
        \item $\left\{(x_1,x_2,x_3) \in \R^3 \: : \: x_1x_2x_3 = 0\right\}$
        \item $\left\{(x_1,x_2,x_3) \in \R^3 \: : \: x_1 = x_3\right\}$
    \end{enumerate}
\end{tcolorbox}
\vskip1em
%===========================================================
% SOLUTION 1
%===========================================================
\begin{enumerate}[label=\textbf{\alph{*}.}]
    \item Yes: This is a subsets as the equation being equal to zero would imply any two added together would be equal to zero and any being multiplied by a scalar would be equal to zero as well.
    \item No: Consider the two vectors $(1,0,0),(0,1,1)\in \mathbb{R}^3$ both would be in the set but adding them together would give $(1,1,1)$ which is not in the set.
    \item Yes: As the set would be equal to $\{(x,y,x)\in \mathbb{R}^3\}$ any two added together would be of the form $(a,b,a)+(c,d,c)=(a+c,b+d,a+c)$. As scalar multiplication is distributed the 1st and last elements of any vector would still be equal.
\end{enumerate}
\newpage
%===========================================================
% PROBLEM 2
%===========================================================
\begin{tcolorbox}
    \textbf{Problem 2.} Let $V$ be an arbitrary $\K$-vector space and $U_1, U_2$ be
    arbitrary subspaces of $V$. Prove or disprove each of the following.
    \begin{enumerate}[label=(\alph*)]
        \item $U_1 \cap U_2$ is a subspace of $V$.
        \item $U_1 \cup U_2$ is a subspace of $V$.
    \end{enumerate}
\end{tcolorbox}
\vskip1em
%===========================================================
% SOLUTION 2
%===========================================================
\begin{enumerate}[label=(\alph*)]
    \item
          \begin{proof}
              Suppose $V$ is an arbitrary $\mathbb{K}-$vector space and $U_1,U_2$ are arbitrary subspaces of $V$. Consider the set $U_1\cap U_2$ as both subspaces contain the zero vector we have that $\vec{0}\in U_1\cup U_2$. Now $\forall \vec{x},\forall \vec{y}\in U_1 \cap U_2$ we have $\vec{x},\vec{y}\in U_1$ and $\vec{x},\vec{y}\in U_2$ as $U_1$ is a subspace we have $\vec{x}+\vec{y}\in U_1$ by the same reasoning $\vec{x}+\vec{y}\in U_2$ hence $\vec{x}+\vec{y}\in U_1\cap U_2$. Let $\lambda \in \mathbb{K}$ and $\vec{v} \in U_1 \cap U_2$ as $U_1,U_2$ are both subspaces we have $\lambda \vec{x}\in U_1$ and $\lambda \vec{x}\in U_2$ hence $\lambda \vec{x}\in U_1\cap U_2$.
          \end{proof}
    \item
          \begin{proof}[Disproof]
              Let $V=\mathbb{R}^2$ be a $\mathbb{R}$-vector space. Consider the two sets $U_1=\{(x,0)\in \mathbb{R}^2\}$ and $U_2=\{(0,x)\in \mathbb{R}^2\}$ these are both subspaces of $\mathbb{R}^2$ however $U_1\cup U_2=\{(x,0)\in \mathbb{R}^2 \;\text{or}\; (0,y)\in \mathbb{R}^2\}$. Adding the two vectors $(1,0),(0,1)\in U_1\cup U_2$ we have $(1,0)+(0,1)=(1,1)$ but $(1,1)\not \in \ U_1 \cup U_2$.
          \end{proof}
\end{enumerate}
\newpage
%===========================================================
% PROBLEM 3
%===========================================================
\begin{tcolorbox}
    \textbf{Problem 3.} Given two vectors $\mathbf{u}=(x_1,\ldots,x_n)$ and $\
        \mathbf{v}=(v_1,\ldots,v_n)$ in $\R^n$, recall that the \textit{dot product} is
    defined as
    \begin{align*}
        \mathbf{u}\cdot \mathbf{v}=x_1 y_1 + \cdots + x_n y_n.
    \end{align*}
    Let $U$ and $N$ be the following subspaces of $\R^3$:
    \begin{align*}
        U & =\left\{ (x,y,x+y) \: : \: x,y \in \R\right\}                                \\
        N & = \left\{ \mathbf{v} \in \R^3 \: : \: \mathbf{u} \dotp \mathbf{v} = 0 \text{
            for every } \mathbf{u} \in U \right\}
    \end{align*}
    Find the missing vector entry that makes the following statement true (and
    provide a proof):
    $$N = \left\{ (x,x,-x) \in \R^3 \: : \: x \in \R \right\}.$$
\end{tcolorbox}
\vskip1em
%===========================================================
% SOLUTION 3
%===========================================================
\begin{proof}
    The set should be $N = \left\{ (x,x,-x) \in \R^3 \: : \: x \in \R \right\}.$
    Let $(x,y,x+y)\in U$ and let $(a,a,-a)\in N$ the $(x,y,x+y)\cdot (a,a,-a)=ax+ay+-a(x+y)=ax+ay+-ax+-ay=0$. Now to show that $N$ is a subspace. We have that $(0,0,0)\in N$. Now let $(x,x,-x),(y,y,-y)\in N$ adding the vectors yields $(x,x,-x)+(y,y,-y)=\big(x+y,x+y,-(x+y)\big)$ as $x+y\in \mathbb{R}$ we get that $\big(x+y,x+y,-(x+y)\in \mathbb{R}\big)$. Let $\lambda \in \mathbb{R}$ and $(x,x,-x)\in N$ as $\lambda x\in \mathbb{R}$ we get $(\lambda x,\lambda x, -\lambda x)\in N$ therefore $N$ is a subspace.


\end{proof}
\newpage
%===========================================================
% PROBLEM 4
%===========================================================
\begin{tcolorbox}
    \textbf{Problem 4.}
    Let $U_1$ and $U_2$ be the following subspaces of $\Q^4$:
    \begin{align*}
        U_1 & =\left\{ (x,y,z,y) \: : \: x,y,z \in \Q \right\} \\
        U_2 & = \left\{ (0,x,0,-x) \: : \: x \in \Q \right\}
    \end{align*}
    Prove that $\Q^4 = U_1 \oplus U_2$.
\end{tcolorbox}
\vskip1em
%===========================================================
% SOLUTION 4


%===========================================================
\begin{proof}
    Let $U_1$ and $U_2$ be as defined above. Let $(x,y+a,z,y-a)\in U_1 \oplus U_2$ then we have $x,y+a,z,y-a\in \mathbb{Q}$ which implies $(x,y+a,z,y-a)\in\mathbb{Q}^4$ so we have $U_1 \oplus U_2 \subseteq \mathbb{Q}^4$. Now let $(a,b,c,d)\in \mathbb{Q}^4$ consider the two vectors $(a, \frac{b+d}{2},c,\frac{b+d}{2})\in U_1$ and $(0,\frac{b-d}{2},0,-\frac{b-d}{2})\in U_2$ adding these two vectors we get $$(a, \frac{b+d}{2},c,\frac{b+d}{2})+(0,\frac{b-d}{2},0,-\frac{b-d}{2})=(a,b,c,d)$$
    therefore we have $\mathbb{Q}^4\subseteq U_1 \oplus U_2$ which implies $\mathbb{Q}^4=U_1\oplus U_2$.


\end{proof}
\newpage
%===========================================================
% PROBLEM 5
%===========================================================
\begin{tcolorbox}
    \textbf{Problem 5.} Recall that $\Poly_m(\K)$ is the $\K$-vector space of
    polynomials of degree (at most) $m$ and with coefficients in $\K$.
    \begin{enumerate}[label=\textbf{\alph*.}]
        \item Find a list of four distinct, nonzero polynomials that span $\mathcal{P}_2(\R)$.
        \item Prove that the polynomials found in part (a) is linearly dependent.
    \end{enumerate}
\end{tcolorbox}
\vskip1em
%===========================================================
% SOLUTION 5
%===========================================================
\begin{enumerate}[label=\textbf{\alph*.}]
    \item
    \item
          \begin{proof}
          \end{proof}
\end{enumerate}
\end{document}
Annotations
