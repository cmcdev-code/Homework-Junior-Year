
%===========================================================
% do not change this formatting please :-)
%===========================================================
\documentclass[letter,12pt]{article}
\usepackage[margin=1in]{geometry}

\usepackage{amssymb,amsmath,amsthm,bm,mathrsfs,colortbl,fancyhdr,tcolorbox,enumitem}
\def\honorcode{\textit{In accordance with the Hokie Honor Code, I affirm that I
have neither given nor received unauthorized assistance on this assignment.}}
\fancyhead{}
\fancyhead[L]{Collin McDevitt } %Replace "NAME: " WITH YOUR NAME
\fancyhead[C]{MATH 3144 HOMEWORK 02}
\fancyhead[R]{PAGE \thepage}
\fancyfoot{}
\renewcommand{\footrulewidth}{0.4pt}
\fancyfoot[C]{\honorcode}
\date{\today}
%===========================================================
% convenient commands -- feel free to add more as you see fit
%===========================================================
\newcommand{\C}{\mathbb{C}}
\newcommand{\K}{\mathbb{K}}
\newcommand{\Poly}{\mathcal{P}}
\newcommand{\Q}{\mathbb{Q}}
\newcommand{\R}{\mathbb{R}}
\newcommand{\dotp}{\boldsymbol{\cdot}}
\newcommand{\Span}{\operatorname{Span}}
\begin{document}
\pagestyle{fancy}
%===========================================================
%===========================================================
%===========================================================
% PROBLEM 1
%===========================================================
\begin{tcolorbox}
\textbf{Problem 1.}
\begin{enumerate}[label=(\alph*)]
\item Considering $\C$ as a $\R$-vector space, find a basis for $\C$.
\item Considering $\C$ as a $\C$-vector space, find a basis for $\C$.
\end{enumerate}
\end{tcolorbox}
\vskip1em
%===========================================================
% SOLUTION 1
%===========================================================
\begin{enumerate}[label=(\alph*)]
\item
\begin{proof}
    A basis for $\mathbb{C}$ as a $\R$-vector space is $B=\{1, i\}$. This is shown to be a basis by the following: Let $a+bi\in \mathbb{C}$ then consider the linear combination $a(1)+b(i)=a+bi$ as an arbitrary element of $\C$ is a linear combination of the elements of $B$ we have $\Span(B)=\mathbb{C}$. Lastly to show linear independence of $B$ consider the linear combination $a(1)+b(i)=0=0+0i$ as two complex numbers are equal if and only if their real parts are equal and imaginary parts are equal we get $a=0$ and $b=0$. Thus $B$ is a basis for $\C$ as a $\R$-vector space.
\end{proof}
\item
\begin{proof}
    A basis for $\mathbb{C}$ as a $\C$-vector space is given by $B=\{1+i\}$. Let $a+bi\in \C$ then consider the linear combination $(\frac{a+b}{2}+\frac{b-a}{2}i)(1+i)=\frac{a+b}{2}+\frac{a+b}{2}i+\frac{b-a}{2}i-\frac{b-a}{2}=a+bi$ as an arbitrary element of $\C$ is a linear combination of the element of $B$ we have $\Span(B)=\mathbb{C}$. \textit{Note that $\frac{a+b}{2}+\frac{b-a}{2}i\in C$ is a scalar as this is a $\C$-vector space}. Lastly to show linear independence of $B$ consider the linear combination $a+bi\in \C$ and $1+i\in B$ we have $(a+bi)(1+i)=0$ if and only if $a+bi=0$ as $\C$ is a field hence no zero divisors. 
\end{proof}
\end{enumerate}
\newpage
%===========================================================
% PROBLEM 2
%===========================================================
\begin{tcolorbox}
\textbf{Problem 2.} Let $V$ be a $\K$-vector space, and suppose that $S_1, S_2$
are subsets of $V$ satisfying the following conditions:
\begin{itemize}
\item $S_1$ and $S_2$ are both finite.
\item $S_1 \cap S_2 = \emptyset$.
\item $S_1 \cup S_2$ is a linearly independent set.
\end{itemize}
\vskip1em
\begin{enumerate}[label=(\alph*)]
\item Prove that $\Span(S_1 \cup S_2) = \Span(S_1) \oplus \Span(S_2).$
\item What would change about the claim in (a) if $S_1 \cup S_2$ was not assumed
to be linearly independent?
\end{enumerate}
\end{tcolorbox}
\vskip1em
%===========================================================
% SOLUTION 2
%===========================================================
\begin{enumerate}[label=(\alph*)]
\item
\begin{proof}
    Assume that $V$ is a $\mathbb{K}$-vector space, and that $S_1,S_2$ are as described.
    First I will show that $\Span(S_1)+\Span(S_2)$ is a direct sum. As $\Span(S_1),\Span(S_2)$ are both vector spaces we have that $\vec{0}\in \Span(S_1)+\Span(S_2)\not = \emptyset$. Now assume that $\vec{v}\in \Span(S_1)\cap \Span(S_2)$ then $\vec{v}\in \Span(S_1)$ and $\vec{v}\in \Span(S_2)$ so $v=k_1\vec{s_1}+...+k_n\vec{s_n}$ where $k_i\in \K$ and $\vec{s}\in S_1$ and $v= c_1\vec{u_1}+...+c_n\vec{u_n}$ where $c_i\in \K$ and $\vec{u_i}\in S_2$. Then $c_1\vec{u_1}+...+c_n\vec{u_n}+-(k_1\vec{s_1}+...+k_n\vec{s_n})=c_1\vec{u_1}+...+c_n\vec{u_n}+(-k_1\vec{s_1})+...+(-k_n\vec{s_n})=\vec{0}$ but as $S_1\cup S_2$ is linearly independent and the previous equation is a linear combination of $S_1\cup S_2$ we have that for each $c_i,k_i\in \K$ that $c_i=k_i=0$ hence only $\vec{0}\in \Span(S_1)\cap \Span(S_2)$ which implies $\Span(S_1)+ \Span(S_2)$ is a direct sum. 
    
    Now suppose $\vec{v}\in \Span(S_1\cup S_2)$ then $\vec{v}= k_1\vec{s_1}+...+k_n\vec{s_n}+c_1\vec{u_1}+...+c_n\vec{c_n}$ where $k_i ,c_i\in \K$ and $\vec{s_i} \in S_1$ and $\vec{u_i}\in S_2$. As $k_1\vec{s_1}+...+k_n\vec{s_n}\in \Span(S_1)$ and $c_1\vec{u_1}+...+c_n\vec{u_n}\in \Span(S_2)$ we have that $\vec{v}\in \Span(S_1)\oplus \Span(S_2)$. Thus $\Span(S_1\cup S_2)\subseteq \Span(S_1) \oplus \Span(S_2)$. 
    
    Let $\vec{v}\in \Span(S_1)\oplus \Span(S_2)$ then $\vec{v}= k_1\vec{s_1}+...+k_n\vec{s_n}+c_1\vec{u_1}+...+c_n\vec{c_n}$ where $k_i ,c_i\in \K$ and $\vec{s_i} \in S_1$ and $\vec{u_i}\in S_2$. Then we have as this is just a linear combination of the elements of $S_1\cup S_2$ we have that $\vec{v}\in \Span(S_1 \cup S_2)$ therefore $\Span(S_1) \oplus \Span(S_2)\subseteq \Span(S_1 \cup S_2)$ which implies $\Span(S_1 \cup S_2)= \Span(S_1) \oplus \Span(S_2)$
\end{proof}
\item Then $\Span(S_1)+\Span(S_2)$ would no longer be a direct sum. But the equation would still be true if you replaced '$\oplus$' with '$+$' i.e. $\Span(S_1 \cup S_2) = \Span(S_1) + \Span(S_2)$.
\end{enumerate}
\newpage
%===========================================================
% PROBLEM 3
%===========================================================
\begin{tcolorbox}
\textbf{Problem 3.} Let $V = \Q^4$, considered as a $\Q$-vector space, and let
$U$ be the subspace
$$U = \Span\!\big(\mathbf{u_1}=(1,-1,2,1),\,\mathbf{u_2}=(2,-3,6,3)\,\big).$$
Extend the set $\left\{\mathbf{u_1},\mathbf{u_2}\right\}$ into a basis for $V$.
That is, find two vectors $\mathbf{v_1},\mathbf{v_2} \in V$ so that
$$\left\{\mathbf{u_1},\mathbf{u_2},\mathbf{v_1},\mathbf{v_2}\right\}$$
is a basis for $V$.
\end{tcolorbox}
\vskip1em
%===========================================================
% SOLUTION 3
%===========================================================
\begin{proof}

    Adding the two vectors $(0,0,1,0)$ and $(0,0,0,1)$ will make a basis. 

    Let $\langle a,b,c,d\rangle \in V$ and consider the linear combination 
    \[
    (3a+2b)\langle 1,-1,2,1\rangle+ (-a-b)\langle 2,-3,6,3\rangle +(c+2b)\langle 0,0,1,0\rangle
    + (d+b)\langle 0,0,0,1\rangle = 
    \]
    \[
    \langle (3a+2b) + (-a-b)2, -(3a+2b)-3(-a-b),2(3a+2b)+6(-a-b)+(c+2b), (3a+2b)+3(-a-b)+(d+b)  \rangle =
    \]

    $$
    \langle 3a+2b-2a-2b,-3a-2b+3a+3b,6a+4b-6a-6b+c+2b,3a+2b-3a-3b+d+b\rangle = 
    $$

    \[\langle a,b,c,d\rangle \]

    \vspace{6mm}

    Therefore we have an arbitrary element of $V$ as a linear combination of the vectors $\mathbf{u_1},\mathbf{u_2},\mathbf{v_1},\mathbf{v_2}$. Lastly to show linear independence. Consider the linear combination $a,b,c,d\in \Q$
        \[
        a(1,-1,2,1)+b(2,-3,6,3)+c(0,0,1,0)+d(0,0,0,1)=0
        \]
        \[
        (a+2b, -a-3b, 2a+6b+c, a+3b+d) = (0,0,0,0)
        \]

    From this we get the system of equations 
    \begin{equation*}
        \begin{cases}
            a+2b=0\\
            -a-3b=0\\
            2a+6b+c=0\\
            a+3b+d=0
        \end{cases}
    \end{equation*}

    From the first two equations we get $a+2b-a-3b=0$ which implies $b=0$ substituting in $0$ for $b$ we get $a+2\cdot 0=0$ which implies $a=0$. Replacing $a,b$ in the bottom equations we get that $c=0$ and $d=0$ as well. As the scalars where chosen arbitrarily we have that the set is linear independent hence it is a basis for $V$.



\end{proof}
\newpage
%===========================================================
% PROBLEM 4
%===========================================================
\begin{tcolorbox}
\textbf{Problem 4.}
Let $V=\Poly_3(\R)$ be the $\R$-vector space of polynomials of degree $3$ or
less. Let $U$ be the subspace (you can take this for granted)
$$U = \left\{ p(x) \in \Poly_3(\R) \: : \: p'(7)=0\right\}$$
where $p'(x)$ is the derivative of $p(x)$ and $p'(7)$ is the derivative evaluated
at $x=7$.
Find a basis for $U$.
\end{tcolorbox}
\vskip1em
%===========================================================
% SOLUTION 4
%===========================================================
\begin{proof}

Consider the set $B=\left\{\frac{x^3}{147}+\frac{x^2}{14}-2x,\;\frac{x^2}{14}-x,\;1\right\}$. First I will show the linear independent of $B$. Consider the linear combination $a\left(\frac{x^3}{147}+\frac{x^2}{14}-2x\right)+b\left(\frac{x^2}{14}-x\right)+c(1)=0$ then as the right of the equation has no $x^3$ and the coefficient on $x^3$ is $\frac{a}{147}$ we have that $a=0$.  Likewise we have that $b=0$ as the right of the equation has no $x^2$ lastly we have that $c=0$. Now we have that $\dim(U)\leq \mathcal{P}_3(\mathbb{R})=4$. If $\dim(U)=4$ then we would be able to add another vector to $B$ and have $B$ still be linearly independent denote this vector by $ax^3+bx^2+cx+d$ where $a,b,c,d\in \mathbb{R}$ however consider the linear combination $$ax^3+bx^2+cx+d-a147\left(\frac{x^3}{147}+\frac{x^2}{14}-2x\right)+-(14b-a147)(\frac{x^2}{14}-x)+d\cdot 1 =$$
$$ax^3+bx^2+cx+d-ax^3- \frac{a147x^2}{14} +294ax-14bx^2+14bx +\frac{147ax^2}{14}-a147x-d=$$
$$cx+294ax+14bx=x(c+294a+14b)$$ 
If $c+294a+14b=0$ then the set would not be linearly independent. If $c+294a+14b\not =0$ then taking the derivative and evaluating at $x=7$ would give a non-zero value. This implies that no such vector exists and that $\dim(U)=3$. Thus $B$ is a basis for $U$ by \textbf{Theorem 2.39}.
\end{proof}
\newpage
%===========================================================
% PROBLEM 5
%===========================================================
\begin{tcolorbox}
\textbf{Problem 5.} The classical ``Inclusion-Exclusion Principle'' states that,
for two finite sets $A_1, A_2$, the cardinality of the union satisfies:
$$|A_1 \cup A_2| = |A_1| + |A_2| - |A_1 \cap A_2|.$$
Notice that we have a similar formula for vector spaces $V_1, V_2$:
$$\dim(V_1 + V_2) = \dim(V_1) + \dim(V_2) - \dim(V_1 \cap V_2).$$
For three sets, $A_1, A_2, A_3$, the Inclusion-Exclusion Principle says
\begin{align*}
|A_1 \cup A_2 \cup A_3| =& |A_1| + |A_2| + |A_3| \\
& - |A_1 \cap A_2| - |A_1 \cap A_3| - |A_2 \cap A_3| \\
& + |A_1 \cap A_2 \cap A_3|.
\end{align*}
Give an example showing that, sadly, the following analogous formula does not
hold for vector spaces $V_1, V_2, V_3$:
\begin{align*}
\dim(V_1 + V_2 + V_3) =& \dim(V_1) + \dim(V_2) + \dim(V_3) \\
& - \dim(V_1 \cap V_2) - \dim(V_1 \cap V_3) - \dim(V_2 \cap V_3) \\
& +\dim(V_1 \cap V_2 \cap V_3).
\end{align*}
\vskip0.5em
{\footnotesize \textsc{Hint: consider subspaces of a familiar low-dimensional
vector space.}}
\end{tcolorbox}
\vskip1em

Consider the three $\R$-vector spaces $V_1=\{(a,0):a\in \mathbb{R}\}, V_2 = \{(0,b):b\in \mathbb{R}\},V_3=\{(0,0)\}$ We have $\dim(V_1+V_2+V_3)=2$. However $\dim(V_1)+\dim(V_2)+\dim(V_3)-\dim(V_1\cap V_2)-\dim(V_1\cap V_3)-\dim(V_2\cap V_3)+\dim(V_1\cap V_2\cap V_3)=1+1+0-1-0-0+0=1$. Thus the formula does not hold.


\end{document}
Annotations
