%===========================================================
% do not change this formatting please :-)
%===========================================================
\documentclass[letter,12pt]{article}
\usepackage[margin=1in]{geometry}
\usepackage{amsfonts,amssymb,amsmath,amsthm,bm,enumitem,mathrsfs,colortbl,fancyhdr,tcolorbox,enumitem}
\def\honorcode{\textit{In accordance with the Hokie Honor Code, I affirm that I have neither given nor received unauthorized assistance on this assignment.}}
\fancyhead{}
\fancyhead[L]{NAME: } %Replace "NAME: " WITH YOUR NAME
\fancyhead[C]{MATH 3144 HOMEWORK 10}
\fancyhead[R]{PAGE \thepage}
\fancyfoot{}
\renewcommand{\footrulewidth}{0.4pt}
\fancyfoot[C]{\honorcode}
\date{\today}
%===========================================================
% convenient commands -- feel free to add more as you see fit
%===========================================================
\newcommand{\C}{\mathbb{C}}
\newcommand{\K}{\mathbb{K}}
\newcommand{\Lin}{\mathcal{L}}
\newcommand{\Mat}{\mathcal{M}}
\newcommand{\mbf}[1]{\mathbf{#1}}
\newcommand{\Poly}{\mathcal{P}}
\newcommand{\Bas}{\mathcal{B}}
\newcommand{\Q}{\mathbb{Q}}
\newcommand{\R}{\mathbb{R}}
\newcommand{\dotp}{\boldsymbol{\cdot}}
\newcommand{\Span}{\operatorname{Span}}
\newcommand{\Null}{\operatorname{Null}}
\newcommand{\sgn}{\mathrm{sgn}}


\begin{document}
\pagestyle{fancy}
%===========================================================
%===========================================================


%===========================================================
% PROBLEM 1
%===========================================================
\begin{tcolorbox}
  \textbf{Problem 1.}
  Recall the following useful technique for computing the determinant of a matrix.
  \begin{center}
    \fbox{\parbox{0.8\textwidth}{
        \textbf{Theorem. (Cofactor Expansion, Laplace).} Let $A$ be an $n \times n$ matrix and let $M_{i,j}$ denote the $(n-1)\times(n-1)$ submatrix obtained by deleting Row $i$ and Column $j$ from $A$. The determinant of an $n \times n$ matrix $A$ can be computed \emph{along the $i^\text{th}$ row} as the sum
        \begin{align*}
          \det A = \sum_{\text{col. }j} (-1)^{i+j} A_{i,j} \det(M_{i,j})
        \end{align*}
        or \emph{along the $j^\text{th}$ column} as the sum
        \begin{align*}
          \det A = \sum_{\text{row. }i} (-1)^{i+j} A_{i,j} \det(M_{i,j})
        \end{align*}
    }}
  \end{center}
    Let $T \in \Lin(\C^4)$ be an operator with matrix (in the standard basis) given by
    $$A = \begin{pmatrix} 1 & -1 & 1 & -2 \\ 0 & 0 & 0 & -1 \\ -1 & 1 & -1 & 2 \\ 0 & 1 & 0 & 2 \end{pmatrix}.$$
    \begin{enumerate}
    \item Find the characteristic polynomial for $A$.
    \item Find a eigenvalues for $A$.
    \item Find basis $\mathcal{G}$ for $\C^4$ so that $\Mat(T,\mathcal{G})$ is upper triangular with eigenvalues along the diagonal.
    \item Is $A$ diagonalizable? Why or why not?
    \end{enumerate}
\end{tcolorbox}
\vskip1em

%===========================================================
% SOLUTION 1
%===========================================================


\newpage


%===========================================================
% PROBLEM 2
%===========================================================
\begin{tcolorbox}
  \textbf{Problem 2.} Let $T \in \Lin(\R^4)$ be an operator whose matrix (in the standard basis) is given by
  $$ \begin{pmatrix} -2 & 1 & 0 & 3 \\ -2 & 0 & 1 & 1 \\ 0 & 0 & 1 & -1 \\ 0 & 0 & 1 & 1 \end{pmatrix}$$
  Find a basis $\mathcal{B}$ for $\R^4$ so that $\Mat(T,\mathcal{B})$ is block-diagonal. That is,
  $$\Mat(T,\mathcal{B}) = \begin{pmatrix} * & * & 0 & 0 \\ * & * & 0 & 0 \\ 0 & 0 & * & * \\ 0 & 0 & * & * \end{pmatrix}$$
  \vskip0.5em
  \hrule
  \vskip0.5em
      {\footnotesize
        \textsc{Hint:} Given $S \in \Lin(\C^2)$ with 
        $$\Mat\!\left(S,\left\{\mathbf{b_1},\mathbf{b_2}\right\}\right) = \begin{pmatrix} k e^{i \theta} & 0 \\ 0 & k e^{-i \theta} \end{pmatrix}$$
        then
        $$\Mat\!\left(S,\left\{\operatorname{Re}(\mathbf{b_1}),\operatorname{Im}(\mathbf{b_1})\right\}\right) = \begin{pmatrix} k \cos \theta & -k \sin \theta \\ k \sin \theta & k \cos \theta\end{pmatrix}.$$
      }
\end{tcolorbox}
\vskip1em

%===========================================================
% SOLUTION 2
%===========================================================


\end{document}
