%===========================================================
% do not change this formatting please :-)
%=======================`'====================================
\documentclass[letter,12pt]{article}
\usepackage[margin=1in]{geometry}
\usepackage{amsfonts,amssymb,amsmath,amsthm,bm,enumitem,mathrsfs,colortbl,fancyhdr,tcolorbox,enumitem}
\def\honorcode{\textit{In accordance with the Hokie Honor Code, I affirm that I have neither given nor received unauthorized assistance on this assignment.}}
\fancyhead{}
\fancyhead[L]{Collin McDevitt} %Replace "NAME: " WITH YOUR NAME
\fancyhead[C]{MATH 3144 HOMEWORK 10}
\fancyhead[R]{PAGE \thepage}
\fancyfoot{}
\renewcommand{\footrulewidth}{0.4pt}
\fancyfoot[C]{\honorcode}
\date{\today}
%===========================================================
% convenient commands -- feel free to add more as you see fit
%===========================================================
\newcommand{\C}{\mathbb{C}}
\newcommand{\K}{\mathbb{K}}
\newcommand{\Lin}{\mathcal{L}}
\newcommand{\Mat}{\mathcal{M}}
\newcommand{\mbf}[1]{\mathbf{#1}}
\newcommand{\Poly}{\mathcal{P}}
\newcommand{\Bas}{\mathcal{B}}
\newcommand{\Q}{\mathbb{Q}}
\newcommand{\R}{\mathbb{R}}
\newcommand{\dotp}{\boldsymbol{\cdot}}
\newcommand{\Span}{\operatorname{Span}}
\newcommand{\Null}{\operatorname{Null}}
\newcommand{\sgn}{\mathrm{sgn}}


\begin{document}
\pagestyle{fancy}
%===========================================================
%===========================================================


%===========================================================
% PROBLEM 1
%===========================================================
\begin{tcolorbox}
  \textbf{Problem 1.}
  Recall the following useful technique for computing the determinant of a matrix.
  \begin{center}
    \fbox{\parbox{0.8\textwidth}{
        \textbf{Theorem. (Cofactor Expansion, Laplace).} Let $A$ be an $n \times n$ matrix and let $M_{i,j}$ denote the $(n-1)\times(n-1)$ submatrix obtained by deleting Row $i$ and Column $j$ from $A$. The determinant of an $n \times n$ matrix $A$ can be computed \emph{along the $i^\text{th}$ row} as the sum
        \begin{align*}
          \det A = \sum_{\text{col. }j} (-1)^{i+j} A_{i,j} \det(M_{i,j})
        \end{align*}
        or \emph{along the $j^\text{th}$ column} as the sum
        \begin{align*}
          \det A = \sum_{\text{row. }i} (-1)^{i+j} A_{i,j} \det(M_{i,j})
        \end{align*}
    }}
  \end{center}
    Let $T \in \Lin(\C^4)$ be an operator with matrix (in the standard basis) given by
    $$A = \begin{pmatrix} 1 & -1 & 1 & -2 \\ 0 & 0 & 0 & -1 \\ -1 & 1 & -1 & 2 \\ 0 & 1 & 0 & 2 \end{pmatrix}.$$
    \begin{enumerate}
    \item Find the characteristic polynomial for $A$.
    \item Find a eigenvalues for $A$.
    \item Find basis $\mathcal{G}$ for $\C^4$ so that $\Mat(T,\mathcal{G})$ is upper triangular with eigenvalues along the diagonal.
    \item Is $A$ diagonalizable? Why or why not?
    \end{enumerate}
\end{tcolorbox}
\vskip1em
1.
Expanding along row 2 we get 

$$\det(A-\lambda\text{Id})=-\lambda\det\begin{pmatrix}
    1-\lambda & 1 &-2\\-1 & -1 - \lambda & 2\\0 & 0 & 2-\lambda
\end{pmatrix}-1\det\begin{pmatrix}
    1-\lambda & -1 & 1 \\ -1 & 1 & -1 - \lambda \\0 & 1 & 0
\end{pmatrix}$$

Choosing row $3$ for each of these determinants we get. 

$$=-\lambda(2-\lambda)((1-\lambda)(-1-\lambda)+1))+((1-\lambda)(-1-\lambda)+1)$$

$$=\big((1-\lambda)(-1-\lambda)+1\big)\big(-\lambda (2-\lambda )+1\big)$$
The left is a difference of squares hence we get $$=\lambda^2(\lambda^2-2\lambda +1)=\lambda^2(\lambda-1)^2$$

2.
The eigenvalues are the roots of the characteristic polynomial. Hence $\lambda=0,1$. 

3. 
First to find the eigenvectors for $\lambda=0$

Solving for $A(a,b,c,d)=(0,0,0,0)$ we get the system of equations \begin{equation}
    \begin{cases}
        a-b+c-2d=0\\
        -d=0\\
        -a+b-c+2d=0\\
        b+2d=0
    \end{cases}
\end{equation}

This immediately shows that $d=0$ from which it follows that $b=0$. This implies that $a=c$ hence $\Span((1,0,-1,0)^t)=E(0,A)$. 

Now for $\lambda=1$ we get the system of equations \begin{equation}
    \begin{cases}
        -b+c-2=0\\-b-d=0\\-a+b-2c+2d=0\\b+d=0
    \end{cases}
\end{equation}

This implies that $b=-d$ and $c=2+b$ subbing these into the 3rd equation we get $-a+b-2(2+b)-2b=0$ which implies $a=-3b-4$. Hence $\Span((-1,-1,1,1)^t)=E(1,A)$.

Now finding a basis to span $\Null A^2$. We have $A^2(a,b,c,d)=(0,0,0,0)$
gives the system of equations \begin{equation}
    \begin{cases}
        2b+9d=0\\
        -1b-2d=0\\
        2b-d=0\\
        2b+3d=0
    \end{cases}
\end{equation}
Hence it is spanned by $\{(1,0,0,0)^t,(0,0,1,0)^t\}$

Doing the same for $\Null(A-\text{Id})^2$ we have that $(A-\text{Id})^2(a,b,c,d)=(0,0,0,0)$ gives the system of equations \begin{equation}
    \begin{cases}
    -a-2c+d=0\\
    2a+3c-d=0
    \end{cases}
\end{equation}

Which gives $a=-c$ and $a=-d$ hence we have that $\{(1,0,-1,-1)^t,(0,1,0,0)^t\}$ spans $\Null(A-\text{Id})^2$. Then we choose the basis $$\mathcal B=\{b_1=((1,0,-1,0)^t,b_2=(1,0,0,0)^t,b_3=(-1,-1,1,1)^t,b_4=(0,1,0,0)^t)\}$$ solving $A(1,0,0,0)^t=(1,0,-1,0)^t=1\cdot b_1$ doing the same for $A(0,1,0,0)^t=(-1,0,1,1)^t=b_4+b_3$ hence we get the matrix $$\Mat(A,\mathcal B)=\begin{pmatrix}
    0 & 1 & 0 & 0\\
    0 & 0 & 0 & 0\\
    0 & 0 & 1 & 1\\
    0 & 0 & 0 & 1
\end{pmatrix}$$
\newpage

4. This is not diagonalizable as the geometric multiplicity of both the eigenvalues values is $1$ while the algebraic is $2$.

%===========================================================
% PROBLEM 2
%===========================================================
\begin{tcolorbox}
  \textbf{Problem 2.} Let $T \in \Lin(\R^4)$ be an operator whose matrix (in the standard basis) is given by
  $$ \begin{pmatrix} -2 & 1 & 0 & 3 \\ -2 & 0 & 1 & 1 \\ 0 & 0 & 1 & -1 \\ 0 & 0 & 1 & 1 \end{pmatrix}$$
  Find a basis $\mathcal{B}$ for $\R^4$ so that $\Mat(T,\mathcal{B})$ is block-diagonal. That is,
  $$\Mat(T,\mathcal{B}) = \begin{pmatrix} * & * & 0 & 0 \\ * & * & 0 & 0 \\ 0 & 0 & * & * \\ 0 & 0 & * & * \end{pmatrix}$$
  \vskip0.5em
  \hrule
  \vskip0.5em
      {\footnotesize
        \textsc{Hint:} Given $S \in \Lin(\C^2)$ with 
        $$\Mat\!\left(S,\left\{\mathbf{b_1},\mathbf{b_2}\right\}\right) = \begin{pmatrix} k e^{i \theta} & 0 \\ 0 & k e^{-i \theta} \end{pmatrix}$$
        then
        $$\Mat\!\left(S,\left\{\operatorname{Re}(\mathbf{b_1}),\operatorname{Im}(\mathbf{b_1})\right\}\right) = \begin{pmatrix} k \cos \theta & -k \sin \theta \\ k \sin \theta & k \cos \theta\end{pmatrix}.$$
      }
\end{tcolorbox}
\vskip1em


First to find the eigenvalues. 

\[
\det\begin{pmatrix}
    -2-\lambda & 1 & 0 & 3\\
    -2 & -\lambda & 1 & 1\\
    0 & 0 & 1-\lambda & -1\\
    0 & 0 & 1 & 1-\lambda
\end{pmatrix}    
\]

Using  cofactor expansion on row $4$. We get 

$$
=\det\begin{pmatrix}
    -2-\lambda & 1 & 3\\
    -2 & -\lambda & 1\\
    0 & 0 & -1
\end{pmatrix}
-(-1-\lambda)\det\begin{pmatrix}
    -2-\lambda & 1 & 0\\
    -2 & -\lambda & 1\\
    0 & 0 & 1-\lambda
\end{pmatrix}$$
$$= -((-2-\lambda)(-\lambda)+2)+(1-\lambda)(1-\lambda)((-2-\lambda)-\lambda+2)$$

$$=((2+\lambda)\lambda +2)+(1-\lambda)(1-\lambda)((2+\lambda)\lambda+2)$$
$$=((2+\lambda)\lambda+2)((1-\lambda)^2+1)$$
$$=(\lambda^2+2\lambda+2)(\lambda^2-2\lambda+2)$$
So $\lambda=1\pm i,-1\pm i$. The eigenvectors are $(1,i,i,1),(1,-i,-i,1),(1-i,2,0,0),(1+i,2,0,0)$. Then taking the basis to be $$\mathcal B=\{b_1=(1,0,0,1),b_2(0,1,1,0),b_3=(1,2,0,0),b_4=(1,0,0,0)\}$$.

We have $$T(1,0,0,1)=(1,-1,-1,1)=b_1-b_2,$$
$$T(0,1,1,0)=(1,1,1,1)=b_1+b_2$$
$$T(1,2,0,0)=(0,-2,0,0)=-b_3+b_4$$
$$T(1,0,0,0)=(-2,-2,0,0)=-b_3-b_4$$

From that we get the matrix $$\Mat(T,\mathcal B)=\begin{pmatrix}
    1 & -1 &0 & 0 \\
    1 & 1& 0& 0\\
    0 & 0 & -1 & 1\\
    0 & 0 & -1 & -1
\end{pmatrix}$$

\end{document}
