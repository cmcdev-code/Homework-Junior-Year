\documentclass{amsart}
\usepackage{amsmath, amssymb, amscd}

\setlength{\textwidth}{6.5in}
\setlength{\textheight}{9in}
\setlength{\topmargin}{-.25in}
\setlength{\evensidemargin}{0in}
\setlength{\oddsidemargin}{0in}

\theoremstyle{plain}
\newtheorem{theorem}{Theorem}[section]
\newtheorem{proposition}[theorem]{Proposition}
\newtheorem{lemma}[theorem]{Lemma}
\newtheorem{corollary}[theorem]{Corollary}

\theoremstyle{definition}
\newtheorem{definition}[theorem]{Definition}
\newtheorem{assumption}[theorem]{Assumption}

\theoremstyle{remark}
\newtheorem{remark}[theorem]{Remark}
\newtheorem{example}[theorem]{Example}
\newtheorem{notation}[theorem]{Notation}

\begin{document}

\section*{Math 4324\ \ \  Topologies }

\subsection*{Hand in Friday, February 2.} Homework due on a class day should be submitted as an email attachment sent to me by the beginning of class.



\vspace{.15in}


\noindent
\subsection*{1.}  Suppose that $\{ U_{\alpha} \} _{\alpha \in A}$ is a collection of sets satisfying the property
\[
    \forall a, b \in A, \ \ \ U_a \bigcap U_b \in \{ U_{\alpha} \} _{\alpha \in A}.
\]

{\bfseries a.} Show that for any finite collection $\{ U_1, . . .  , U_n \}$ of sets from this collection,
${\displaystyle \bigcap _{j = 1} ^n U_j}$ is in the collection.

\begin{proof}
    Using mathematical induction we have the base case $n=1$ then $\bigcap_{j=1}^1U_j=U_1$ as $U_1\in \{U _\alpha\}_{\alpha \in A}$ the base case is complete. Now assume that there exists $n\in \mathbb{N}$ with $n>1$ where $\bigcap_{i=1}^{n}U_i\in \{U_\alpha\}_{\alpha\in A}$. Then $\bigcap_{i=1}^{n+1}U_i=(\bigcap_{i=1}^{n}U_i)\cap U_{n+1}$ based on the induction hypothesis we have $\bigcap_{i=1}^{n}U_i\in \{U_\alpha\}_{\alpha \in A}$ therefore for some $\beta \in A$ we have $U_{\beta}=\bigcap_{i=1}^{n}U_i$. This implies $(\bigcap_{i=1}^{n}U_i)\cap U_{n+1}=U_{\beta}\cap U_{n+1}$ which by the assumption of the collection of sets we get $U_\beta \cap U_{n+1}\in \{U_{\alpha}\}_{\alpha \in A}$ which completes the proof.
\end{proof}


{\bfseries b.} Give an example of a collection that satisfies the property stated at the beginning of this problem and in which there is an infinite collection of sets whose intersection is not in the collection. As always, prove that your example has the asserted properties.

\begin{proof}
    Consider the collection $\{U_\alpha\}_{\alpha \in \mathbb{N}\setminus \{0\}}$ of sets of the form $U_n=(0,\frac{1}{n})$ where $n=\mathbb{N}\setminus\{0\}$. We have this satisfying the property at the beginning as given any two positive integers $a,b$ we have $U_a\bigcap U_b=U_{\min(\{a,b\})}$ as $\min(\{a,b\})\in \mathbb{N}\setminus\{0\}$ we have $ U_a\cap U_b\in \{U_\alpha\}_{\alpha \in \mathbb{N}\setminus \{0\}}$. 

    Now take the intersection $\bigcap_{n=1}^\infty U_n$ we have that this intersection is just the empty set. This is proven by assuming that it is non empty so there is a real number $r\in \bigcap_{n=1}^{\infty}U_n$ this implies that there exists a real number $0<r<\frac{1}{n}$ for all positive integers $n$. Rearranging the inequality we get $0<n<\frac{1}{r}$  which would imply that the positive integers are bounded above a contradiction. Hence the intersection is empty. Now to show that $\emptyset \not \in \{U_n\}_{n\in \mathbb{N}\setminus \{0\}}$ assume that $\emptyset \in \{U_n\}_{n\in \mathbb{N}\setminus \{0\}}$ then we would have for some $n\in \mathbb{N}\setminus \{0\}$ the following is true $\frac{1}{n}=0$ which would imply $1=0$ a contradiction. 
    
\end{proof}

\vspace{.1in}





{\bfseries c.} In the setting described at the beginning of this problem, assume that $A_1$ and $A_2$ are subsets of $A$. Show that
\[
    \left( \bigcup _{\beta \in A_1} U_{\beta} \right) \bigcap \left( \bigcup _{\gamma \in A_2} U_{\gamma} \right) = \bigcup _{\beta \in A_1 \ \mbox{and} \ \gamma \in A_2} (U_{\beta} \bigcap U_{\gamma}).
\]

\begin{proof}
    This will be a proof by double containment. Assume $A_1,A_2$ are both subsets of $A$.
    Assume that $x\in \left( \bigcup_{\beta \in A_1}\right) \bigcap \left( \bigcup_{\gamma \in A_2}U_{\gamma}\right)$. Then for some $\beta \in A_1$ and some $\gamma \in A_2$ we have $x\in U_\beta \cap U_\gamma$ which implies $\bigcup _{\beta \in A_1 \text{and} \gamma \in A_2}(U_\beta \cap U_\gamma)$. This implies \[
        \left( \bigcup _{\beta \in A_1} U_{\beta} \right) \bigcap \left( \bigcup _{\gamma \in A_2} U_{\gamma} \right) \subseteq \bigcup _{\beta \in A_1 \ \mbox{and} \ \gamma \in A_2} (U_{\beta} \bigcap U_{\gamma})
        \]
    Now let $x\in \bigcup _{\beta \in A_1 \ \mbox{and} \ \gamma \in A_2} (U_{\beta} \bigcap U_{\gamma})$ then $x\in U_\beta \cap U_\gamma$ for some $\beta \in A_1, \; \gamma \in A_2$. This implies $x\in \left( \bigcup_{\beta \in A_1} U_\beta\right)\cap \left(U_{\gamma \in A_2}U_\gamma\right)$ which implies \[   \left( \bigcup _{\beta \in A_1} U_{\beta} \right) \bigcap \left( \bigcup _{\gamma \in A_2} U_{\gamma} \right) \supseteq \bigcup _{\beta \in A_1 \ \mbox{and} \ \gamma \in A_2} (U_{\beta} \bigcap U_{\gamma})\]

    By double containment we have that the sets are equal. 

\end{proof}


\vspace{.15in}




\noindent
\subsection*{2.} Let $X$ be an infinite set and let $\mathcal T _c$ be the set of subsets $U$ of $X$ satisfying: $U=X$ or $U=\emptyset$ or $X\setminus U$ is countable. Show that $\mathcal T _c$ is a topology on $X$. This is often called the countable-complement topology.

\begin{proof}
    Assume $X$ is an infinite set and $\mathcal T _ c$ is as stated. We have $\emptyset,X\in \mathcal T _c$ by the definition of $\mathcal T _c$. Consider the intersection of $U_1\cap U_2$ where $U_1,\in U_2 \in \mathcal T_c$. If either $U_1=\emptyset$ or $U_2=\emptyset$ we have $U_1\cap U_2=\emptyset$ and if $U_1=X$ then $U_1\cap U_2=U_2$. In either case the intersection is in $\mathcal T_c$. Now given $U_1,U_2\in \mathcal T_c$ assuming neither is the empty set or $X$. Then $X\setminus (U_1\cap U_2)=X\cap \overline{(U_1\cap U_2)}=X\cap(\overline{U_1}\cup \overline{U_2})=X\cap\overline U_1 \cup X\cap \overline {U_2}$. (note the 'overline' is set complement, I used De Morgan's laws and also distributivity of sets over intersection,  union). We have that the union of two countable sets is countable following the equality that implies $X\setminus (U_1\cap U_2)$ is countable which implies $U_1\cap U_2\in \mathcal T_c$. That covers the base case so now assume for some $k\in \mathbb{N}$ with $k\geq 2$ we have that the intersection of $k$ elements of $\mathcal T_c$ is in $\mathcal T_c$. Then given the intersection of $k+1$ elements of $\mathcal T_c$ we have $U_1\cap ... \cap U_{k+1}$ then we have the intersection of $k$ of the elements is in $\mathcal T_c$. WLOG $U_{k+1}\not =\emptyset$ or $U_{k+1}\not = X$.
    \begin{itemize}
        \item If $U_1\cap ...\cap U_k=\emptyset$ then $U_1\cap ...\cap U_{k+1}=\emptyset$
        \item If $U_1\cap ... \cap U_{k}=X$ then $U_1\cap ... U_{k+1}=U_{k+1}$
        \item If $X\setminus{U_1\cap ... \cap U_{k}}$ is countable then using the same reasoning as above we have $X\setminus{U_1\cap...\cap U_{k+1}}$ is countable as well.
    \end{itemize}
    In all cases we have that $\mathcal T_c$ is closed under finite intersections. 

    Now given a subset of $S\subset \mathcal T_c\setminus \{\emptyset\}$ we have $X\setminus \bigcup_{U\in S}U\subseteq X\setminus U$ where $U\in S$. As subsets of countable sets are countable then we have arbitrary unions are closed in $\mathcal {T}_c$ as well. In the case that the empty set is in a union the resulting set is equal to not having it be in the union hence the reason for the removal in $S$.


\end{proof}

\vspace{.15in}

Munkres introduces several topologies on the real line $\mathbb R$. The standard topology $\mathcal T$ is generated by $\{ U_{a, b} : a\in \mathbb R, \ b\in \mathbb R, \ \mbox{and} \ a < b\}$. Here $U_{a, b} = \{ x\in \mathbb R : a < x < b\}$. The lower limit topology $\mathcal T _l$ is generated by $\{ I_{a, b} : a\in \mathbb R, \ b\in \mathbb R, \ \mbox{and} \ a < b\}$, where $I_{a, b} = \{ x\in \mathbb R : a \le x < b\}$.

In doing problem 3. you may {\bfseries assume} without proof that the sets I have chosen to generate the topologies are bases for the topologies, so that Munkres Lemma 13.3 applies.



\vspace{.15in}

\noindent
\subsection*{3.}  Let $\mathcal T _Q$ be the topology on $\mathbb R$ generated by $\{ U_{a, b} : a\in \mathbb Q \ \mbox{and} \ b\in \mathbb Q \}$. Let $\mathcal T _{Q,l}$ be the topology on $\mathbb R$ generated by $\{ I_{a, b} : a\in \mathbb Q \ \mbox{and} \ b\in \mathbb Q \}$. Show that $\mathcal T _Q = \mathcal T$. Show that $\mathcal T _{Q,l} \ne \mathcal T _l$.

\begin{proof}{${\mathcal{T}_Q=\mathcal{T}}$}
    
    Let $x\in \mathbb{R}$ and $x\in (a,b)$ where $a,b\in \mathbb{Q}$ with $a<b$ then as the rationals are a subset of the real numbers the same open interval is a basis for $\mathcal {T}$. Which implies by Lemma 13.3 $\mathcal{T}_Q\subset \mathcal{T}$.
    Now let $x\in \mathbb{R}$ and $x\in (a,b)$ where $a,b \in \mathbb{R}$ with $a<b$. Then as the rationals are dense in the reals we have two rational numbers $c,d$ such that $a<c<x<d<b$. This implies $x\in (c,d)\subset (a,b)$ by Lemma 13.3 we have $\mathcal T\subset \mathcal T_c$.
    By double containment $\mathcal T_c= \mathcal T$
 

\end{proof}
\vspace{.15in}

The next proof using contrapositive of Lemma 13.3 which says let $\mathcal{B}$ and $\mathcal{B}^\prime$ be bases for the topologies $\mathcal{T}$ and $\mathcal{T}^\prime$ on $X$. Then the following are equivalent:
\begin{enumerate}
    \item $\mathcal{T}^\prime$ is not finer than $\mathcal{T}$
    \item There exists $x\in X$ and there exists a basis element $B\in \mathcal{B}$ with $x\in B$ such that for all $B^\prime\in \mathcal{B}^\prime$ with $x\in B^\prime$ we have $B^\prime \not \subset B$
\end{enumerate}


\begin{proof}{$\mathcal T_{Q,l}\not= \mathcal T_l$}
    
 Let $\mathcal{T}_{Q,l},\mathcal T_l$ be as given. I will show that $\mathcal{T}_{Q,l}$ is not finer than $\mathcal{T}_l$. Let $x=\sqrt{2}$ and let $B=[\sqrt{2},2)$ then for all $a,b\in \mathbb{Q}$ with $a,b$ and $\sqrt{2}\in [a,b)$ we have $[a,b)\not \subset [\sqrt{2},2)$. This is true as $\sqrt{2}\not \in \mathbb{Q}$ implies $a<\sqrt{2}$ which implies $a\not \in [\sqrt{2},2)$. Therefore we have by the contrapositive of Lemma 13.3 $\mathcal{T}_{Q,l}$ is not finer than $\mathcal{T}_l$ which implies that they are not equal. 



\end{proof}



\subsection*{Definitions.} A {\bfseries sequence} in a topological space $X$ is a function $f : \mathbb N \rightarrow X$. Using the notation $x_n$ for $f(n)$, we also can think of a sequence as a list, indexed by the natural numbers, of elements of $X$. Thinking this way, we often use the notation $(x_n)$ for the sequence. We say that $(x_n)$ {\bfseries converges} to $y\in X$ if, for every open set $U$ containing $y$, there is an $N\in \mathbb N$ such that for all $n > N$, $x_n \in U$. When this happens, we call $y$ a {\bfseries limit} of the sequence $(x_n)$. We say that $(x_n)$ {\bfseries converges} if one or more elements of $X$ is a limit or are limits of $(x_n)$. The elements $x_n$ in $X$ are called the {\bfseries values} of the sequence. Note that the list of values in the sequence is always infinitely long but that some sequences have only finitely many different values. For example, in $\mathbb R$, the sequence in which $x_n = 0$ when $n$ is even and $x_n = 1$ when $n$ is odd has only two different values, and a constant sequence has only one value.

\vspace{.15in}

\noindent
\subsection*{4.}

{\bfseries a.} Suppose that $(x_n)$ is a sequence in an infinite space $X$ that has the finite-complement topology. Suppose that for all distinct $k$ and $m$, $x_k\ne x_m$. Must $(x_n)$ converge? If so, what element or elements does it converge to? As always, prove your assertions.

It doesn't converge.
\begin{proof}
    Consider the set $\mathbb{N}$ with the finite complement topology  $\mathcal{T}_c$ and the sequence $(x_n)=n$. For any $b\in \mathbb{N}$ with $b\in U\in \mathcal{T}_c$ we have no $N\in \mathbb{N}$ where $x_n\in U$ for all $n>N$. This is shown by assuming that such an $N,n$ do exist with the desired properties. Then we can construct the set $U^\prime=U\setminus \{x_j\}$ where $j>N$ and $x_j\not = b$. This element $x_j$ will exist as the sequence has no repeating elements and $U^\prime\in \mathcal{T}_c$ as $X\setminus U^\prime= X\setminus U\cup \{x_j\}$ and the union of two countable sets is countable. This violates the universal quantifier on $n>N$ hence it cannot convergence. 
\end{proof}


\vspace{.1in}
{\bfseries b.} Suppose that $(x_n)$ is a sequence in an infinite space $X$ that has the finite-complement topology. Suppose that for every even $k$, $x_k$ equals the same value $z\in X$, and assume that for all distinct odd $m$ and $l$, $x_m \ne x_l$. Does $(x_n)$ converge? If so, state what element(s) it converges to and prove that that convergence does happen. If not, prove that it does not converge to any element in $X$.

This sequence does not converge.

\begin{proof}
    This counterexample is on $\mathbb{N}$ with the finite complement topology $\mathcal T_c$. 
    Let $f:\mathbb{N}\mapsto \mathbb{N}$ where \[f(n)=\begin{cases}
        0 & \text{if } x\equiv 0\bmod 2\\
        n & \text{if } x\equiv 1 \bmod 2
    \end{cases}\] 
    Now take the sequence $(x_n)=f(n)$ this satisfies the properties stated in the problem. Now for any $b\in \mathbb{N}$ with $b\in U\in \mathcal T_c$ we have no $N\in \mathbb{N}$ where $x_n\in U$ for all $n> N$.  Assume that such a $N,n$ exist with the desired properties. Then we can construct the set $U^\prime=U\setminus\{x_j\}$ where $j>N$ and $x_j\not = b$ again such a $j$ is guaranteed to exist as each odd number is distinct. Then we have $X\setminus U^\prime = X\setminus \cup \{x_j\}$ as the union of two countable sets is countable we have $U^\prime\in \mathcal T_c$ however this violates the universal quantifier on $n>N$ a contradiction.
\end{proof}


\vspace{.15in}

\noindent
\subsection*{5.}

{\bfseries a.} Suppose that $(x_n)$ is a sequence in an infinite space $X$ with the discrete topology. Show that $(x_n)$ converges if and only if there exists $c\in X$ and there exists $N\in \mathbb N$ such that for all $n > N$, $x_n = c$.

\begin{proof}
    Let $(x_n)$ be a sequence in an infinite space $X$ with the discrete topology. 
    For the forward direction assume that $(x_n)$ converges. By the definition of convergence we get some $y\in X$ for all open set $U$ with $y\in U$ there exists an $N\in \mathbb{N}$ with every $n>N$ we have $x_n\in U$. However as this is the discrete topology there is a singleton open set  $S$ where $y\in S$. As $S$ is an open set then we have the existence of $N\in \mathbb{N}$ for all $n>N$ we have $x_n\in S$ but as $S$ is a singleton containing $y$ this implies $x_n=y$ which completes the forward direction. 
    Now for the backwards direction assume $(x_n)$ is a sequence in $X$ and there exists $c\in X$ and there exists a $N\in \mathbb{N}$ such that for all $n>N$ we have $x_n=c$. Now for any open set $U$ with $c\in U$ we have $x_n\in U$ for all $n>N$ as $x_n=c$ this implies convergence of the sequence.


\end{proof}

\vspace{.1in}
{\bfseries b.} Suppose that $(x_n)$ is a sequence in an infinite space $X$ with the countable-complement topology. Show that $(x_n)$ converges if and only if there exists $c\in X$ and there exists $N\in \mathbb N$ such that for all $n > N$, $x_n = c$.

\begin{proof}
    Assume $(x_n)$ sequence in an infinite space $X$ with the countable-complement topology $\mathcal{T}_c$. For the forward direction assume that $(x_n)$ converges. As $(x_n)$ converges we have some $c\in X$ such that for all open sets $U$ with $c\in U$ there exists $N\in \mathbb{N}$ such that for all $n>N$ we have $x_n\in U$. Take the set of all elements of the sequence $U_s=\{x_n: n\in \mathbb{N}\text{ and }x_n\not = c \}$ and we have that $X\setminus U_s$ is an open set as 
    \begin{align*}
        X\setminus (X \setminus(U_s ))& = X \cap \overline{(X \cap \overline{U_s})}\\
        &=X \cap (\overline{X}\cup U_s)\\
        &= (X\cap \overline X) \cup X\cap U_s\\
        &= \emptyset \cup U_s\\
        &= U_s
    \end{align*}
    (I used De Morgan's laws, distributivity of sets over union and intersection, intersection of a set with its complement is emptyset, intersection of any set with universal set is itself)
    Then as $U_s$ is countable as its just elements of the sequence we have $X\setminus U_s\in \mathcal{T}_c$. As $(x_n)$ is convergent and $c\in X\setminus U_s$ but no other elements of the sequence are in $X\setminus U_s$. This implies for all $n>N$ we have $x_n=c$ which completes the proof for the forward direction. 

    For the backwards direction assume that there exists $c\in X$ and there exists $N\in \mathbb{N}$ such that for all $n>N$, $x_n=c$. Then for any open set $U$ with $c\in U$ we have $x_n\in U$ for $n>N$ as $x_n=c$. This completes the proof.

\end{proof}











\end{document}