\documentclass{amsart}
\usepackage{amsmath, amssymb, amscd}

\setlength{\textwidth}{6.5in}
\setlength{\textheight}{9in}
\setlength{\topmargin}{-.25in}
\setlength{\evensidemargin}{0in}
\setlength{\oddsidemargin}{0in}

\theoremstyle{plain}
\newtheorem{theorem}{Theorem}[section]
\newtheorem{proposition}[theorem]{Proposition}
\newtheorem{lemma}[theorem]{Lemma}
\newtheorem{corollary}[theorem]{Corollary}

\theoremstyle{definition}
\newtheorem{definition}[theorem]{Definition}
\newtheorem{assumption}[theorem]{Assumption}

\theoremstyle{remark}
\newtheorem{remark}[theorem]{Remark}
\newtheorem{example}[theorem]{Example}
\newtheorem{notation}[theorem]{Notation}

\begin{document}

\section*{Math 4324\ \ \  Topologies }

\subsection*{Hand in Friday, February 2.} Homework due on a class day should be submitted as an email attachment sent to me by the beginning of class.



\vspace{.15in}


\noindent
\subsection*{1.}  Suppose that $\{ U_{\alpha} \} _{\alpha \in A}$ is a collection of sets satisfying the property
\[
    \forall a, b \in A, \ \ \ U_a \bigcap U_b \in \{ U_{\alpha} \} _{\alpha \in A}.
\]

{\bfseries a.} Show that for any finite collection $\{ U_1, . . .  , U_n \}$ of sets from this collection,
${\displaystyle \bigcap _{j = 1} ^n U_j}$ is in the collection.

\begin{proof}
    Using mathematical induction we have the base case $n=1$ then $\bigcap_{j=1}^1U_j=U_1$ as $U_1\in \{U _\alpha\}_{\alpha \in A}$ the base case is complete. Now assume that there exists $n\in \mathbb{N}$ with $n>1$ where $\bigcap_{i=1}^{n}U_i\in \{U_\alpha\}_{\alpha\in A}$. Then $\bigcap_{i=1}^{n+1}U_i=(\bigcap_{i=1}^{n}U_i)\cap U_{n+1}$ based on the induction hypothesis we have $\bigcap_{i=1}^{n}U_i\in \{U_\alpha\}_{\alpha \in A}$ therefore for some $\beta \in A$ we have $U_{\beta}=\bigcap_{i=1}^{n}U_i$. This implies $(\bigcap_{i=1}^{n}U_i)\cap U_{n+1}=U_{\beta}\cap U_{n+1}$ which by the assumption of the collection of sets we get $U_\beta \cap U_{n+1}\in \{U_{\alpha}\}_{\alpha \in A}$ which completes the proof.
\end{proof}


{\bfseries b.} Give an example of a collection that satisfies the property stated at the beginning of this problem and in which there is an infinite collection of sets whose intersection is not in the collection. As always, prove that your example has the asserted properties.

\vspace{.1in}

{\bfseries c.} In the setting described at the beginning of this problem, assume that $A_1$ and $A_2$ are subsets of $A$. Show that
\[
    \left( \bigcup _{\beta \in A_1} U_{\beta} \right) \bigcap \left( \bigcup _{\gamma \in A_2} U_{\gamma} \right) = \bigcup _{\beta \in A_1 \ \mbox{and} \ \gamma \in A_2} (U_{\beta} \bigcap U_{\gamma}).
\]


\vspace{.15in}

\noindent
\subsection*{2.} Let $X$ be an infinite set and let $\mathcal T _c$ be the set of subsets $U$ of $X$ satisfying: $U=X$ or $U=\emptyset$ or $X\setminus U$ is countable. Show that $\mathcal T _c$ is a topology on $X$. This is often called the countable-complement topology.

\subsection*{Remarks.} I follow Munkres in using ``countable" to refer to sets that are either finite or countably infinite. Some books may use ``countable" to mean ``countably infinite." Those books usually use ``at most countable" when they mean either finite or countably infinite. The only facts you need to know about ``countable" in order to do problem 3 are: a subset of a countable set is countable (Munkres Corollary 7.3) and a union of finitely many countable sets is countable (consequence of Munkres Theorem 7.5.) You may use these facts even if you do not understand their proofs. As we go further in the course, there will be times when you need to understand more about countable and not countable. The information you need is in Munkres Section 7. If you are not familiar with the ideas in that section, please come to an office hour to discuss them.

\vspace{.15in}

Munkres introduces several topologies on the real line $\mathbb R$. The standard topology $\mathcal T$ is generated by $\{ U_{a, b} : a\in \mathbb R, \ b\in \mathbb R, \ \mbox{and} \ a < b\}$. Here $U_{a, b} = \{ x\in \mathbb R : a < x < b\}$. The lower limit topology $\mathcal T _l$ is generated by $\{ I_{a, b} : a\in \mathbb R, \ b\in \mathbb R, \ \mbox{and} \ a < b\}$, where $I_{a, b} = \{ x\in \mathbb R : a \le x < b\}$.

In doing problem 3. you may {\bfseries assume} without proof that the sets I have chosen to generate the topologies are bases for the topologies, so that Munkres Lemma 13.3 applies.

\vspace{.15in}

\noindent
\subsection*{3.}  Let $\mathcal T _Q$ be the topology on $\mathbb R$ generated by $\{ U_{a, b} : a\in \mathbb Q \ \mbox{and} \ b\in \mathbb Q \}$. Let $\mathcal T _{Q,l}$ be the topology on $\mathbb R$ generated by $\{ I_{a, b} : a\in \mathbb Q \ \mbox{and} \ b\in \mathbb Q \}$. Show that $\mathcal T _Q = \mathcal T$. Show that $\mathcal T _{Q,l} \ne \mathcal T _l$.

\vspace{.15in}

\subsection*{Definitions.} A {\bfseries sequence} in a topological space $X$ is a function $f : \mathbb N \rightarrow X$. Using the notation $x_n$ for $f(n)$, we also can think of a sequence as a list, indexed by the natural numbers, of elements of $X$. Thinking this way, we often use the notation $(x_n)$ for the sequence. We say that $(x_n)$ {\bfseries converges} to $y\in X$ if, for every open set $U$ containing $y$, there is an $N\in \mathbb N$ such that for all $n > N$, $x_n \in U$. When this happens, we call $y$ a {\bfseries limit} of the sequence $(x_n)$. We say that $(x_n)$ {\bfseries converges} if one or more elements of $X$ is a limit or are limits of $(x_n)$. The elements $x_n$ in $X$ are called the {\bfseries values} of the sequence. Note that the list of values in the sequence is always infinitely long but that some sequences have only finitely many different values. For example, in $\mathbb R$, the sequence in which $x_n = 0$ when $n$ is even and $x_n = 1$ when $n$ is odd has only two different values, and a constant sequence has only one value.

\vspace{.15in}

\noindent
\subsection*{4.}

{\bfseries a.} Suppose that $(x_n)$ is a sequence in an infinite space $X$ that has the finite-complement topology. Suppose that for all distinct $k$ and $m$, $x_k\ne x_m$. Must $(x_n)$ converge? If so, what element or elements does it converge to? As always, prove your assertions.

\vspace{.1in}
{\bfseries b.} Suppose that $(x_n)$ is a sequence in an infinite space $X$ that has the finite-complement topology. Suppose that for every even $k$, $x_k$ equals the same value $z\in X$, and assume that for all distinct odd $m$ and $l$, $x_m \ne x_l$. Does $(x_n)$ converge? If so, state what element(s) it converges to and prove that that convergence does happen. If not, prove that it does not converge to any element in $X$.




\vspace{.15in}

\noindent
\subsection*{5.}

{\bfseries a.} Suppose that $(x_n)$ is a sequence in an infinite space $X$ with the discrete topology. Show that $(x_n)$ converges if and only if there exists $c\in X$ and there exists $N\in \mathbb N$ such that for all $n > N$, $x_n = c$.


\vspace{.1in}
{\bfseries b.} Suppose that $(x_n)$ is a sequence in an infinite space $X$ with the countable-complement topology. Show that $(x_n)$ converges if and only if there exists $c\in X$ and there exists $N\in \mathbb N$ such that for all $n > N$, $x_n = c$.












\end{document}