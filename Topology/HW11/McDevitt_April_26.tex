\documentclass{amsart}
\usepackage{amsmath, amssymb, amscd}

\setlength{\textwidth}{6.5in}
\setlength{\textheight}{9in}
\setlength{\topmargin}{-.25in}
\setlength{\evensidemargin}{0in}
\setlength{\oddsidemargin}{0in}

\theoremstyle{plain}
\newtheorem{theorem}{Theorem}[section]
\newtheorem{proposition}[theorem]{Proposition}
\newtheorem{lemma}[theorem]{Lemma}
\newtheorem{corollary}[theorem]{Corollary}

\theoremstyle{definition}
\newtheorem{definition}[theorem]{Definition}
\newtheorem{assumption}[theorem]{Assumption}

\theoremstyle{remark}
\newtheorem{remark}[theorem]{Remark}
\newtheorem{example}[theorem]{Example}
\newtheorem{notation}[theorem]{Notation}

\begin{document}

\section*{Math 4324\  Applications of the winding number} 

\subsection*{Hand in Monday, April 26.} 

\vspace{.25in}
\subsection*{} The {\bf final exam} will be on Friday, May 3, 10:05 am -- 12:05 pm, in the classroom. No books, no notes, no calculators. Bring paper and something to write with. The exam will cover important points from throughout the semester. The concepts needed to do homework assignments will be the concepts needed to do  final exam problems. 

\vspace{.25in}
\noindent
\subsection*{1.} Let $D$ be  a disk with boundary circle $C$, and let $f: D\to \mathbb{R}^2$ be a continuous map. Suppose $P$ is a point in $\mathbb{R}^2$ with $P\not \in f(C)$, and the winding number of the restriction $f|_C$ of $f$ to $C$ around $P$ is not zero. Show that there is some point $Q$ in $D$ such that $f(Q) = P$.

\begin{proof}
    
\end{proof}

\vspace{.15in}
\noindent
\subsection*{2.} Fulton textbook, page 52, exercise 4.15. {\bf  Hint.} Is $f$ homotopic to the map $g(P) = P^*$, where $P^*$ refers to the point on the circle at the other end of the diameter segment that hits $P$? When the circle has center the origin, $P^* = -P$. 


\vspace{.15in}
\noindent
\subsection*{3.} Fulton textbook, page 53, exercise 4.17. {\bf  Hint.} Is $f$ homotopic to the map $g(P) = \widetilde{P}$, where $\widetilde{P}$ refers to the point that is a counterclockwise rotation of $\pi /2$ from $P$. 

\vspace{.15in}
\noindent
\subsection*{4.} Fulton textbook, page 55, exercise 4.24.


\vspace{.15in}
\noindent
\subsection*{5.} Fulton textbook, page 55, exercise 4.27.




 
\end{document}