\documentclass{amsart}
\usepackage{amsmath, amssymb, amscd}

\setlength{\textwidth}{6.5in}
\setlength{\textheight}{9in}
\setlength{\topmargin}{-.25in}
\setlength{\evensidemargin}{0in}
\setlength{\oddsidemargin}{0in}

\theoremstyle{plain}
\newtheorem{theorem}{Theorem}[section]
\newtheorem{proposition}[theorem]{Proposition}
\newtheorem{lemma}[theorem]{Lemma}
\newtheorem{corollary}[theorem]{Corollary}

\theoremstyle{definition}
\newtheorem{definition}[theorem]{Definition}
\newtheorem{assumption}[theorem]{Assumption}

\theoremstyle{remark}
\newtheorem{remark}[theorem]{Remark}
\newtheorem{example}[theorem]{Example}
\newtheorem{notation}[theorem]{Notation}

\begin{document}

\section*{Math 4324\  Applications of the winding number} 

\subsection*{Hand in Monday, April 26.} 

\vspace{.25in}
\subsection*{} The {\bf final exam} will be on Friday, May 3, 10:05 am -- 12:05 pm, in the classroom. No books, no notes, no calculators. Bring paper and something to write with. The exam will cover important points from throughout the semester. The concepts needed to do homework assignments will be the concepts needed to do  final exam problems. 

\vspace{.25in}
\noindent
\subsection*{1.} Let $D$ be  a disk with boundary circle $C$, and let $f: D\to \mathbb{R}^2$ be a continuous map. Suppose $P$ is a point in $\mathbb{R}^2$ with $P\not \in f(C)$, and the winding number of the restriction $f|_C$ of $f$ to $C$ around $P$ is not zero. Show that there is some point $Q$ in $D$ such that $f(Q) = P$.

\begin{proof}

    Suppose that $f:D\to \mathbb{R}^2$ where $D$ has boundary circle $C$. Suppose that $P\in \mathbb{R}^2$ with $P\not \in f(D)$. Choose any point $Q\in C$ and and create the function $g:C\times [0,1]\to D$ by the equation $g(\vec v, t)=\vec v(1-t)+t \vec c$ this is a continuous function as it is a weighted sum of continuous functions. Now we have that $h:C\times [0,1]\to \mathbb{R}^2$ defined by the equation $h(\vec v,t)=f(g(\vec v,t))$ is a homotopy between $f|_C$ and the constant curve at $\vec c$. This is shown to be a homotopy as $h(\vec v,0)=f(g(\vec v,0))= f(\vec v)$ and $h(\vec v,1)=f(g(\vec v,1))=f(\vec c)$. We have that $h$ is continuous as it is the composition of two continuous functions. As the constant curve at $\vec c$ has winding number $0$ we get that $W(f|_C,P)=0$. This proves the contrapositive of the statement.
    
\end{proof}

\vspace{.15in}
\noindent
\subsection*{2.} Fulton textbook, page 52, exercise 4.15. {\bf  Hint.} Is $f$ homotopic to the map $g(P) = P^*$, where $P^*$ refers to the point on the circle at the other end of the diameter segment that hits $P$? When the circle has center the origin, $P^* = -P$. 

If $f:C\to C$  is a continuous mapping with no fixed point, show that degree of $f$ must be $1$. In particular, if $f$ has no fixed point, show that $f$ must be surjective. 

\begin{proof}
    It suffices to assume that $C$ is centered at the origin as we could create a homeomorphism that shifts it. Now assume that $f:C\to C$ is a continuous map with no fixed points. Then consider the homotopy $h:[0,1]\times C\to C$ defined by the equation $h(t,\vec c)=(1-t)f(\vec c)+t(-\vec c)$ this is continuous as it is two weighted continuous functions. As no point is fixed we get the line that intersects $f(\vec c)$ and $g(\vec c)=-\vec c$ does not intersect $\vec 0$. Lastly as $h(0,\vec c)=f(\vec c)$ and $h(1,\vec c)=-\vec c$ we get that $h$ is a homotopy. 
    
    On problem 2b on last weeks homework we showed that $\gamma(x,y)=(-x,-y)$ on the unit circle to the unit circle has degree 1. We have that $g$ is homotopic to $\gamma$ (changing radius of circle) so we get that $g$ has degree 1. As $f$ is homotopic to $g$ we get that $f$ has degree 1.

\end{proof}


\vspace{.15in}
\noindent
\subsection*{3.} Fulton textbook, page 53, exercise 4.17. {\bf  Hint.} Is $f$ homotopic to the map $g(P) = \widetilde{P}$, where $\widetilde{P}$ refers to the point that is a counterclockwise rotation of $\pi /2$ from $P$. 

If $f:D^2\to \mathbb{R}^2$ is continuous and $f(P)\cdot P\not = 0$ for all $P$ in $S^1$, show that there is some $Q\in D^2$ with $f(Q)=0$.

\begin{proof}
    Assume that $f:D^2\to \mathbb{R}^2$ is continuous and $f(P)\cdot P\not =0$ for all $P\in S^1$. Then as for all $\vec c\in C$ we can not have the line that intersects $f(\vec c)$ and $g(\vec c)$ intersects $\vec 0$ which follows from the fact that $f(\vec v)\cdot \vec v\not =0 $ and $g(\vec v)\cdot \vec 0=0$ hence we get by the Dog-on-a-Leash theorem that $W(f|_{S^1},\vec 0)=W(g,\vec 0)$. As $W(g,\vec 0)$ has a non zero winding number then we get by problem 1 that there exists a point $Q\in D^2$ such that $f(Q)=0$.

\end{proof}


\vspace{.15in}
\noindent
\subsection*{4.} Show that if $f:C\to C^\prime$ is a map between circles such that $f(P^\star)=f(P)$ for all $P$, then the degree of $f$ is even.

\begin{proof}
We have $C,C^\prime$ are both homeomorphic to $S^1$. So let $f:S^1\to S^1$ such that $f(P^\star)=f(P)$ for all $P$. 

Let $\gamma_1(\theta)=f(\cos \theta ,\sin \theta )$ for $0\leq \theta \leq \pi $ and $\gamma_2(\theta)=f(\cos \theta +\pi , \sin \theta+\pi)$ for $\pi \leq \theta \leq 2\pi$. 

As $f(\cos \theta,\sin \theta )=f(\cos \theta+\pi,\sin \theta+\pi)$ we get both $\gamma_1,\gamma_2$ are closed path hence $W(\gamma_1,\vec 0),W(\gamma_2,\vec 0)$ both have integer winding numbers. As $\gamma_1=\gamma_2$ we get $W(\gamma_1,\vec 0)=W(\gamma_2,\vec 0)=n$ for some $n\in \mathbb Z$. Hence we have $W(f,\vec 0)=W(\gamma_1,\vec 0)+W(\gamma_2,\vec 0)=2n$. Hence the degree of $f$ is even.

\end{proof}

\vspace{.15in}
\noindent
\subsection*{5.} Fulton textbook, page 55, exercise 4.27.

If $f$ and $g$ are continuous real-valued functions on a sphere $S$ such that that $f(P^\star)=-f(P)$ and $g(P^\star)=-g(P)$ for all $P$, show that $f$ and $g$ must have a common zero on the sphere. 

\begin{proof}
    Create the function $h:S\to \mathbb{R}^2$ where $h(\vec x)=(f(\vec x),g(\vec x))$ this is continuous by Munkres Theorem 18.4. Then we have by Borsuk-Ulam theorem that there exists a point such that $h(\vec x)=h(\vec x^\star)$ this implies $(f(\vec x),g(\vec x))=(-f(\vec x),-g(\vec x))$ hence we get $f(\vec x)= 0$ and $g(\vec x)=0$.
\end{proof}

 
\end{document}