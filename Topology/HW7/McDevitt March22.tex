\documentclass{amsart}
\usepackage{amsmath, amssymb, amscd}

\setlength{\textwidth}{6.5in}
\setlength{\textheight}{9in}
\setlength{\topmargin}{-.25in}
\setlength{\evensidemargin}{0in}
\setlength{\oddsidemargin}{0in}

\theoremstyle{plain}
\newtheorem{theorem}{Theorem}[section]
\newtheorem{proposition}[theorem]{Proposition}
\newtheorem{lemma}[theorem]{Lemma}
\newtheorem{corollary}[theorem]{Corollary}

\theoremstyle{definition}
\newtheorem{definition}[theorem]{Definition}
\newtheorem{assumption}[theorem]{Assumption}

\theoremstyle{remark}
\newtheorem{remark}[theorem]{Remark}
\newtheorem{example}[theorem]{Example}
\newtheorem{notation}[theorem]{Notation}

\begin{document}

\section*{Math 4324\  Versions of Compactness } 

\vspace{.15in}
The {\bf next test} will be run like the previous test, with the due date very likely to be Friday, April 12. 

\subsection*{Hand in Friday, March 22.} 


\vspace{.15in}
\noindent
\subsection*{1.} The concept of first countable is defined on pages 130-131 of the textbook. Suppose that $X$ is a first countable topological space and that $A$ is a subspace of $X$. Show that, if $x_0\in \overline{A}$, then there is a sequence of points $(a_n)$ from $A$ that converges to $x_0$. {\bf Remark.} Because, for any point $x$ in any metric space, $\{ B(x, 1/n)$ is a countable basis at $x$, every metric space is first countable. The book asserts that the proof of the assertion in the problem is very much like the proof of that assertion for metric spaces. I am asking you to write the proof for first countable spaces, not to quote the book's assertion that the ``same" proof applies. {\it Aside (not to be handed in).} How would you go about looking for a first countable space whose topology is not a metric topology? 

\vspace{.15in}
\subsection*{Some comments on textbook sections 20 and 21} It can be useful to know when a topology comes from a metric. Metric spaces have properties that align well with our intuition: open sets can be understood in terms of open balls, allowing precise quantitative expression of terminology like ``near enough"; metric spaces are Hausdorff; and because metric spaces are first countable, in them we can understand the concept of closure, and hence the concept of closed set, in terms of sequences. 

However, not all metric properties are topological properties. The concept of ``bounded" is one example of such a property. Theorem 20.1 shows that, on a metric space (for example $\mathbb R$ with the standard metric), it is possible to define another metric (the ``standard bounded metric") in which every subset is bounded (in fact the distance between any two points is less than or equal to $1$). However, on the small scale, the distances measured by the two metrics are the same. In particular the open balls of radius less than or equal to $1$ in one metric are exactly the same as the open balls of radius less than or equal to $1$ in the other metric. By problem 2 in our March 1 homework, the topologies defined by the two metrics are the same. 

One implication of the ``standard bounded metric" is that it can be used to prove (Theorem 20.5)  that the product topology on the product of countably many copies of $\mathbb R$ (each with the standard topology) arises from a metric on this product space. We showed in class that the box topology on this same product space cannot arise from a metric. Example 2 on page 133 of the textbook shows that, for much the same reason, the product topology on the product of uncountably many copies of $\mathbb R$ (each with the standard topology) does not arise from any metric on this product space.




\vspace{.15in}
\noindent
\subsection*{2.}  Let $X$ be an infinite set with the finite-complement topology. (You can assume that the $X$ is $\mathbb N$ or $\mathbb R$ if you like.) Show that every subset of $X$ is compact. Show that every infinite proper subset of $X$ is not closed. {\bf Remark.} So it can happen that a compact set is not closed! {\it Aside (not to be handed in).} What property of a topological space guarantees that every compact subset of that space is closed?

\vspace{.15in}
\noindent
\subsection*{Definition.} Let $X$ be a topological space, with $A$ a subset of $X$ and $y$ an element of $X$. We call $y$ an $\omega$-accumulation point for $A$ if every open set that contains $y$ contains infinitely many elements of $A$. 

\vspace{.15in}
\noindent
\subsection*{3.} Let  $X = \{ a_j : j \in \mathbb N \} \bigcup \{ b_j : j \in \mathbb N \}$, and give $X$  the topology defined by the basis $\{ \{ a_j , b_j \} : j\in \mathbb N \}$. (Each set in the basis contains exactly two elements, both the with same index.) (You may accept without proof that this set is a basis for a topology.) Show that every non-empty subset of $X$ has a limit point. Show that no subset of $X$ has an $\omega$-accumulation point.


\vspace{.15in}
\noindent
\subsection*{4.} Which single-element subsets $\{ x\}$ of the $X$ in problem 3 are closed sets? Show that your answer is correct. 

\vspace{.15in}
\noindent
\subsection*{Definition.} A topological space $X$ is called $T_1$ if every single-element subset of $X$ is closed. 

\vspace{.15in}
\noindent
\subsection*{Definition.}  A subset $K$ of a topological space $X$ is called limit point compact if every infinite subset of $K$ has a limit point in $K$. 


\vspace{.15in}
\noindent
\subsection*{Definition.}  A subset $L$ of a topological space $X$ is called $\omega$-accumulation point compact if every infinite subset of $L$ has an $\omega$-accumulation point in $L$. (This concept is widely accepted, but there is no widely accepted name for it, so the name I have chosen is not standard.)


\vspace{.15in}
\noindent
\subsection*{5.} 

{\bfseries a.} Show that every $\omega$-accumulation point compact topological space is limit point compact. 

{\bfseries b.} Assume that the topological space $Y$ is $T_1$. Show that if $Y$ is limit point compact, then $Y$ is $\omega$-accumulation point compact. 

{\bfseries c.} Give an example of a topological space that is limit point compact but not $\omega$-accumulation point compact. 

\vspace{.15in}
Problem 6 is an {\bf extra credit} problem. (I designate a problem as extra credit because I think the problem is interesting but I'm aware that it may make an assignment longer than the assignment should be.)

\vspace{.15in}
\noindent
\subsection*{6.} Let $X$ be the product of copies of $\mathbb R$, with the factors indexed by the elements $x$ of the unit interval $[0,1]$. Give $X$ the product topology. Because an element of the product corresponds to the assignment of a real number to each $x\in[0,1]$, we may think of the product as the set of functions $f : [0,1] \rightarrow \mathbb R$. In the product topology, by problem 4 in the February 9 homework assignment, expressed in the language in which elements of a product are functions on the index set, convergence of a sequence, $f_n \rightarrow f$ as $n\rightarrow \infty$, corresponds to pointwise convergence of the sequence of functions: for each $x\in [0,1]$, $f_n(x) \rightarrow f(x)$ as $n\rightarrow \infty$. Let $C $ be the subset of the product that corresponds to the set of continuous functions from $[0,1]$ to $\mathbb R$. Give $C$ the subspace topology. In this problem you will show that this topology on $C$ cannot arise from a metric on $C$ because $C$ is not first countable. Recall that saying that a topological space is first countable means that for every $y$ in the space, there is a countable collection of open sets $\{ V_j^y : j\in \mathbb N \}$ that contain $y$ and that satisfy the property: for 
every open $U$ that contains $y$ there is a $j\in \mathbb N$ for which $V_j \subset U$. In this problem you will show that the constant zero function $G$ does not have such a countable basis for its neighborhoods in $C$.

Suppose that, in $C$, $\{ U_j\}$ is an arbitrary countable collection of open sets, each of which contains $G$. For each $j$, choose a basis element $W_j$ (that is the intersection of a product basis element with $C$) satisfying $G\in W_j\subset U_j$. Let $I_0$ denote the set of indices $x\in [0,1]$ at which function values are constrained by at least one of the $W_j$'s. Show that $I_0$ is countable. Conclude that there exists an $x_1 \in [0,1] \setminus I_0$. 

For such an $x_1$, Let $W$ be the open neighborhood of $G$ defined by $|g(x_1)| < 1/2$. 

Show that, for every $j$, there is an $f_j \in W_j\subset U_j$ such that $f_j(x_1) = 1$. Conclude that, for every $j$, $W_j$ is not contained in $W$ and hence that $U_j$ is not contained in $W$.  







 
\end{document}