\documentclass{amsart}
\usepackage{amsmath, amssymb, amscd}

\setlength{\textwidth}{6.5in}
\setlength{\textheight}{9in}
\setlength{\topmargin}{-.25in}
\setlength{\evensidemargin}{0in}
\setlength{\oddsidemargin}{0in}

\theoremstyle{plain}
\newtheorem{theorem}{Theorem}[section]
\newtheorem{proposition}[theorem]{Proposition}
\newtheorem{lemma}[theorem]{Lemma}
\newtheorem{corollary}[theorem]{Corollary}

\theoremstyle{definition}
\newtheorem{definition}[theorem]{Definition}
\newtheorem{assumption}[theorem]{Assumption}

\theoremstyle{remark}
\newtheorem{remark}[theorem]{Remark}
\newtheorem{example}[theorem]{Example}
\newtheorem{notation}[theorem]{Notation}

\begin{document}

\section*{Math 4324\  Continuous maps } 

\subsection*{Hand in Friday, March 15.}



\vspace{.15in}

\noindent
\subsection*{1.} Suppose that $X$ and $Y$ are topological spaces, that $A \subset X$, and that $f$ and $g$ are continuous maps from $X$ to $Y$ that satisfy, for all $x\in A$, $f(x) = g(x)$. If $Y$ is Hausdorff, show that, for all $z$ in the closure of $A$, $f(z) = g(z)$.

\begin{proof}
    Assume $X$ and $Y$ are topological spaces, that $A\subset X$, and that $f$ and $g$ are continuous maps from $X$ to $Y$ that satisfy, for all $x\in A$, $f(x)=g(x)$. Assuming $Y$ is Hausdorff and that there there exists some point $z$ in the closure of $A$ where $f(z)\not = g(z)$. Then as $Y$ is Hausdorff we have two neighborhoods $U_f$ of $f(z)$ and $U_g$ of $g(z)$ where $U_f\cap U_g=\emptyset$. As $f,g$ are continuous we have that $f^{-1}(U_f)$ and $g^{-1}(U_g)$ are both open. We have that $z\in f^{-1}(U_f)\cap g^{-1}(U_g)$ but we have that $z$ is a limit point as if $z\in A$ then that would be an immediate contraction on $f(z)\not = g(z)$ hence we have $A\cap f^{-1}(U_f)\cap g^{-1}(U_g)\setminus \{z\}\not = \emptyset $ hence we have for some $a\in A\cap f^{-1}(U_f)\cap g^{-1}(U_g)\setminus \{z\}$ then we have a neighborhood $U_{a}$ of $a$ with $U_{a}\subset f^{-1}(U_f)\cap g^{-1}(U_g)$ hence we have $f(a)\in U_f\cap U_g$ which contradicts $U_f$ and $U_g$ being disjoint. 
\end{proof}


\vspace{.15in}

\noindent
\subsection*{2.} Let $f : X_1 \rightarrow Y_1$ and $g : X_2 \rightarrow Y_2$ be continuous maps between topological spaces. Give the products $X_1 \times X_2$ and $Y_1 \times Y_2$ their product topologies. Show that the map $H : X_1 \times X_2 \rightarrow Y_1 \times Y_2$ defined by $H((x_1 , x_2 )) = (f(x_1) , g(x_2) )$ is continuous. 

\begin{proof}
    Assume that $f:X_1 \to Y_1$ and $g: X_2\to Y_2$ are continuous maps between topological spaces. Assume we have $X_1\times X_2$ and $Y_1\times Y_2$ with their product topologies with the map $H: X_1\times X_2 \to Y_1\times Y_2$ where $H((x_1,x_2))=(f(x_1),g(x_2))$. Then consider an arbitrary basis element $U_1\times U_2 \subset Y_1\times Y_2$ then we have by the definition of product topology that $U_1$ is open in $Y_1$ and $U_2$ is open in $Y_2$ and as $f,g$ are continuous we have $f^{-1}(U_1)$ and $g^{-1}(U_2)$ are open hence $f^{-1}(U_1) \times g^{-1}(U_2)$ is open in $X_1\times X_2$ this implies $H^{-1}(U_1\times U_2)=f^{-1}(U_1)\times g^{-1}(U_2)$ is open which shows that $H$ is continuous.
    
\end{proof}




\vspace{.15in}

\noindent
\subsection*{3.} An injective (one-to-one) continuous map $f : X \rightarrow Y$ between topological spaces is a bijection from $X$ to the image $f(X)$. Give $f(X)$ the subspace topology induced by $Y$'s topology.  We call such an $f$ an imbedding if it is a homeomorphism from $X$ to $f(X)$. 

In this context, let $Y$ be $X \times X$ with the product topology. Let $x_0$ be an arbitrary element of $X$. 

 \vspace{.1in}
{\bfseries a.} Show that $f : X \rightarrow X \times X$ defined by $f(x) = (x, x_0 )$ is an imbedding.

\begin{proof}
    Assume that $f:X \to X\times X$ is a map defined by $f(x)=(x,x_0)$ where $x_0$ is an arbitrary element of $X$. Suppose $x,y\in X$ with $f(x)=f(y)$ then we have $(x,x_0)=(y,x_0)$ hence $x=y$ which shows that $f$ is injective. Let $U_1\times U_2 \subset f(X)$ be an arbitrary basis element. Then as $f(X)=X\times \{x_0\}$ we get that $U_2=\{x_0\}$ additionally from the definition of subspace topology we get $U_1=X\cap U_3$ where $U_3$ is some open set in $X$ hence $U_1$ is also open. As $f^{-1}(U_1\times U_2)=U_1$ we have $f^{-1}(U_1\times U_2)$ is open. Hence $f$ is continuous. 

    We have $f^{-1}:f(X)\to X$ where $f^{-1}(x,y)=x$ is a bijection as we have shown that $f$ is injective and it is surjective with its own range. So we just need to show that $f^{-1}$ is continuous. Given an arbitrary basis element $U$ of $X$. We have that $(f^{-1})^{-1}=f$ hence $f(U)=U\times\{x_0\}$ and as $U\times \{x_0\}=f(X)\cap (U\times X)$ and $U\times X$ is open in the product topology we get $U\times \{x_0\}$ is open in the subspace topology hence which shows that $f^{-1}$ is continuous which implies $f$ is a homeomorphism which shows that $f$ is an imbedding.



    

\end{proof}

\vspace{.1in}
{\bfseries b.} Show that $g : X \rightarrow X \times X$ defined by $g(x) = (x, x )$ is an imbedding. 

\begin{proof}
    Let $g:X \to X\times X$ be defined by $g(x)=(x,x)$. Let $x,y\in X$ be two elements with $g(x)=g(y)$ then $(x,x)=(y,y)$ which implies $x=y$ hence $g$ is injective. Now let $U_1 \times U_2 \subset g(X)$ be an open set. Then we have $U_1\times U_2=\{(x,x):x\in X\}\cap A\times B$ for some open sets $A,B$ in $X$ which implies $U_1\times U_2=\{(x,x): x\in A \cap B\}$. Then $g^{-1}(U_1\times U_2)=A\cap B$ and as $A,B$ are open we get $A\cap B$ is open which shows that $g$ is continuous.

    Let the map $g^{-1}:g(X)\to X$ be defined by $g^{-1}(x,x)=x$ we have already shown $g$ is injective and as $g$ is surjective with its own range we have that $g^{-1}$ is a bijection. Let $U\subset X$ be an open set then we have $(g^{-1})^{-1}=g$ hence $(g^{-1})^{-1}(U)=g(U)=\{(x,x):x\in U\}$. As $\{(x,x):x\in U\}=(U\times U) \cap g(X)$ we get $\{(x,x):x\in U\}$ is open in the subspace topology hence $g^{-1}$ is continuous which implies that $g$ is an imbedding.

\end{proof}


\vspace{.15in}

\noindent
\subsection*{4.} Suppose that $h : X \rightarrow Y$ is a homeomorphism of topological spaces. If $Z$ is any other topological space and if $g : Y \rightarrow Z$ is a continuous map, we know that the composition $g\circ h$ is a continuous map from $X$ to $Z$. Show that every continuous map $f : X \rightarrow Z$ arises this way, i.e. for any continuous $f : X \rightarrow Z$, there exists a continuous $G : Y \rightarrow Z$ for which $f = G \circ h$. 
\vspace{.15in}

\begin{proof}
  Assume $h:X\mapsto Y$ is a homeomorphism of topological spaces and $Z$ is some other topological space. Then given an arbitrary continuous function $f:X\to Z$ we need to create a function $G: Y\to Z$ such that $G\circ h=f$. First I will prove the existence and then that it is continuous. Define $G: Y \to Z$ where $G(y)=f(h^{-1}(y))$.
  
  Now to prove that $G$ is continuous let $U\subset Z$ be an open set. Then as $f^{-1}(U)$ is open in $X$ we have that $h(f^{-1}(U))$ is open in $Y$ as homeomorphisms preserve open sets (this is immediate based off both directions of homeomorphisms being continuous) we just need to show $G^{-1}(U)=h(f^{-1}(U))$. 
  
  Let $y\in G^{-1}(U)$ as $h$ is a homeomorphism we have for some unique $x\in X$ that $h(x)= y $  and as $G(y)=f(x)\in U$ we get $x\in f^{-1}(U)$ which implies $y=h(x)\in h(f^{-1}(U))$ hence $y\in h(f^{-1}(U))$ which gives $G^{-1}(U)\subset h(f^{-1}(U))$ 

  Now let $y\in h(f^{-1}U)$. Then as $h$ is a homeomorphism we have for some $x\in f^{-1}(U)$ that $y=h(x)$ but then $G(y)=f(h^{-1}(y))=f(x)\in U$ which implies $y\in G^{-1}(U)$ hence $h(f^{-1}U)\subset G^{-1}(U)$ then $h(f^{-1}U)=G^{-1}(U)$ which implies $G$ is continuous. As for any $x\in X$ we have $G\circ h (x)=f(h^{-1}(h(x)))=f(x)$ we get $f=G\circ h$. 
\end{proof}

\noindent
\subsection*{5.}

{\bf a.} Show that a linearly ordered set with the order topology is Hausdorff. 


\begin{proof}
    Suppose $X$ is a linearly ordered set with the order topology. Let $a,b\in X$ with $a\not = b$ and without loss of generality assume $a<b$. I will proceed by cases. 
    \begin{enumerate}
        \item If there is no element $c\in X$ with $a<c<b$ and $a$ is the minimum element of $X$ and $b$ is the maximum element of $X$ consider the open sets $a\in [a,b)$ and $b\in (a,b]$ we have each of the sets is open and $ [a,b)\cap  (a,b]=\emptyset$. 
        \item If there exists an element $c\in X$ with $a<c<b$ and $a$ is the minimum element of $X$ and $b$ is the maximum. Then consider the open sets $U_a\cap [a,c)$ and $U_b\cap (c,b]$ then we have $a\in  [a,c)$ and $b\in  (c,b]$ but $[a,c)\cap (c,b]=\emptyset$.
        \item If there is no element $c\in X$ with $a<c<b$ and $a$ is not a minimum element of $X$ and $b$ is not a maximum of $X$ then we have the sets $(d,b)$ where $d\in X$ with $d<a$ is open and the set $(a,l)$ where $l\in X$ with $b<l$ is open. We have $a\in  (d,b)$ and $b\in  (a,l)$ but $(d,b)\cap  (a,l)=\emptyset $
        \item If there exists some element $c\in X$ with $a<c<b$ and $a$ is not a minimum of $X$ and $b$ is not a maximum of $X$ then we have the sets $(d,c)$ where $d\in X$ and $d<a$ is open and the set $(c,l)$ where $l\in X$ with $b<l$ is open. We have $a\in(d,c)$ and $b\in (c,l)$ where $l\in X$ with $b<l$ but $ (d,c) \cap (c,l)=\emptyset$
        \item If there exists no element $c\in X$ with $a<c<b$ and without loss of generality $a$ is a minimum of $X$ and $b$ is not the maximum (the case where $a$ is not min of $X$ and $b$ is max of $X$ will follow by almost the same exact reasoning) then consider the open sets $a\in  [a,b)$ and $b\in  (a,d)$ where $d\in X$ with $b<d$ then we have $ [a,b) \cap (a,d)=\emptyset$.
        \item If there exists some element $c\in X$ with $a<c<b$ and without loss of generality $a$ is a minimum of $X$ and $b$ is not the maximum. Then $a\in  [a,c)$ and $b\in  (c,d)$ where $d\in X$ with $b<d$ then $[a,c) \cap (c,d)=\emptyset$ 
    \end{enumerate} 
    This completes all the cases hence we get $X$ is a Hausdorff space.
\end{proof}


{\bf b.} Suppose that $X$ is a topological space. Show that $X$ is Hausdorff if and only if the diagonal subset $\{ (x , x ) : x\in X\}$ of the product $X\times X$ is a closed subset of the product. Assume here that the topology on $X\times X$ is the product topology.

\begin{proof}

    $\newline$
    $(\Rightarrow)$

    Suppose $X$ is a topological space and $X$ is Hausdorff. Now assume that $X\times X$ has the product topology. Assume that $(a,b)\in X\times X$ with $a\not = b$ is a limit point of $\{(x,x):x\in X\}$. Then as $a\not = b$ and $X$ is Hausdorff we have two neighborhoods $U_a,U_b$ of $a,b$ respectively with $U_a\cap U_b=\emptyset$ then we have $(a,b)\in U_a\times U_b$ but as $U_a\cap U_b=\emptyset$ we get $(U_a\times U_b) \cap \{(x,x):x\in X\}=\emptyset$ hence $(a,b)$ is not a limit point of $\{(x,x): x\in X\}$. 
    
    This implies either there are no limit points of $\{(x,x):x\in X\}$ or $\{(x,x):x\in X\}^\prime \subset \{(x,x):x\in X\}$ in either case we get $\overline {\{(x,x):x\in X\}}= \{(x,x):x\in X\}\cup \{(x,x):x\in X\}^\prime = \{(x,x):x\in X\}$ hence $\{(x,x):x\in X\}$ contains it's limit points so its closed. 

    $(\Leftarrow)$

    Assume that $X$ is a topological space and that $X\times X$ has the product topology and $\{(x,x): x\in X\}$ is closed in $X\times X$. Then for any $a,b\in X$ with $a\not = b$ we have $(a,b)\in \{(x,x): x\in X\}^c$ as there exists a basis element of the form $(a,b)\in U_a\times U_b\subset \{(x,x): x\in X\}^c$ but as we have $U_a\times U_b \cap \{(x,x): x\in X\}=\emptyset$ we get that there exists no $x\in X$ such that $(x,x)\in U_a\times U_b$ which implies $U_a\cap U_b=\emptyset$ this implies that $X$ is Hausdorff.

\end{proof}


\vspace{.15in}

\noindent
\subsection*{6.} Let $Y$ be an ordered set in the order topology. Let $f,g:X\to Y$ be continuous.
\begin{enumerate}
    \item Show that the set $\{x: f(x)\leq g(x)\}$ is closed in $X$.
    \begin{proof}
        Assume that $X$ is a topological space and $Y$ is an ordered set in the order topology and $f,g:X\to Y$ are continuous. We just need to show that $\{x: f(x)\leq g(x)\}^c=\{x: f(x)> g(x)\}$ is open. Let $x\in \{x: f(x)> g(x)\}$ then $f(x)\not = g(x)$. I will proceed by cases. \begin{enumerate}
            \item If there exists no $y\in Y$ such that $f(x)>y>g(x)$ and $f(x)=\max(Y)$ and $g(x)=\min(Y)$ we have the neighborhoods $f(x)\in (g(x),f(x)]$ and $g(x)\in [g(x),g(x))$ as $f,g$ are continuous we have $f^{-1}((g(x),f(x)])$ and $g^{-1}([g(x),f(x)))$ are open in $X$ with for all $x_f\in  f^{-1}((g(x),f(x)])$ for all $x_g\in g^{-1}([g(x),f(x)))$ we have $f(x_f)>g(x_g)$ hence we get \[f^{-1}((g(x),f(x)])\cup g^{-1}([g(x),f(x))) \subset \{x: f(x)>g(x)\}\]
            \item If there exists some $y\in Y$ such that $f(x)>y>g(x)$ and $f(x)=\max(Y)$ and $g(x)=\min (Y)$ then we have the neighborhoods $f(x)\in (y,f(x)]$ and $g(x)\in [g(x),f(x))$ and $f^{-1}((y,f(x)])$ is open and $g^{-1}([g(x),f(x)))$ is open. Using the same reasoning as above we have 
            \[
                f^{-1}((y,f(x)])\cup g^{-1}([g(x),y)) \subset \{x: f(x)>g(x)\}
            \]
            \item If there exists no $y\in Y$ with $f(x)>y > g(x)$ and $f(x)\not = \max (Y)$ and $g(x)\not = \min (Y)$ then we have the neighborhoods $f(x)\in (g(x),a)$ where $a\in Y$ with $a> f(x)$ and $g(x)\in (b,f(x))$ where $b\in Y$ with $b< g(x)$ then using same reasoning \[ f^{-1}((g(x),a))\cup g^{-1}((b,f(x))) \subset \{x: f(x)>g(x)\}\]
            \item If there exists some $y\in Y$ where $f(x)>y>g(x)$ and $f(x)\not =\max(Y)$ and $g(x)\not =\min(Y)$ then we have the the neighborhoods $f(x)\in (y,b)$ where $b\in Y$ with $b>f(x)$ and $g(x)\in (c,y)$ where $x\in Y$ with $c<g(x)$. Then 
            \[
                f^{-1}((y,b))\cup g^{-1}((c,y))\subset \{x:f(x)> g(x)\}
            \]
            \item If there exists no $y\in Y$ where $f(x)>y>g(x)$ and without loss of generality $f(x)=\max(Y)$ and $g(x)\not = \min (Y)$ then we have the neighborhoods $f(x)\in (g(x),f(x)]$ and $g(x)\in (c,f(x))$ where $c\in Y$ with $c<g(x)$. Then 
            \[
                f^{-1}((g(x),f(x)])\cup g^{-1}((c,f(x)))\subset \{x:f(x)> g(x)\}
            \]
            \item If there is some $y\in Y$ where $f(x)>y> g(x)$ and without loss of generality $f(x)=\max (Y)$ and $g(x)\not =\min (Y)$ then we have the neighborhoods $f(x)\in (y, f(x)]$ and $g(x)\in (b,y)$ where $b\in Y$ with $b<g(x)$ then 
            \[
                f^{-1}((y,f(x)])\cup g^{-1}((b,y))\subset \{x:f(x)>g(x)\}
            \]
        
        \end{enumerate}
        Hence for all $x\in \{x: f(x)>g(x)\}$ we have the existence of two open sets $U,V$ in $Y$ with $x\in f^{-1}(U)\cup g^{-1}(V)\subset \{x: f(x)>g(x)\}$ then we get 
        \[
            \bigcup_{x\in \{x: f(x)>g(x)\}}f^{-1}(U_x)\cup g^{-1}(V_x)= \{x: f(x)>g(x)\}
        \]
        This implies that $\{x:f(x)>g(x)\}$ is open which implies that $\{x:f(x)\leq g(x)\}$ is closed. 
    \end{proof}


    \item Let $h:X\to Y$ be the function \[ h(x)=\min \{f(x),g(x)\}\]. Show that $h$ is continuous. 
    \begin{proof}
    Assume that $X$ is a topology and $Y$ is an ordered set with the order topology. Let $g,f:X\to Y$ be continuous maps. Let $h:X\to Y$ be the function where $h(x)=\min \{f(x),g(x)\}$. 
    We have $X=\{x:f(x)\leq g(x)\}\cup \{x: g(x)\leq f(x)\}$ both $\{x: g(x)\leq f(x)\},\{x:f(x)\leq g(x)\}$ are closed by part (a). As $f,g$ are continuous then $f^\prime:\{x:f(x)\leq g(x)\}\to Y $ where $f^\prime(x)=f(x)$ is continuous and $g^\prime :\{x: g(x)\leq f(x)\}\to Y$ where $g^{\prime}(x)=g(x)$ is continuous.
    
    We also have $\{x:f(x)\leq g(x)\}\cap \{x: g(x)\leq f(x)\}=\{x: f(x)=g(x)\}$ hence $f^\prime(x)=g^\prime(x)$ for all $x\in \{x:f(x)\leq g(x)\}\cap \{x: g(x)\leq f(x)\}=\{x: f(x)=g(x)\}$ then by the Munkres Theorem 18.3 we have the continuous function $h^\prime: X\to Y$ where $h^\prime(x)=g^\prime(x)$ if $x\in \{x: g(x)\leq f(x)\}$ and $h^\prime(x)=f^\prime(x)$ if $x\in \{x: f(x)\leq g(x)\}$. 

    To show $h=h^\prime$ let $x\in X$ then $h(x)=\min\{f(x),g(x)\}$ and $h^\prime(x)=g(x)$ if $g(x)\leq f(x)$ or $h^\prime (x)=f(x)$ if $f(x)\leq g(x)$ which shows $h^\prime(x) = \min \{g(x),f(x)\}$.
    \end{proof}
\end{enumerate}

 
\end{document}