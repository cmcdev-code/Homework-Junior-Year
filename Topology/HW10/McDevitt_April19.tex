\documentclass{amsart}
\usepackage{amsmath, amssymb, amscd}

\setlength{\textwidth}{6.5in}
\setlength{\textheight}{9in}
\setlength{\topmargin}{-.25in}
\setlength{\evensidemargin}{0in}
\setlength{\oddsidemargin}{0in}

\theoremstyle{plain}
\newtheorem{theorem}{Theorem}[section]
\newtheorem{proposition}[theorem]{Proposition}
\newtheorem{lemma}[theorem]{Lemma}
\newtheorem{corollary}[theorem]{Corollary}

\theoremstyle{definition}
\newtheorem{definition}[theorem]{Definition}
\newtheorem{assumption}[theorem]{Assumption}

\theoremstyle{remark}
\newtheorem{remark}[theorem]{Remark}
\newtheorem{example}[theorem]{Example}
\newtheorem{notation}[theorem]{Notation}

\begin{document}

\section*{Math 4324\  Winding Number and Degree } 

\subsection*{Hand in Friday, April 19.} 

\vspace{.15in}
\subsection*{Definition.}  Let $f : S \to C$ be a continuous map from a circle in the plane to a circle in the plane. Define the {\bf degree} of $f$ to be the winding number of this map around the point $\vec{c}$ at the center of $C$. If you prefer to think of winding numbers in terms of continuous maps from intervals, give the name $\theta$ to a variable running through the interval $[0,2\pi ]$, let $\gamma : [0,2\pi ] \to S$ parametrize $S$ by $\gamma (\theta ) = (x_0 + r\cos \theta , y_0 + r\sin \theta )$ for appropriate $x_0$, $y_0$, and $r$, and define the degree of $f$ to be the winding number of $f\circ \gamma$ around $\vec{c}$. [from the textbook by Fulton]

\vspace{.15in}
\noindent
\subsection*{1.}  Show that, for an $f$ as in the above definition, if $f$ is not surjective, then the degree of $f$ equals zero. 

\begin{proof}
    Using the parametrization $\gamma:[0,2\pi]\to S$ by $\gamma(\theta)=(x_0+r\cos \theta, y_0 +r\sin \theta )$ where $(x_0,y_0)$ is the center of $S$ and $r$ is the radius of $S$. Then we have that $f$ is a closed curve as $$f\circ \gamma(0)=f(x_0+r\cos 0, y_0 +r\sin 0)=f(x_0+r\cos 2\pi , y_0 +r\sin 2\pi)=f\circ \gamma(2\pi)$$
    
    Now let $\vec p_1\in C$ be a point that is not in the image of $f$. Then we have a point $\vec p_2\in C$ that is colinear with the line intersecting $(x_0,y_0)$ and $\vec p_1$. 

    We have the constant curve $g: S \to C$ given by the equation $g(\vec x) = \vec p_2$ for all $\vec x\in S$. 
    
    
    Then we create the homotopy $H:[0,2\pi]\times [0,1]\to \mathbb{R}^2\setminus \{(x_0,y_0)\}$ given by $$H(\theta,s)=f(\gamma (\theta))(1-s)+s\cdot \vec p_2$$ for all $\theta \in [0,2\pi]$ and $s\in [0,1]$. 

    We have $H(\theta,0)=f(\gamma (\theta))+0\cdot \vec p_2=f(\gamma(\theta))$, and $H(\theta,1)=f(\gamma(\theta ))\cdot 0 + 1\cdot \vec p_2=g(\gamma (\theta))$. 

    We have that $H$ is continuous as it is the sum of two weighted continuous functions. 
    
    Additionally we have that the image of $H$ is contained in $\mathbb{R}^2\setminus \{(x_0,y_0)\}$ as for any $\theta\in [0,2\pi]$ we have for all $s\in[0,1]$ that $H(\theta,s)\neq (x_0,y_0)$ as the only point colinear with  $(x_0,y_0)$ and $\vec{p_2}$ is $\vec p_1$ and by our assumption that $\vec p_1$ is not in the image of $f$. Then we have that $H$ is a homotopy between $f$ and $g$. 

    Then we have that $f$ and $g$ are homotopic and thus have the same winding number. We have that the winding number of $g$ is zero as it is a constant curve. Then we have that the winding number of $f$ is zero.


\end{proof}





\vspace{.15in}
\noindent
\subsection*{2.} Calculate the degree of each of the following maps from the unit circle centered at the origin to the unit circle centered at the origin. 



\vspace{.1in}
{\bfseries a.}  $f(x,y) = (x,y)$
{

Using the parametrization  $\gamma(\theta)=(\cos \theta ,\sin \theta)$, for $\theta \in [0,2\pi]$. With the four sectors $$U_1\{(x,y):x,y\in \mathbb{R} \text{ with }x>0\}$$
$$U_2=\{(x,y):x,y\in \mathbb{R} \text{ with }y>0\}$$
$$U_3=\{(x,y):x,y\in \mathbb{R} \text{ with } x<0\}$$
$$U_4=\{(x,y):x,y\in \mathbb{R} \text{ with } y<0\}$$

Then we have the the six subdivisions $t_0=0,t_1=\pi/4,t_2=3\pi/4,t_3=5\pi/4,t_4=7\pi/4,t_5=2\pi$. With the angle function being for $\theta_1$ being its angle in the range $(-\pi/2,\pi/2)$ for $\theta_2$ being its angle in the range $(0,\pi)$ and for $\theta_3$ being its angle in the range $(\pi/2,3\pi/2)$ and $\theta_4$ being its angle in the range $(\pi,2\pi)$. 

Then 

$W(f\circ \gamma ,\vec 0)=\frac{1}{2\pi}\big(\theta_1(f(\gamma(t_1)))-\theta_1(f(\gamma(t_1)))+\theta_2(f(\gamma(t_2)))-\theta_2(f(\gamma(t_1)))+\theta_3(f(\gamma(t_3)))-\theta_3(f(\gamma(t_2)))+\theta_4(f(\gamma(t_4)))-\theta_4(f(\gamma(t_3)))+\theta_1(f(\gamma(t_5)))-\theta_1(f(\gamma(t_4)))\big)$. 

Then $W(f\circ \gamma ,\vec 0)=\frac{1}{2\pi}\big( \frac{\pi}{4}-0 + \frac{3\pi}{4}-\frac{\pi}{4}+\frac{5\pi}{4}-\frac{3\pi}{4}+\frac{7\pi}{4}-\frac{5\pi}{4}+0 -\frac{\pi}{4} \big)=\frac{1}{2\pi}\big(\frac{7\pi}{4}+\frac{\pi}{4}\big)=1$



}

\vspace{.1in}
{\bfseries b.} $g(x,y) = (-x,-y)$

{ 
    Using the same parametrization with $\gamma$ as in {\bfseries a.} With the four sectors 
    $$U_1=\{(x,y): x,y\in \mathbb{R} \text{ with } x<0\}$$

    $$U_2=\{(x,y): x,y\in \mathbb{R} \text{ with } y<0\}$$

    $$U_3=\{(x,y): x,y\in \mathbb{R} \text{ with } x>0\}$$

    $$U_4=\{(x,y):x,y\in \mathbb{R} \text{ with } y>0\}$$

    With the same subdivisions as in {\bfseries a.} and the angle functions $\theta_1$ being its angle in the range $(\frac{\pi}{2},\frac{3\pi}{2})$ for $\theta_2$ being its angle in the range $(\pi,2\pi)$ and for $\theta_3$ being its angle in the range $(\frac{3\pi}{2},\frac{5\pi}{2})$ and $\theta_4$ being its angle in the range $(2\pi,3\pi)$. 

    Then $W(g\circ \gamma, \vec 0)=\frac{1}{2\pi}\big(\theta_1(g (\gamma (t_1)))-\theta_1(g(\gamma (t_0)))+\theta_2(g(\gamma(t_2)))-\theta_2(g(\gamma(t_3)))+\theta_3(g(\gamma(t_3)))-\theta_3(g(\gamma(t_2))) +\theta_4(g(\gamma(t_4)))-\theta_4(g(\gamma(t_3)))+\theta_1 (g(\gamma(t_5)))-\theta_1(g(\gamma(t_4))) \big)$

    Then $W(g\circ \gamma,\vec 0)=\frac{1}{2\pi}\big( \frac{5\pi}{4} - \pi+ \frac{7\pi}{4}-\frac{5\pi}{4}+\frac{9\pi}{4}-\frac{7\pi}{4} + 3\pi - \frac{9\pi}{4}  \big)=\frac{1}{2\pi}\big( 3\pi - \pi \big)=1$
}

\vspace{.1in}
{\bfseries c.} $h(x,y) = (x,-y)$

\vspace{.1in}
{\bfseries d.} $k(\cos (\theta ), \sin (\theta )) = (\cos (n\theta ), \sin (n\theta ))$, where $n$ is an arbitrary integer


\vspace{.15in}
\subsection*{Definition.}  If $Y$ is a topological subspace of a topological space $X$, a {\bf retraction} from $X$ to $Y$ is a continuous map $r : X \to Y$ that satisfies, for all $y\in Y$, $r(y) = y$. When such a retraction exists, we call $Y$ a {\bf retract} of $X$. [from the textbook by Fulton]

\vspace{.15in}
\noindent
\subsection*{3.} Show that, if $Y$ is a retract of $X$ and if every continuous map from $X$ to $X$ has a fixed point, then every continuous map from $Y$ to $Y$ has a fixed point. {\bf Hint.} Start with an arbitrary continuous map $f : Y\to Y$. How can you make a continuous map $g : X\to X$ whose behavior has the needed implications for $f$'s behavior?
 


\vspace{.15in}
\noindent
\subsection*{4.} Let $B$ be the open unit disk in $\mathbb R ^2$ and let $D$ be the closed unit disk in $\mathbb R ^2$. Show that, for any $\vec{p} \in B$, the unit circle $C$ in $\mathbb R ^2$ is a retract of $D\setminus \{ \vec{p}\}$. {\bf Hint.} When $\vec{p}$ is the origin, the map $\vec{x} \mapsto \frac{\vec{x}}{|\vec{x}|}$ is the retraction. When $\vec{p}$ is more general, consider solving $|\vec{p} + t(\vec{x}-\vec{p})| = 1$ for $t$. 



\vspace{.15in}
\noindent
\subsection*{5.} Let $S$ and $C$ be circles in the plane, and let $f : S\to C$ be a continuous map. Show that, for every $\vec{p}$ in the open disk bounded by $C$, the winding number of $f$ around $\vec{p}$ equals the degree of $f$. (In particular the winding number is the same, regardless of which $\vec{p}$ in the open disk is used.) 
 
\end{document}