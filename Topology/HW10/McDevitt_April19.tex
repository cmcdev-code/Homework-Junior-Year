\documentclass{amsart}
\usepackage{amsmath, amssymb, amscd}

\setlength{\textwidth}{6.5in}
\setlength{\textheight}{9in}
\setlength{\topmargin}{-.25in}
\setlength{\evensidemargin}{0in}
\setlength{\oddsidemargin}{0in}

\theoremstyle{plain}
\newtheorem{theorem}{Theorem}[section]
\newtheorem{proposition}[theorem]{Proposition}
\newtheorem{lemma}[theorem]{Lemma}
\newtheorem{corollary}[theorem]{Corollary}

\theoremstyle{definition}
\newtheorem{definition}[theorem]{Definition}
\newtheorem{assumption}[theorem]{Assumption}

\theoremstyle{remark}
\newtheorem{remark}[theorem]{Remark}
\newtheorem{example}[theorem]{Example}
\newtheorem{notation}[theorem]{Notation}

\begin{document}

\section*{Math 4324\  Winding Number and Degree } 

\subsection*{Hand in Friday, April 19.} 

\vspace{.15in}
\subsection*{Definition.}  Let $f : S \to C$ be a continuous map from a circle in the plane to a circle in the plane. Define the {\bf degree} of $f$ to be the winding number of this map around the point $\vec{c}$ at the center of $C$. If you prefer to think of winding numbers in terms of continuous maps from intervals, give the name $\theta$ to a variable running through the interval $[0,2\pi ]$, let $\gamma : [0,2\pi ] \to S$ parametrize $S$ by $\gamma (\theta ) = (x_0 + r\cos \theta , y_0 + r\sin \theta )$ for appropriate $x_0$, $y_0$, and $r$, and define the degree of $f$ to be the winding number of $f\circ \gamma$ around $\vec{c}$. [from the textbook by Fulton]

\vspace{.15in}
\noindent
\subsection*{1.}  Show that, for an $f$ as in the above definition, if $f$ is not surjective, then the degree of $f$ equals zero. 





\vspace{.15in}
\noindent
\subsection*{2.} Calculate the degree of each of the following maps from the unit circle centered at the origin to the unit circle centered at the origin. 



\vspace{.1in}
{\bfseries a.}  $f(x,y) = (x,y)$
{
Using the parametrization $\gamma(\theta)=(\cos \theta,\sin \theta)$ for $\theta \in [0,2\pi]$. 


I will be using the three sectors  $$U_1=\{(x,y): 0< \text{angle in polar}(x,y)<3\pi/2\}$$  
$$
U_2=\{(x,y): \pi/2 < \text{angle in polar}(x,y)<2\pi\}
$$
$$
U_3=\{(x,y): \pi < \text{angle in polar}(x,y)< 5\pi/2\}
$$

}

With the following four subdivisions $t_0= 0,t_1=\pi/2,t_2=\pi,t_3=2\pi$. 

Each angle function $\theta_i$ just gives the angle in polar coordinates.

Then 
\[
    W(f,\vec 0)=\frac{1}{2\pi}\big(\theta_1(\gamma (t_1))- \theta_1 (\gamma(t_0))+\theta_2(\gamma(t_2))-\theta_2(\gamma(t_1))+\theta_3(\gamma(t_3))-\theta_3(\gamma(t_2))\big)
\]

We have that for each angle function $\theta_i$ that $\theta_i(f(\gamma(t_i)))=\theta_i(\gamma(t_i))=t_i$.

After canceling terms in the equation we get $W(f,\vec 0)=\frac{1}{2\pi}\big(2\pi \big)=1$


\vspace{.1in}
{\bfseries b.} $g(x,y) = (-x,-y)$

{ 
    Using the same parametrization, sectors, intervals, and angle functions as in \bfseries a. As the angle functions agree at the intersection of the sectors we have that all in the equation will cancel with each other hence we get 
    \[
    W(g,\vec 0)=\frac{1}{2\pi}\big(\theta_3(g(\gamma (t_3))-\theta_1(g(\gamma(t_0))))\big)
    \]


}

\vspace{.1in}
{\bfseries c.} $h(x,y) = (x,-y)$

\vspace{.1in}
{\bfseries d.} $k(\cos (\theta ), \sin (\theta )) = (\cos (n\theta ), \sin (n\theta ))$, where $n$ is an arbitrary integer


\vspace{.15in}
\subsection*{Definition.}  If $Y$ is a topological subspace of a topological space $X$, a {\bf retraction} from $X$ to $Y$ is a continuous map $r : X \to Y$ that satisfies, for all $y\in Y$, $r(y) = y$. When such a retraction exists, we call $Y$ a {\bf retract} of $X$. [from the textbook by Fulton]

\vspace{.15in}
\noindent
\subsection*{3.} Show that, if $Y$ is a retract of $X$ and if every continuous map from $X$ to $X$ has a fixed point, then every continuous map from $Y$ to $Y$ has a fixed point. {\bf Hint.} Start with an arbitrary continuous map $f : Y\to Y$. How can you make a continuous map $g : X\to X$ whose behavior has the needed implications for $f$'s behavior?
 


\vspace{.15in}
\noindent
\subsection*{4.} Let $B$ be the open unit disk in $\mathbb R ^2$ and let $D$ be the closed unit disk in $\mathbb R ^2$. Show that, for any $\vec{p} \in B$, the unit circle $C$ in $\mathbb R ^2$ is a retract of $D\setminus \{ \vec{p}\}$. {\bf Hint.} When $\vec{p}$ is the origin, the map $\vec{x} \mapsto \frac{\vec{x}}{|\vec{x}|}$ is the retraction. When $\vec{p}$ is more general, consider solving $|\vec{p} + t(\vec{x}-\vec{p})| = 1$ for $t$. 



\vspace{.15in}
\noindent
\subsection*{5.} Let $S$ and $C$ be circles in the plane, and let $f : S\to C$ be a continuous map. Show that, for every $\vec{p}$ in the open disk bounded by $C$, the winding number of $f$ around $\vec{p}$ equals the degree of $f$. (In particular the winding number is the same, regardless of which $\vec{p}$ in the open disk is used.) 
 
\end{document}