\documentclass{amsart}
\usepackage{amsmath, amssymb, amscd}

\setlength{\textwidth}{6.5in}
\setlength{\textheight}{9in}
\setlength{\topmargin}{-.25in}
\setlength{\evensidemargin}{0in}
\setlength{\oddsidemargin}{0in}

\theoremstyle{plain}
\newtheorem{theorem}{Theorem}[section]
\newtheorem{proposition}[theorem]{Proposition}
\newtheorem{lemma}[theorem]{Lemma}
\newtheorem{corollary}[theorem]{Corollary}

\theoremstyle{definition}
\newtheorem{definition}[theorem]{Definition}
\newtheorem{assumption}[theorem]{Assumption}

\theoremstyle{remark}
\newtheorem{remark}[theorem]{Remark}
\newtheorem{example}[theorem]{Example}
\newtheorem{notation}[theorem]{Notation}

\begin{document}


\noindent
\subsection*{1.}
In each of the following topological spaces, give an example of an intersection of infinitely many open sets that is not itself an open set. 
\begin{enumerate}
    \item $\mathbb{R}$ with it's standard topology. Consider the intersection 
    \[
        \bigcap_{n\in \mathbb{N}}(-1/n,1/n)
    \] 
    We have $\bigcap_{n\in \mathbb{N}}(-1/n,1/n)=\{0\}$. This is shown as $-1/n<0<1/n$ for all $n\in \mathbb{N}$. Any element $j\in(0,1)$ is not in the intersection as there exists $n\in \mathbb{N}$ such that $1/n<j$ by the Archimedean property. Likewise for any $-j\in (-1,0)$ there exists a $n\in \mathbb{N}$ such that $-j<-1/n$ again by the Archimedean property. Now $\{0\}$ is not open as there exists no basis element in the standard topology that is a subset of $\{0\}$. This is shown as all basis elements are of the form $(a,b)$ where $a,b\in \mathbb{R}$ where $a<b$ but $|\{0\}|=1$ but $|(a,b)|$ is uncountable. 
    \item $\mathbb{R}$ with its lower limit topology. Consider \[\bigcap _{n\in \mathbb{N}}[0,1/n)\]. We have that $\bigcap_{n\in \mathbb{N}}[0,1/n)=\{0\}$ using the same reasoning as above. Now $|\{0\}|=1$ but any basis element $[a,b)$ where $a,b\in \mathbb{R}$ with $a<b$ is uncountable hence no basis element is a subset of $\{0\}$ which implies it is not open. 
    \item $\mathbb{R}$ with the finite complement topology. Consider \[\bigcap _{n\in \mathbb{N}}\mathbb{R}\setminus \{1/n\}\]. We have each $n\in \mathbb{N}$ that $\mathbb{R}\setminus (\mathbb{R}\setminus \{1/n\})=\{1/n\}$ hence $\mathbb{R}\setminus \{1/n\}$ is open. But $\bigcap_{n\in \mathbb{N}}\mathbb{R}\setminus \{1/n\}=\mathbb{R}\setminus \{1,1/2,1/3,...\}$ the complement of this set is not finite hence not open. 
\end{enumerate}

\noindent
\subsection*{2.}
Let $\mathbb{R}$ have the lower limit topology is $(0,1)$ open?
Yes 
\begin{proof}
Consider the union $\bigcup_{n\in \mathbb{N}}[1/n,1)$ as each of the sets is open and this is a union we have that it is open in the lower limit topology so just need to demonstrate double containment. Let $j\in \bigcup_{n\in \mathbb{N}}[1/n,1)$ then for some $n\in \mathbb{N}$ we have $j\in [1/n,1)$ but as $0<1/n$ for all $n\in \mathbb{N}$ we get the inequality $0<j<1$ hence $j\in (0,1)$. Now let $j\in (0,1)$ then by the Archimedean property for some $n\in \mathbb{N}$ we have $1/n<j<1$ hence $j\in \bigcup [1/n,1)$. Which shows double containment hence $\bigcup_{n\in \mathbb{N}}[1/n,1)=(0,1)$ which completes the proof.

\end{proof}

\subsection*{3}
In the set $\mathbb{R}$, consider the collection of subsets consisting of $\mathbb{R},\emptyset$, and all sets whose complements are finite sets of irrational numbers. Is this collection a topology on $\mathbb{R}$?

Yes
\begin{proof}
   As $\emptyset ,\mathbb{R}$ are in this collection $\mathcal{C}$ we just need to demonstrate finite intersections  and arbitrary unions are in $\mathcal{C}$.

   Consider the intersection of two elements $A,B\in \mathcal{C}$ then we have $\mathbb{R}\setminus {A\cap B}=(\mathbb{R}\setminus A) \cup (\mathbb{R}\setminus B)$ as the union of two finite sets is finite that completes the base case. Now assume for some $n\in \mathbb{N}$ where $n\geq 2$ we have that the intersection of $n$ elements of $\mathcal C$ is in $\mathcal{C}$. Then given $n+1$ elements $A_1,...,A_{n+1}$ consider the intersection $A_1\cap ...\cap A_{n+1}$ then we have $\mathbb{R}\setminus (A_1\cap...\cap A_2) =(\mathbb{R}\setminus A_1\cup ...\cup \mathbb{R}\setminus A_n)\cup \mathbb{R}\setminus A_{n+1}$ using the induction hypothesis we have $\mathbb{R}\setminus A_1\cup...\cup\mathbb{R}\setminus A_n\in \mathcal{C}$ by the base case the intersection of two elements of $\mathcal{C}$ is also in $\mathcal{C}$ hence that completes finite intersections.

   Let $\mathcal{B}\subset \mathcal{C}$ consider the arbitrary union of elements $\bigcup_{b\in \mathcal{B}}U_b$ where $U_b\in \mathcal{B}$. Then $\mathbb{R}\setminus\bigcup_{b\in \mathcal{B}}U_{b}\subset \mathbb{R}\setminus U_b$ where $U_b$ is any $U_b\in \mathcal B$ as subsets of finite sets are finite this shows that arbitrary unions are in $\mathcal C$ hence it is a topology. 

\end{proof}

\subsection*{4}
Suppose that $Y$ is a Hausdorff topological space. Let $a,b$ distinct elements of $Y$. Suppose that $(a_n)$ is a sequence in $Y$ that converges to $a$ and $(b_n)$ is a sequence in $Y$ that converges to $b$. Show that there exists an $N$ such that, for all $n>N$, $a_n\not = b_n$.

\begin{proof}
    As $Y$ is a Hausdorff space and $a,b$ are distinct elements then there exists two neighborhoods $U_a,U_b$ for $a,b$ respectively where $U_a\cap U_b=\emptyset$. But as $(a_n)$ is convergent we have for some $N_1\in \mathbb{N}$ that for all $n\geq N_1$ that $a_n\in U_a$. Likewise for $(b_n)$ for some $N_2\in \mathbb{N}$ we have for all $n\geq N_2$ that $b_n\in U_b$. Let $N=\max(N_1,N_2)$ then for all $n\geq N$ we have $a_n\in U_a$ and $b_n\in U_b$ but as these sets are disjoint we have $a_n\not = b_n$. 
\end{proof}

\subsection*{5}
\begin{enumerate}
    \item Show that, in any metric space $(X,d)$ and for any $x_0\in X$, the closure of $B_d(x_0,1)$ is contained in $D_d(x_0,1)$.
    \begin{proof}
        Let $(X,d)$ be an arbitrary metric space and $x_0\in X$. We have $\overline{B_d(x_0,1)}=B_d(x_0,1)\cup B_d(x_0,1)^\prime$. We have $B_d(x_0,1)\subset D_d(x_0,1)$ which follows from the strict inequality on $B_d(x_0,1)$. Now let $y\in B_d(x_0,1)^\prime$ then for every $\epsilon$-neighborhood we have for some distinct $x\in B_d(x_0,1)$ that $0<d(x,y)<\epsilon$ from the definition of the open ball we have $d(x_0,x)<1$ applying the triangle inequality we get $d(x_0,y)\leq d(x_0,x)+d(x_0,y)<1+\epsilon$ as this is true for all $\epsilon >0$ we get the strict inequality $d(x_0,y)\leq 1$ hence $y\in D_d(x_0,1)$ which shows $B_d(x_0,1)^\prime\subset D_d(x_0,1)$ as both the set and its limit points are subsets of $D_d(x_0,1)$ this completes the proof.

    \end{proof}
    \item Is it the case that, in every metric space $(X,d)$ and for every $x_0\in X, D_d(x_0,1)$ is contained in the closure of $B_d(x_0,1)$?
    \begin{proof}
        No this is not true consider the metric $(\mathbb{R},d)$ where $d(x,y)=\begin{cases}
            1 \; & \; \text{ if } x\not = y \\
            0 \; & \; \text { if } x =y
        \end{cases}$
        . Consider the open ball $B_d(0,1)=\{0\}$ which follows because of the strict inequality but $D_d(0,1)=\mathbb{R}$. The closure of $B_d(0,1)=B_d(0,1)\cup B_d(0,1)^\prime$ but for any $x\in \mathbb{R}$ with $x\not = 0 $ we have that any epsilon neighborhood of $x$ with $0<\epsilon<1$ that $B_d(x,\epsilon)=\{x\}$ hence not every neighborhoods even intersects $B_d(0,1)$ so the limit point is the emptyset. Therefore $\overline{B_d(0,1)}=\{0\}\not \supset D_d(0,1)=\mathbb{R}$.
    \end{proof}
\end{enumerate}

\subsection*{6} 
For all natural numbers $j$ let $X_j$ be $\mathbb{R}$ with the standard topology. Let $X=\prod_{j\in \mathbb{N}}X_j$ define the elements of $X$ by $\vec x=(x_1,x_2,...)$

\begin{enumerate}
    \item Show that the set of $5-$bounded elements of $X$ is closed in both the product topology and the box topology.
    \begin{proof}
        Denote the set of $5$ bounded elements by $B_5$ we have that a set is closed if and only if it contains it's limit points. Let $\vec{x}\in B_5$ then we have for any neighborhood $U$ of $\vec{x}$ that $U_i\not=  \mathbb R_i$ for a finite number of $i$. As this is the product topology we have for each $U_i\not = \mathbb{R}_i$  as $U_i$ is open a basis element $u_i=(a_i,b_i)\subset U_i $ where $a_i,b_i\in \mathbb{R}$ with $-5\leq a_i<\pi_i(\vec{x})<b_i\leq 5$ then we have the set $A=\mathbb{R}_1 \times u_2\times ...\times u_i\times ...\times \mathbb R_n \times ...$ is open and $A\subset U$ and $A\cap B_5\setminus \{\vec{x}\}\not = \emptyset$ as there exists some $\vec{y}\in B_5\setminus \{\vec{x}\}$ with $a_i<\pi_i(\vec{y})<b_i$ where the $a_i,b_i$ come from the bases element $u_i$ when $U_i\not = \mathbb{R}_i$ if $U_i=\mathbb{R}_i$ then $\pi_i(\vec{y})\in [-5,5]$ hence this shows $A\cap B_5\setminus \{\vec{x}\}\not = \emptyset$. Which implies every element of $B_5$ is a limit point of $B_5$. 

        If $\vec{x}\not \in B_5$ then there exists some $i\in \mathbb{N}$ we have $|\pi_i(\vec{x})|> 5$. If $\pi_i(\vec{x})>5$ then consider the open set $\pi_i^{-1}((a,b))$ where $a,b\in \mathbb{R}$ with $5<a<\pi_i(\vec{x})<b$. We have $\pi_i^{-1}((a,b))\cap B_5=\emptyset$ as every element of $B_5$ is $5-$bounded. If $\pi_i(\vec{x})<-5$ then we we have the open set $\pi_i^{-1}(a,b)$ with $a<\pi_i(\vec{x})<b<-5$ again this open set $\pi_i^{-1}(a,b)\cap B_5=\emptyset$ as $B_5$ is 5 bounded. This shows that $B_5$ contains its limit points hence it is closed in the product topology.

        We have that the box topology is finer than the product topology. Which implies that $B_5$ if it is closed in the product topology then it is also closed in the box topology.

    \end{proof}
    \item Show that the set of bounded elements of $X$ is closed in the box topology.
    \begin{proof}
    Denote the set of bounded elements by $B$. We have that $B$ is closed if and only if it contains it's limit points. Let $\vec{x} \in B$ then consider the open set $U=\prod_{i\in \mathbb{N}}(\pi_i(\vec{x})-1,\pi(\vec{x})+1)$ we have that $U\cap B=\emptyset$ because if it did not then there would exist $\vec{y}\in B$ with $|\pi_i(\vec{x})-\pi_i(\vec{y})|<1$ for all $i\in \mathbb{N}$ but as $\vec{y}$ is bounded we have for some $M\in \mathbb{R}$ with $M>0$ that $-M\leq \pi_i(\vec{y})\leq M$ for all $i\in \mathbb{N}$. Then we have for all $i\in \mathbb{N}$ that $|\pi_i(\vec{x})-\pi_i(\vec{y})|<1$ which by the triangle inequality yields $|\pi_i(\vec{x})|-|\pi_i(\vec{y})|<1$ which implies $|\pi_i(\vec{x})|<1+M$ which is a contradiction on $\vec{x}$ being bounded. Hence $U\cap B=\emptyset$. Now suppose that $\vec{x}$ is bounded 
    \end{proof}
        
    
\end{enumerate}


\end{document}