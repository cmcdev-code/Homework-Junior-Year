\documentclass{amsart}
\usepackage{amsmath, amssymb, amscd}

\usepackage{enumitem}


\setlength{\textwidth}{6.5in}
\setlength{\textheight}{9in}
\setlength{\topmargin}{-.25in}
\setlength{\evensidemargin}{0in}
\setlength{\oddsidemargin}{0in}

\theoremstyle{plain}
\newtheorem{theorem}{Theorem}[section]
\newtheorem{proposition}[theorem]{Proposition}
\newtheorem{lemma}[theorem]{Lemma}
\newtheorem{corollary}[theorem]{Corollary}

\theoremstyle{definition}
\newtheorem{definition}[theorem]{Definition}
\newtheorem{assumption}[theorem]{Assumption}

\theoremstyle{remark}
\newtheorem{remark}[theorem]{Remark}
\newtheorem{example}[theorem]{Example}
\newtheorem{notation}[theorem]{Notation}

\begin{document}


\noindent
\subsection*{1.}

\begin{proof}
    Suppose $X$ is a compact topological space and $f: X\to \mathbb R$ is as described. As $X$ is compact we have the open cover $\{U_x\}_{x\in X}$ where for each $U_x\in \{U_x\}_{x\in X}$ we have that there exists a constant $M_x$ such that for all $z\in U_x$ that $|f(z)|\leq M_x$. 
    
    Then as $X$ is compact we have that the collection has a finite subcover $\{U_{x_1},...,U_{x_n}\}$. Each of these in the finite subcover has a corresponding constant $\{M_{x_1},...,M_{x_n}\}$ with the property that for all $z\in U_{x_i}$ that $|f(z)|\leq M_{x_i}$ where $i=1,...,n$. Let $M=\max\{M_{x_1},...,M_{x_n}\}$ then for any $z\in X$ we have that $z\in U_{x_i}$ for some $i=1,...,n$. Hence $|f(z)|\leq M_{x_i}\leq M$ so this choice of $M$ works for all $z\in X$.
\end{proof}

\subsection*{2.}

\begin{proof}
    Suppose that $X$ is a topological space $C$ is a connected subset of $X$ and $C_{\alpha}$ is a connected subset of $X$. With for all $\alpha A$ that $C_\alpha \bigcap C\not = \emptyset$. Suppose that $C\bigcup \big ( \cup _{\alpha \in A} C_\alpha \big)$ is not connected. Then there exists a separation $U\cup V$ of $C\bigcup \big(\cup _{\alpha \in A} C_\alpha \big)$. 
    
    Then as $C$ is connected we have either $C\subset U$ with $C\cap V=\emptyset$ or $C\subset V$ with $C\cap U=\emptyset$. If $C$ wasn't fully contained in only one then $(C\cap U)$, $(C\cap V)$ would be a separation of $C$ but $C$ is connected. For each $\alpha \in A$ we get that $C_\alpha$ is contained in exactly one $U$ or $V$.
    
    WLOG suppose $C\cup U$ then as $C\cap C_\alpha \not = \emptyset $ we get that $C_\alpha \subset U$. Hence we get $C\bigcup \big(\cup _{\alpha \in A}C_\alpha \big)\subset U$ and $\left(C\bigcup \big(U_{\alpha\in A} C_\alpha \big)\right) \bigcap V=\emptyset$ hence $V=\emptyset$ so this separation can not exist. So we have $C\bigcup \big( \cup _{\alpha \in A}C_\alpha \big)$ is connected. 
\end{proof}

\subsection*{3.}
\begin{enumerate}[label=(\alph*)]
    \item {We have that a set is closed if and only if it contains it's limit points. As any singleton in $\{1/n: n\in \mathbb{N}\}$ is ope. Which is shown as we have for any $1/n\in \{1/n:n\in \mathbb{N}\}$ we have that $(1/n-\epsilon, 1/n+\epsilon)$ for $\epsilon >0$ is open in $\mathbb{R}$ choosing a sufficiently small $\epsilon>0$ we get $(1/n-\epsilon,1/n+\epsilon)\cap \{1/n:n\in \mathbb{N}\}=\{1/n\}$. As the choice of $1/n$ was arbitrary we get that all singletons are open.
    
    Hence any $B\subset \{1/n: n\in \mathbb N\}$ we get that $B^\prime=\emptyset$ as for any $b\in \{1/n:n\in \mathbb N\}$ we have the neighborhood $\{b\}$ that doesn't intersect $B$ at any place other then possibly itself. Hence $B$ is closed as $B$ is arbitrary we have that all subsets are closed. 
    }
    \item {We have that all singletons other then $\{0\}$ are open using the same reasoning as above. Hence $\{1/n: n\in \mathbb{N}\}$ is open so we get $\{1/n: n\in \mathbb{N}\}^c=\{0\}$ is closed. We have that all singletons are closed as any singleton other than $\{0\}$ don't have any limit points. As finite unions of closed sets are closed then all finite subsets are closed. 
    
    Lastly any infinite set containing $0$ is closed. This is shown as let $B\subset \{0\}\cup \{1/n:n\in \mathbb{N}\}$ be an infinite set containing $0$. Then we have $B^c$ is a union of singletons each of which are open hence $B^c$ is open so $B$ is closed. 
    
    Any infinite set not containing $0$ is not closed as $0$ is a limit point of said set. 
    }
\end{enumerate}

\subsection*{4.}
\begin{proof}


    $\\ (\rightarrow)$


    Assume that $f$ is continuous and that there exists a $m,n\in \mathbb{N}$ with $m<n$ where $f(m)> f(n)$. Choose the neighborhood $U_{f(n)}=\{1,...,f(n)\}$ of $f(n)$. Then we have that there exists a neighborhood $U_n$ of $n$ where $f(U_n)\subset U_{f(n)}$. From the definition of the open sets we get that $m\in U_n$ but as $f(m)> f(n)$ we have $f(m)\in f(U_n)$ but $f(m)\not \in U_{f(n)}$. This is a contradiction on $f$ being continuous hence $f(m)\leq f(n)$

    $(\leftarrow)$


    Let $n\in \mathbb{N}$ then take an arbitrary neighborhood $U_{f(n)}$ of $f(n)$. Then we have $U_{n}=\{1,...,n\}$ that $f(U_n)\subset U_{f(n)}$ this follows as $f(n)\in f(U_n)$ and $f(n)\in U_{f(n)}$ and for any $m\in U_n$ with $m<n$ we have that $f(m)\leq f(n)$ hence $f(m)\in U_{f(n)}$. This is one of the equivalent definitions of continuity. Hence $f$ is continuous.

\end{proof}



\end{document}