\documentclass{amsart}
\usepackage{amsmath, amssymb, amscd}

\setlength{\textwidth}{6.5in}
\setlength{\textheight}{9in}
\setlength{\topmargin}{-.25in}
\setlength{\evensidemargin}{0in}
\setlength{\oddsidemargin}{0in}

\theoremstyle{plain}
\newtheorem{theorem}{Theorem}[section]
\newtheorem{proposition}[theorem]{Proposition}
\newtheorem{lemma}[theorem]{Lemma}
\newtheorem{corollary}[theorem]{Corollary}

\theoremstyle{definition}
\newtheorem{definition}[theorem]{Definition}
\newtheorem{assumption}[theorem]{Assumption}

\theoremstyle{remark}
\newtheorem{remark}[theorem]{Remark}
\newtheorem{example}[theorem]{Example}
\newtheorem{notation}[theorem]{Notation}

\begin{document}

\section*{Math 4324\  Product spaces } 

\subsection*{Hand in Friday, February 9.} 
\vspace{.15in}


\noindent
\subsection*{1.}  Suppose that  $X = \prod _{\alpha \in A} X_{\alpha}$ is the product space formed by an arbitrary collection of topological spaces $X_{\alpha}$. Among the assertions you are asked to prove in this problem is an assertion stated by Munkres as a theorem but not proven in Munkres. Prove the assertion; don't just quote Munkres's theorem. 

\vspace{.1in}
{\bfseries a.} Assume that $X$ has the product topology. Show that $X$ is Hausdorff if and only if every $X_{\alpha}$ is Hausdorff. 


\begin{proof}
    Assume that $X$ is Hausdorff with the product topology. Let $x,y\in X$ where $\pi_j(x)=\pi_j(y)$ for all $j\in A\setminus \{\alpha\}$ with $\pi_\alpha(x) \not = \pi_\alpha(y)$ for a single $\alpha \in A$. Then as $x\not = y$ as they differ in the $\alpha'th$ index we have that there are two open sets $U_1$ and $U_2$ where $x\in U_1$ and $y\in U_2$ and $U_1\cap U_2 =\emptyset$. However as all indices of $x,y$ are equal in all but the $\alpha$'th index we have that for all $\beta \in A$ with $\beta\not = \alpha$ that any two open sets $U_{\beta_1},U_{\beta_2}\in X_\beta$ with $\pi_\beta(x)\in U_{\beta_1}$ and $\pi_\beta(y)\in U_{\beta_2}$ that $U_{\beta_1}\cap U_{\beta_2}\not = \emptyset$. But as $X$ is Hausdorff we have that there are two open sets $U_{\alpha_1}, U_{\alpha_2}$ in $X_\alpha$ such that $\pi_\alpha(x) \in U_{\alpha_1}$ and $\pi_\alpha(y) \in U{\alpha_2}$ but $U_{\alpha_1}\cap U_{\alpha_2}=\emptyset$. As $\alpha$ was chosen arbitrarily we have that every $X_\alpha$ is Hausdorff. 

    Now assume that each $X_\alpha$ is Hausdorff. Then let $x,y\in X$ with $x\not = y$ then for some $\beta\in A$ we have that $\pi_\beta(x)\not = \pi_\beta(y)$. As each $X_\alpha$ is Hausdorff take the two sets $U_{\beta_1},U_{\beta_2}\in X_\beta$ where $x_\beta \in U_{\beta_1},y_\beta \in U_{\beta_2}$ but $U_{\beta_1}\cap U_{\beta_2}=\emptyset$. Now take the two open sets in the product topology $\pi_{\beta}^{-1}(U_{\beta_1})$ and $\pi_{\beta}^{-1}(U_{\beta_2})$ we have that $x\in\pi_{\beta}^{-1}(U_{\beta_1}) $ and $y\in\pi_{\beta}^{-1}(U_{\beta_2})$ but $\pi_{\beta}^{-1}(U_{\beta_1})\cap U_\beta^{-1}(\beta_2)=\emptyset$ as $x,y$ are arbitrarily we have that $X$ is Hausdorff.
    

\end{proof}



\vspace{.1in}
{\bfseries b.} Assume that $X$ has the box topology. Show that $X$ is Hausdorff if and only if every $X_{\alpha}$ is Hausdorff. 


\begin{proof}
    We can show that the box topology is finer than the product topology. This is shown as let $x\in X$ and $x\in U$ where $U=\prod U_\alpha$ where $U_\alpha=X_\alpha$ in all but a finite number of $\alpha$ (Munkres Theorem 19.1). We have that $U$ is also in the basis for the box topology hence the box topology is finer than the product topology. As all open sets of the product topology are open in the box topology the proof in {\bfseries a}. also holds.
\end{proof}


\vspace{.15in}
\noindent
\subsection*{2.} Suppose that $X = X_1 \times X_2$ is a product space formed by two infinite topological spaces, each with the finite-complement topology. Show that both the box and product topologies on $X$ are strictly finer than the finite-complement topology on $X$. 


As $X$ is the cartesian product of a finite number of sets we have that both the box and product topology are the same. So in the proof we only need to consider one and it will hold for the other. 

For containment I will be using Munkres Lemma 13.3. 

Let $x\in X$ with $x\in X_1\setminus \{x_{1_1},x_{1_2},....,x_{1_n}\}\times X_2\setminus \{x_{2_1},x_{2_2},...,x_{2_m}\}$ where $x_{1_{i}}\in X_1$ for $1 \leq i\leq n$ and $x_{2_j}\in X_2$ where $1\leq j \leq m$ for some $n,m\in \mathbb{N}$. Then as $X_1\setminus \{x_{1_1},x_{1_2},....,x_{1_n}\}, X_2\setminus \{x_{2_1},x_{2_2},...,x_{2_m}\}$ are open in $X_1,X_2$ respectively we have by the definition of product topology that $X_1\setminus \{x_{1_1},x_{1_2},....,x_{1_n}\}\times X_2\setminus \{x_{2_1},x_{2_2},...,x_{2_m}\}$ is in the product topology. Hence $x\in X_1\setminus \{x_{1_1},x_{1_2},....,x_{1_n}\}\times X_2\setminus \{x_{2_1},x_{2_2},...,x_{2_m}\}\subset X_1\setminus \{x_{1_1},x_{1_2},....,x_{1_n}\}\times X_2\setminus \{x_{2_1},x_{2_2},...,x_{2_m}\}$ therefore by Munkres lemma 13.3 that the product topology is finer then the finite complement topology.

Now to prove that the product topology is strictly finer then the finite complement topology.

We have that $X_1\times (X_2\setminus \{x_2\})$ where $x_2\in X_2$ is in the product topology as both $X_1,X_2\setminus \{x_2\}$ are open in their finite complement topologies. However as $X_1\times (X_2\setminus \{x_2\})=X\setminus (\cup_{x\in X_1} (x,x_2))$ is not in the finite complement topology we have that the product topology is strictly finer than the finite complement topology.


\vspace{.15in}

\noindent
\subsection*{3.} Let $X =\prod _{k\in \mathbb N} X_k$ be the product of countably many copies of the real line $\mathbb R$  (that is, every $X_k$ is $\mathbb R$), with each factor $\mathbb R$ having the topology it gets from the standard metric on $\mathbb R$. Let $X$ have the box topology. Let $(\vec{x}_n)$ be a sequence of elements of $X$, where, for each $n$, $\vec{x}_n$ has its $n^{th}$ entry equal to $1$ and the rest of its entries equal to $0$.  Show that $(\vec{x}_n)$ does not converge to the element of $X$ represented by the vector in which every entry is zero. 

\begin{proof}

    As $(-1,1)$ is open in $\mathbb{R}$ we have that $\prod_{k\in \mathbb{N}}(-1,1)$ is in the box topology. Then for all $\vec{x_n}$ in the sequence $(\vec x_n)$ we have that $\vec{x_n}\not \in \prod_{k\in \mathbb{N}}(-1,1)$ as $1\not \in (-1,1)$ and given any $n\in \mathbb{N}$ we have that the $n+1$'th element of $\vec{x_n}$ is $1$ hence it is not in $\prod_{k\in \mathbb{N}}(-1,1)$. As the vector with all zeros is in $\prod_{k\in \mathbb{N}}(-1,1)$ we have that it does not converge.

    
\end{proof}



\vspace{.15in}

\noindent
\subsection*{4.} Suppose that  $X = \prod _{\alpha \in A} X_{\alpha}$ is the product space formed by an arbitrary collection of topological spaces $X_{\alpha}$. Let $X$ have the product topology. Let $(\vec{x}_n)$ be a sequence in $X$. Show that this sequence converges to $\vec{x} \in X$ if and only if for every $\alpha \in A$, the sequence defined by $f(n) = \pi _{\alpha} (\vec{x}_n)$ converges to $ \pi _{\alpha} (\vec{x})$. {\bfseries Aside.} Another way to state what you're asked to prove is : the sequence $(\vec{x}_n)$ converges to $\vec{x} \in X$ if and only if for every $\alpha \in A$, the sequence $\pi _{\alpha} (\vec{x}_1) , \pi _{\alpha} (\vec{x}_2), \pi _{\alpha} (\vec{x}_3) , . . . $ in $X_{\alpha}$ converges to $ \pi _{\alpha} (\vec{x})$ in $X_{\alpha}$.


\begin{proof}
    Suppose $(\vec{x}_n)$ is a sequence in $X$ and assume that it converges to $\vec{x}\in X$ let the sequence $\pi_\alpha(\vec{x}_n)$ be as stated assume that there exists an $\alpha \in A$ such that $f(n)=\pi_\alpha(\vec x_n)$ does not converge to $\pi_\alpha(\vec x)$. Then there exists an open set $U_\alpha\in X_\alpha$ we have that for all $N\in \mathbb{N}$ such that there exists $n\in \mathbb{N}$ with $n>N$ that $\pi_\alpha(\vec x_n)\not \in U_\alpha$. However as $\pi_\alpha^{-1}(U_\alpha)$ is open and $\vec{x} \in \pi_\alpha^{-1}(U_\alpha)$ we have that for some $M\in \mathbb{N}$ for all $m>M$  that $\vec x_m\in \pi_\alpha^{-1}(U_\alpha)$ this implies that $\pi_\alpha(\vec x_m) \in U_\alpha$ which contradicts the assumption that $\pi_\alpha(\vec x_n)$ does not converge. 
    
    Assume that for every $\alpha\in A$ we have that the sequence $f(n)=\pi_\alpha(\vec v_n)$ converges to $\pi_{\alpha}(\vec x)$. Then given any open set $U$ with $x\in U$ we have by Munkres Theorem 19.1 that there is a finite number of $U_\alpha$ where $X_\alpha$ is a proper subset of $U_\alpha$. For the finite number $U_\alpha$ from $U$ where $U_\alpha\not = X_\alpha$ choose the minimum $N$ for each such that $\pi_\alpha(\vec{x}_n)$ converges to $\pi_\alpha(\vec{x})$ take the maximum $N$ from that and we have that for all $\vec{x_n}$ with $n>N$ that $\vec{x_n}\in U$. This shows that $(\vec x_n)$ converges to $\vec x$.
\end{proof}


\vspace{.15in}

\noindent
\subsection*{5.} Let $X =\prod _{k\in \mathbb N} X_k$ be the product of countably many copies of the real line $\mathbb R$  (that is, every $X_k$ is $\mathbb R$), with each factor $\mathbb R$ having the topology it gets from the standard metric on $\mathbb R$. Let $X$ have the box topology. Let $(\vec{x}_n)$ be a sequence in $X$ that satisfies for all $k$ and for all $n$, $\pi_k  (\vec{x}_n) = 1/n$. Show that $(\vec{x}_n)$ does not converge to the element of $X$ represented by the vector in which every entry is zero. {\bfseries Aside.} An equivalent way to describe $(\vec{x}_n)$ is to say that, for each $n$, the $n^{th}$ value of the sequence, $\vec{x}_n$, is the vector with every entry equal to $1/n$.

 \begin{proof}
    We have that for all $n\in \mathbb{N}$ that $(-\frac{1}{n},\frac{1}{n})$ is open in $\mathbb{R}$. So as this is the box topology we have that $\prod_{k\in\mathbb{N}}(-\frac{1}{k},\frac{1}{k})$ is open. As the element of all zeros is in $\prod_{k\in\mathbb{N}}(-\frac{1}{k},\frac{1}{k})$ we have that for all $n \in \mathbb{N}$ sequence $\vec{x}_n\not \in  \prod_{k\in\mathbb{N}}(-\frac{1}{k},\frac{1}{k})$ because we have when $k\geq n$ that $\pi_k(\vec{x}_n)=\frac{1}{n}$ but $\frac{1}{k}\leq \frac{1}{n}$. This shows that the sequence $(\vec{x}_n)$ does not converge to the element of $X$ of all zeros.

 \end{proof}

 
\end{document}