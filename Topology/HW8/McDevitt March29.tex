\documentclass{amsart}
\usepackage{amsmath, amssymb, amscd}

\setlength{\textwidth}{6.5in}
\setlength{\textheight}{9in}
\setlength{\topmargin}{-.25in}
\setlength{\evensidemargin}{0in}
\setlength{\oddsidemargin}{0in}

\theoremstyle{plain}
\newtheorem{theorem}{Theorem}[section]
\newtheorem{proposition}[theorem]{Proposition}
\newtheorem{lemma}[theorem]{Lemma}
\newtheorem{corollary}[theorem]{Corollary}

\theoremstyle{definition}
\newtheorem{definition}[theorem]{Definition}
\newtheorem{assumption}[theorem]{Assumption}

\theoremstyle{remark}
\newtheorem{remark}[theorem]{Remark}
\newtheorem{example}[theorem]{Example}
\newtheorem{notation}[theorem]{Notation}

\begin{document}

\section*{Math 4324\  Second Compactness Assignment} 

\subsection*{Hand in Friday, March 29.} 

\vspace{.15in}
\noindent
\subsection*{1.}  Show that, if the topological space $X$ is compact, then every infinite subset of $X$ has an $\omega$-accumulation point in $X$, i.e. that every compact space is $\omega$-accumulation-point-compact. {\bfseries Hint.} If $Y$ is a subset that has no $\omega$-accumulation point, choose a neighborhood of each point of $X$ that helps you show that $Y$ must be finite. 

\begin{proof}
    Let $X$ be a compact topological space and $Y\subset X$ be a set with no $\omega-$accumulation point. Then for each $y \in Y$ we have a neighborhood $y\in U_y$ but $\mathcal U_y=Y\cap U_y$ is finite. Then we create the open cover $\{\mathcal U_y\}_{y\in Y}$ but as $X$ is compact we have a finite subcover $\{\mathcal U_{y_1},\ldots,\mathcal U_{y_n}\}$. But then $Y= \bigcup_{i=1}^n \mathcal U_{y_i}$ and so $Y$ is finite. This shows that any set that does not have an $\omega-$accumulation point is finite hence any infinite set must have an $\omega-$accumulation point.
\end{proof}


\vspace{.15in}
\noindent
\subsection*{2.} Show that, if a topological space $X$ is sequentially compact (every sequence in $X$ has a subsequence converging to a point in $X$), then $X$ is $\omega$-accumulation-point-compact. 

\begin{proof}
    Assume that $X$ is a sequentially compact. Then take an arbitrary infinite set $A\subset X$ then create an arbitrary sequence $\{a_n\}_{n\in \mathbb{N}}$ in $A$. 
    
    
    Then as $X$ is sequentially compact we have a subsequence $\{a_{n_k}\}_{k\in \mathbb{N}}$ for some $a\in A$ we have any neighborhood $U$ of $a$ which contains all points of the subsequence after some $n_k\in \mathbb{N}$. 
    Hence $U$ contains an infinite number of elements of $A$ we have that $a$ is an $\omega-$accumulation point of $A$. As $A$ was an arbitrary infinite set we have that $X$ is $\omega-$accumulation-point-compact.
\end{proof}

\vspace{.15in}
\noindent
\subsection*{3.} Show that, if a topological space $X$ is $\omega$-accumulation-point-compact and first countable, then $X$ is sequentially compact. {\bfseries Hint.} For an arbitrary sequence, show that, if no single value appears with infinitely many indices, then the sequence must have infinitely many distinct values. Use first countable to show that an $\omega$-accumulation point is a limit of a subsequence. 


\begin{proof}
    Let $X$ be a first countable $\omega-$accumulation-point-compact space. Then take an arbitrary sequence $\{x_n\}_{n\in \mathbb{N}}$ in $X$. In the event that some value in our sequence occurs an infinite number of times we can just create the constant subsequence and we are done. So we have that each value in our sequence occurs finitely many times hence our sequence contains infinitely many distinct points. 

    Then as $\{x_n\}_{n\in \mathbb{N}}$ is an infinite subset of $X$ we have for some $x \in \{x_n\}_{n\in \mathbb{N}}$ is an $\omega-$accumulation point. 
    
    Now as $X$ is first countable take the countable basis $\{B_n\}_{n\in \mathbb{N}}$ of $x$ each of which intersects $\{x_n\}_{n\in \mathbb{N}}$ an infinite number of times. Then create the subsequence $x_{n_k}\in \bigcap_{i=1}^{n_k} (B_i ) \cap \{x_n\}_{n\in \mathbb{N}}$ where for each $n_k<n_{k+1}$ such an element is always guaranteed to exist as the intersection is infinite. 
    
    Lastly to show that this subsequence converges and has limit point $x$. Let $U_x$ be an arbitrary neighborhood of $x$ then we have as $X$ is first countable that for some $B\in \{B_n\}_{n\in \mathbb N}$  we have $B\subset U_x$ and for some $n_K\in \mathbb{N}$ we have that for all $n_k \geq n_K$ that $x_{n_k} \in B\subset U_x$ hence we have that this subsequence converges to $x$ and so $X$ is sequentially compact.

    
\end{proof}


\vspace{.15in}
\noindent
\subsection*{4.} 

\noindent
{\bfseries a.} Show that, if a topological space $X$ is second countable, then every open cover of $X$ has a countable subcover.

\begin{proof}
   Suppose that $X$ is second countable and $A\subset X$ and $\{U_a\}_{a\in J}$ is an arbitrary open covering of $A$. Then as $X$ is second countable we have that the covering $\{B_a\}_{a\in A}$ where $B_a$ is a basis element is a countable covering of $A$. Then we create the subcover \[\{U_a: \text{ for each }B\in  \{B_a\}_{a\in A}  \text{ take a single } a\in J \text{ such that } B\subset U_a\}\]. 

   This subcover is countable becuase we choose the elements by examining the elements of the basis cover which was countable. 




\end{proof}

\vspace{.1in}
\noindent
{\bfseries b.} Show that every second countable, $\omega$-accumulation-point-compact space $X$ is compact. {\bfseries Hint.} For a countable open cover $\{ U_j\}$ with no finite subcover, construct an infinite subset $Y$ of $X$ with the property that each $U_j$ contains at most finitely many elements of $Y$. 

\begin{proof}
    Let $X$ be a second countable $\omega-$accumulation-point-compact space. By part (a) of this problem we have that all open covers of a second countable space have a countable subcover so it suffices to only examine arbitrary countable covers.

    Let $\{U_n\}_{n\in \mathbb{N}}$ be a countable open cover with no finite subcover as this collection has no finite subcover we have that there is an infinite $\{U_{n_k}\}_{n_k\in \mathbb{N}}$ that contain at least one unique point. Then we create the set $Y=\{y_1,y_2,...\}$ where $y_i$ is a unique point of $U_{n_i}$  . Now despite $Y$ being an infinite set we have that it has no $\omega-$point as each point in $Y$ is contained in a single $U_n$. This would contradict the $\omega-$accumulation-point-compactness of $X$ so the assumption that $\{U_n\}_{n\in \mathbb{N}}$ has no finite subcover is false which implies that $X$ is compact.

\end{proof}

\vspace{.15in}
\begin{definition} In a metric space $(X,d)$, the {\it diameter} of a nonempty bounded subset $Y$ is the supremum (least upper bound) of $\{ d(y_1, y_2) : y_1, y_2 \in Y\}$. \end{definition}

\vspace{.15in}
\begin{definition} Let $\{ U_{\alpha}\} _{\alpha \in A}$ be an open cover of a metric space $X$. A {\it Lebesgue number} for this cover is a number $\delta$ satisfying: for every subset $Y$ of $X$ with diameter less than $\delta$, there is a $U_{\alpha}$ for which $Y\subset U_{\alpha}$. \end{definition}

 
\end{document}