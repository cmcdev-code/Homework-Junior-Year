\documentclass{amsart}
\usepackage{amsmath, amssymb, amscd}

\setlength{\textwidth}{6.5in}
\setlength{\textheight}{9in}
\setlength{\topmargin}{-.25in}
\setlength{\evensidemargin}{0in}
\setlength{\oddsidemargin}{0in}

\theoremstyle{plain}
\newtheorem{theorem}{Theorem}[section]
\newtheorem{proposition}[theorem]{Proposition}
\newtheorem{lemma}[theorem]{Lemma}
\newtheorem{corollary}[theorem]{Corollary}

\theoremstyle{definition}
\newtheorem{definition}[theorem]{Definition}
\newtheorem{assumption}[theorem]{Assumption}

\theoremstyle{remark}
\newtheorem{remark}[theorem]{Remark}
\newtheorem{example}[theorem]{Example}
\newtheorem{notation}[theorem]{Notation}

\begin{document}

\section*{Math 4324\  Order Topology and Introduction to Continuity } 

\subsection*{Hand in Friday, March 1.} 




\vspace{.15in}
\noindent
\subsection*{1.}  Recall that a map $f : X \rightarrow Y$ between topological spaces is said to be continuous at a point $x_0 \in X$ if and only if, for each open set $V$ that contains $f(x_0)$, there is an open $U$ satisfying $x_0 \in U \subset f^{-1}(V)$. Let $\mathcal B _X$ be a basis for the topology of $X$ and $\mathcal B _Y$ a basis for the topology of $Y$. Show that $f$ is continuous at $x_0 \in X$ if and only if, for each $W\in \mathcal B _Y$ that contains $f(x_0)$, there is a $\widetilde{W} \in \mathcal B _X$ satisfying $x_0 \in \widetilde{W} \subset f^{-1} (W)$. 


\begin{proof}
    Suppose that $f:X\mapsto Y$ is continuous at $x_0\in X$ then we have for all neighborhoods $V_{f(x_0)}$ of $f(x_0)$ that there exists an open neighborhood $U_{x_0}$ of $x_0$ with $x_0\in U_{x_0}\subset f^{-1}(V_{f(x_0)})$ but as $U_{x_0}$ is open then it is the union of bases elements hence there exists a basis element $U_{b}$ with $x_0\in U_{b}\subset f^{-1}(V_{f(x_0)})$ as $V_{f(x_0)}$ is an arbitrary neighborhood of $f(x_0)$ then this implies that the above is true for all neighborhoods $V_{b}\in \mathcal B_Y$ of $f(x_0)$ which completes the forward direction. 

    Assume that for $f:X\mapsto Y$ we have for each $x_0\in X$ that for each $W\in \mathcal B_Y$ that contains $f(x_0)$, there is a $\widetilde{W}\in \mathcal B_X$ with $x_0\in \widetilde{W} \subset f^{-1}(W)$. Now for any $x_0\in X$ consider an arbitrary neighborhood $V$ of $f(x_0)$. Then we have $V=\bigcup U_\alpha $ where $U_\alpha \in \mathcal B_Y $ which implies for some $U_\alpha ^\prime \in \mathcal B_y$ that $f(x_0)\in U_{\alpha}^\prime\subset V$ this implies (by the assumption) that there exists $\widetilde{W} \in \mathcal B_X$ with $x_0\in \widetilde{W}\subset  f^{-1}(U_{\alpha}^\prime)\subset f^{-1}(\bigcup U_\alpha)=f^{-1}(V)$ which shows that $f$ is continuous at $x_0\in X$.

\end{proof}


\vspace{.15in}

\noindent
\subsection*{2.}  When $X$ is a metric space, we know that the set of open balls, $\{ B(x,r) : x\in X \ \mbox{and} \ r>0\}$ is a basis for $X$'s metric topology $\mathcal T$. 


\vspace{.1in}
{\bfseries a.} Show that $\{ B(x,r) : x\in X \ \mbox{and} \ 0<r<1\}$ is a basis for a topology on $X$. Call this topology $\mathcal T _1$. 

\begin{proof}
    Let $x\in X$ then we have $x\in B(x,0.5)$ hence we have the first condition for being a basis satisfied.  Now suppose that we have $x\in B(y_1,r_1)\cap B(y_2,r_2)$ where $y_1,y_2\in X$ and $0<r_1<1$ and $0<r_2<1$. Then let $r_3=\min(r_1-d(x,y_1),r_2-d(x,y_2))$ then consider the set $B(x,r_3)$. Then for any $x_0\in B(x,r_3)$ we have $d(x_0,y_1)\leq d(x_0,x)+d(x,y_1)< r_3+d(x,y_1)\leq r_1-d(x,y_1)+d(x,y_1)\leq r_1$ note that the strict inequality follows from $d(x_0,x)<r_3$ and I used the triangle inequality for the first inequality and the third inequality follows due to the choice of $r_3$. Then for any $x_0\in B(x,r_3)$ we have $d(x_0,y_2)\leq d(x_0,y_2)+d(x,y_2)<r_3+d(x,y_2)\leq r_2-d(x,y_2)+d(x,y_2)=r_2$ which shows $x_0\in B(y_2,r_2)$ hence we get $B(x,r_3)\subset B(y_1,r_1)$ and $B(x,r_3)\subset B(y_2,r_2)$ which implies $x\in B(x,r_3)\subset B(y_1,r_1)\cap B(y_2,r_2)$. Which shows that it is a basis. 
\end{proof}


\vspace{.1in}
{\bfseries b.} Show that $\mathcal T = \mathcal T _1$. 
\begin{proof}
    Suppose $B(x_0,r_0)\in \{ B(x,r) : x\in X \ \mbox{and} \ 0<r<1\}$ then as $x\in X$ and $0<r_0$ we get $B(x_0,r_0)\in \{ B(x,r) : x\in X \ \mbox{and} \ r>0\}$ hence $\{ B(x,r) : x\in X \ \mbox{and} \ 0<r<1\}\subset \{ B(x,r) : x\in X \ \mbox{and} \ r>0\}$ which implies $\mathcal T_1\subset \mathcal T$. 

    Now using Munkres Lemma 13.3 for an arbitrary $x\in X$ and an arbitrary $B(x_0,r_0)\in \{B(x,r): x\in X \text{ and } 0<r<1\}$ with $x\in B(x_0,r_0)$ then we have $B(x_0,r_0)\in \{B(x,r): x \in X \text{ and } 0<r\}$ then as $x\in B(x_0,r_0)\subset B(x_0,r_0)$ we get $\mathcal T\subset \mathcal T_1$ as we have double inclusion we get $\mathcal T_1 =\mathcal T$ 
\end{proof}

\vspace{.15in}

\noindent
\subsection*{3.} Let $X$ be $\{0\} \bigcup \{ 1/k : k\in \mathbb N \}$ be ordered by the usual ``less than" $<$, i.e. the order it gets from ``less than" when $X$ is regarded as a subset of $[0,1]$. Give $X$ the associated order topology. In the associated product topology on $X\times X$, show that every open set that contains $(0,1)$ contains infinitely many points with second coordinate $1$ and that there is an open set that contains $(0,1)$ and in which every point has second coordinate $1$. 

\begin{proof}
    Let $X= \{0\}\bigcup \{1/k:k\in \mathbb{N}\}$ with the order topology. Now let $A\times B$ be an arbitrary open set in the product topology with $(0,1)\in A\times B$. Then as this is the product topology we have $A,B$ are open sets of $X$ with $0\in A$ and $1\in B$. Then we have two basis elements of the form $[0,b)\subset A$ where $b=1/n$ for some $n\in \mathbb{N}$ and $(a,1]$ where $a=1/k$ for some $k\in \mathbb{K}$. Then we have $(1/j,1)\in [0,b)\times (a,1]\subset A\times B$ where $j\in \mathbb{N}$ and $j\geq n$ this shows that there is an infinite number of points where the second coordinate is $1$. 
    
    We have $(1/2,1]=\{1\}$ and $[0,1/2)$ are open in $X$ and as this is the product topology we have that $[0,1/2)\times \{1\}$ is open in $X\times X$ but $[0,1/2)\times \{1\}=\{(a,1):a=0\text{ or } a=1/n\text{ where }n>2\}$ hence every element has 2nd coordinate $1$.
\end{proof}

If we start with the same order on $X$ and give $X\times X$ the associated dictionary order, then in the topology $X\times X$ gets from the dictionary order, show that, for every $y\in X$, every open set that contains $(0,1)$ contains infinitely many points with second coordinate $y$.

\begin{proof}
    Assume that $X\times X$ has the dictionary order and the order topology. Let $A\times B$ be an open set with $(0,1)\in A\times B$. Then we have a basis element of the form $\big((a,b),(c,d)\big)=\{(x,z)\in X\times X: (a,b)< (x,z)\text{ and }(x,z)<(c,d)\}$ where $a,b,c,d\in X$ with $(0,1)\in \big((a,b),(c,d)\big)\subset A\times B$. Then as $(0,1)\in \big((a,b),(c,d)\big)$ we have $(a,b)<(0,1)<(c,d)$. This implies that $a=0$ and $b<1$ as this is the order topology we get that $0<c$ as if it where not then we would have $c=0$ and $d=1$ which would imply that the basis is $\big((0,1),(0,1)\big)$ which is not an element of the order topology. Hence we get the strict inequality $0<c$ then we have for all $n\in \mathbb{N}$ with $1/n<c$ that for all $y\in X$ the inequality $(a,b)<(1/n,y)<(c,d)$ as there is an infinite number of natural numbers $k\in \mathbb{N}$ such that $0=a<1/k<c$ this completes the proof.



\end{proof}

 
\end{document}