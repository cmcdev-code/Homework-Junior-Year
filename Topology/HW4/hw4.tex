\documentclass{amsart}
\usepackage{amsmath, amssymb, amscd}

\setlength{\textwidth}{6.5in}
\setlength{\textheight}{9in}
\setlength{\topmargin}{-.25in}
\setlength{\evensidemargin}{0in}
\setlength{\oddsidemargin}{0in}

\theoremstyle{plain}
\newtheorem{theorem}{Theorem}[section]
\newtheorem{proposition}[theorem]{Proposition}
\newtheorem{lemma}[theorem]{Lemma}
\newtheorem{corollary}[theorem]{Corollary}

\theoremstyle{definition}
\newtheorem{definition}[theorem]{Definition}
\newtheorem{assumption}[theorem]{Assumption}

\theoremstyle{remark}
\newtheorem{remark}[theorem]{Remark}
\newtheorem{example}[theorem]{Example}
\newtheorem{notation}[theorem]{Notation}

\begin{document}

\section*{Math 4324\  Closed sets } 

\subsection*{Hand in Friday, February 16.} 




\vspace{.15in}
\noindent
\subsection*{1.} In $\mathbb R$ let $A = \{ 1/k : k\in \mathbb N \}$.

\vspace{.1in}
{\bfseries a.} If $\mathbb R$ has the standard topology, what is the boundary of $A$, and what is the set of limit points of $A$. Show that your answers are correct. 
    
The boundary points of $A$ are the set $A\cup \{0\}$. This is shown as for all points of $A$ are in $\partial A$ as for any $1/n\in A$ any neighborhood $O$ of $1/n$ we have $1/n\in \mathbb{R}\cap O$. Now as $O$ is open then there exists a basis element $B$ such that $x\in B$ and as this is the standard topology we have $B=(a,b)$ for some $a,b\in \mathbb{R}$ with $a<x<b$. We have  $(a+x)/2 \in O \cap A^c $ therefore $A\subset \partial A$. For any neighborhood $O$ containing $0$ that again as $O$ is open there exists a basis element $(a,b)$ with $a,b\in \mathbb{R}$ and $a<0<b$ and for some $n\in \mathbb{N}$ we have $1/n<b$ hence $1/n\in O\cap A\not = \emptyset$ additionally we have for some $c\in \mathbb{R}$ where $c\not \in A$ with $0<c<b$ that $c\in O\cap A^c\not = \emptyset$. Now assume $x\in \mathbb{R}$ with $x\not \in A\cup \{0\}$. Then if $x<0$ we immediately have an neighborhood $(x-1,0)$ with $(x-1,0)\cap A=\emptyset$ like wise if $x>0$ then we have the neighborhood $(1,x+1) $ with $(1,x+1)\cap A=\emptyset$. If $0<x<1$ we have for some $n\in \mathbb{N}$ that $1/(n+1)<x<1/n$ we have the neighborhood of x $(1/(n+1),1/n)$ with $A\cap (1/(n+1),1/n)=\emptyset$. This shows that $A\cup \{0\}=\partial A$.


We have that the limit point of $A$ is $\{0\}$. This is shown as let $O$ be an neighborhood of $0$
. Then as it is open we have a basis element $ (a,b)\subset O$ where $a,b\in \mathbb{R}$ with $a<0<b$. We have that there exists $n\in \mathbb{N}$ such that $0<1/n<b$ therefore $1/n\in O$ which shows that $0$ is a limit point. We have no element of $A$ is a limit point as for all $1/n\in A$ we have a neighborhood $(a,b)$ with $a,b\in \mathbb{R}$ where $1/n \in (a,b)$ but $1/(n+1),1/(n-1)\not \in (a,b)$ with the condition $n>1$. This shows that no element of $A$ is a limit point. Lastly assume $x\in \mathbb{R}\setminus(\{0\}\cup A)$ then if $x<0$ we have the neighborhood $(x-1,0)$ with $(x-1,0)\cap A=\emptyset$ and for $x>0$ we have the neighborhood $(1,x+1)$ with $(1,x+1)\cap A=\emptyset$. If $0<x<1$ then for some $n\in \mathbb{N}$ we have $(1/(n+1),1/n)$ with $x\in (1/(n+1),1/n)$ but $(1/(n+1),1/n)\cap A= \emptyset$. This implies that the only limit point of $A$ is $\{0\}$. 

\vspace{.1in}
{\bfseries b.} If $\mathbb R$ has the upper limit topology, what is the boundary of $A$, and what is the set of limit points of $A$. Show that your answers are correct. 

The boundary for $\mathbb R$ is $A$.


Let $1/n\in A$ then for any neighborhood $O$ of $1/n$ we have $1/n\in O \cap A$ therefore $O\cap A\not = \emptyset$. Now as $O$ is open we have a basis element with $(a,1/n]\subset O$ with $a\in \mathbb{R}$ and $a<1/n$. Then consider the intersection $O \cap A^c$ as $(a+\frac{1}{n})/2 \in (a,1/n)\subset O$ and $(a+\frac{1}{n})/2\in A^c$ that $O\cap A^c \not = \emptyset$. This implies that $A\subset \partial A$. Now let $x\in \mathbb{R}$ if $x\leq 0$ then we have the neighborhood $(x-1,0]$ and $(x-1,0]\cap A=\emptyset$. If $x>1$ then we have the neighborhood $(1,x+1]$ with $(1,x+1]\cap A=\emptyset$. If $0<x<1$ then we have the neighborhood $(a,x]$ where $a\in \mathbb R$ where $(a,x]\cap A=\emptyset$ this $a$ exists as for some $n\in \mathbb{N}$ we have $1/(n+1)<a<x<1/n$. 
This shows $A=\partial A$. 

We have that there are no limit points of $A$. 

Let $1/n\in A$ then we have the neighborhood $(1/(n+1),1/n]$ then as $(1/(n+1),1/n)\cap A=\{1/n\}$ we have no element of $A$ is a limit point. Now suppose $x\in \mathbb{R}$ if $x\leq 0$ then we have the neighborhood $(x-1,0]$ with $(x-1,0]\cap A=\emptyset$ if $x>1$ then $(1,x]$ is a neighborhood of $x$ and $(1,x]\cap A=\emptyset$. If $0<x<1$ then we have the neighborhood $(a,x]$ where $a\in \mathbb{R}$ and $(a,x]\cap A=\emptyset$ such an $a$ exists as for some $n\in \mathbb{N}$ the inequality is true $1/(n+1)<a<x<1/n$. This implies that there are no limit points of $A$ with the upper limit topology.


\vspace{.15in}

\noindent
\subsection*{2.} Let $A$ and $B$ be subsets of a topological space $X$. 

\vspace{.1in}
{\bfseries a.} If $A\subset B$, show that $\overline{A} \subset \overline{B}$. 

\begin{proof}
    Assume $A$ and $B$ are subsets of a topological space $X$ with $A\subset B$. Then we have $\bar A=A\cup A^\prime\subset B\cup A^\prime$. Let $x\in A^\prime$ then for all neighborhoods $O$ of $x$ we have $O\cap A\setminus 
    \{x\}\not = \emptyset$ as $A\subset B$ then $O\cap B\setminus \{x\}\not = \emptyset$ hence $x\in B^\prime$. This implies $A^\prime\subset B^\prime$ so we have $\bar A=A\cup A^\prime \subset B\cup B^\prime = \bar B$.
\end{proof}

\vspace{.1in}
{\bfseries b.} Show that $\overline{A\bigcup B} = \overline{A} \bigcup \overline{B}$. 

\begin{proof}
    
\end{proof}


\vspace{.15in}

\noindent
\subsection*{3.} 

\vspace{.1in}
{\bfseries a.} Give an example of a topological space $X$ and subsets $A$ and $B$ for which $\overline{A\bigcap B} \ne \overline{A} \bigcap \overline{B}$. Show that your example has the asserted property. 

\vspace{.1in}
{\bfseries b.} Give an example of a topological space $X$ and a collection of subsets $\{ A_j : j\in \mathbb N \}$ for which 
\[
\bigcup _{j\in \mathbb N } \overline{A_j}
\]
is not equal to 
\[
\overline{\bigcup _{j\in \mathbb N } A_j}.
\]
Show that your example has the asserted property. 

\vspace{.15in}

\noindent
\subsection*{4.}  Show that if $X$ is a Hausdorff topological space and $Y$ is a finite subset of $X$, then $Y$ has no limit points. 





\vspace{.15in}

\noindent
\subsection*{5.} 

{\bfseries a.} If $X$ is an infinite set with the finite complement topology and if $Y$ is a finite subset of $X$, what are the limit points of $Y$? Show that your answer is correct. 


\vspace{.1in}
{\bfseries b.} If $X$ is an infinite set with the finite complement topology and if $Y$ is an infinite subset of $X$, what are the limit points of $Y$? Show that your answer is correct. 




\vspace{.15in}

\noindent
\subsection*{6.} 

{\bfseries a.} Let $X$ be the product of countably many copies of $\mathbb R$, i.e. $X = \prod _{j\in \mathbb N } X_j$, where each $X_j$ is $\mathbb R$ with the standard topology. Let $Y$ be the subset of $X$ of elements $\vec{x} = ( x_1 , x_2 , . . . )$ for which at most finitely many of the entries $x_j$ are nonzero. If we give $X$ the box topology, what is the closure of $Y$? Show that your answer is correct. 

\vspace{.1in}

{\bfseries b.} Let $X$ be the product of countably many copies of $\mathbb R$, i.e. $X = \prod _{j\in \mathbb N } X_j$, where each $X_j$ is $\mathbb R$ with the standard topology. Let $Y$ be the subset of $X$ of elements $\vec{x} = ( x_1 , x_2 , . . . )$ for which at most finitely many of the entries $x_j$ are nonzero. If we give $X$ the product topology, what is the closure of $Y$? Show that your answer is correct. 

\vspace{.45in}
On Friday, February 16, I will send each of you the first {\bfseries test}, as an email attachment. I will also post the test in the tests folder in the Files section of the course Canvas site. The test will have a first page that is a cover page and that reveals no information about the test content.   Once you look beyond the cover page of the file, you may work on the test only during the next two hours.  During this time you may not consult any person or other source.  You must email your completed test to me 
no later than 5:00 pm on Friday, February 23.  There will be no class 
on Friday, February 23.  

Unless you're unusually good at typing and TeXing, I suggest that you handwrite your answers. You may use time beyond the 2 hours to put your answers into a TeX file, but don't change your answers during that transcription. A pdf file made from a TeX file is easiest for me to read, but I will also accept a scanned pdf file of your handwritten answers or, if necessary, a photo of your handwritten answers.

The most likely way to approach the test is to study during most of 
the week and to open the file and do the test at some time late in the week.  If you want to have me available to answer questions while you are doing the test, contact me ahead of time to ask whether I will be available when you plan to work on the test. 

It is an honor issue that anyone who has seen the test must not risk 
communicating any information about the test to anyone who has not 
finished the test.  This covers conversations about topology   
(including conversations that are overheard by others not in the 
conversation), leaving test scratchwork where it can be seen by others, etc.  If you have 
any questions about the honor expectations, please ask me before 
engaging in any questionable behavior.  

People who have not started the test may engage in all the usual 
preparations for the test, including consulting books, notes, and 
other sources, and discussing the material with classmates, as long as 
all such discussions are not overheard by anyone currently taking the 
test.

The test will cover all the material we have studied through the 
material appearing on the assignment due February 16.  Thus the test will cover material that has appeared on all assignments through this assignment. The sections of Munkres that cover this material are the sections on topological spaces, basis for a topology, metric topology, product topology, and closed sets and limit points. 


 
\end{document}