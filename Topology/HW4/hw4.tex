\documentclass{amsart}
\usepackage{amsmath, amssymb, amscd}

\setlength{\textwidth}{6.5in}
\setlength{\textheight}{9in}
\setlength{\topmargin}{-.25in}
\setlength{\evensidemargin}{0in}
\setlength{\oddsidemargin}{0in}

\theoremstyle{plain}
\newtheorem{theorem}{Theorem}[section]
\newtheorem{proposition}[theorem]{Proposition}
\newtheorem{lemma}[theorem]{Lemma}
\newtheorem{corollary}[theorem]{Corollary}

\theoremstyle{definition}
\newtheorem{definition}[theorem]{Definition}
\newtheorem{assumption}[theorem]{Assumption}

\theoremstyle{remark}
\newtheorem{remark}[theorem]{Remark}
\newtheorem{example}[theorem]{Example}
\newtheorem{notation}[theorem]{Notation}

\begin{document}

\section*{Math 4324\  Closed sets } 

\subsection*{Hand in Friday, February 16.} 




\vspace{.15in}
\noindent
\subsection*{1.} In $\mathbb R$ let $A = \{ 1/k : k\in \mathbb N \}$.

\vspace{.1in}
{\bfseries a.} If $\mathbb R$ has the standard topology, what is the boundary of $A$, and what is the set of limit points of $A$. Show that your answers are correct. 
    
The boundary points of $A$ are the set $A\cup \{0\}$. This is shown as for all points of $A$ are in $\partial A$ as for any $1/n\in A$ any neighborhood $O$ of $1/n$ we have $1/n\in \mathbb{R}\cap O$. Now as $O$ is open then there exists a basis element $B$ such that $x\in B$ and as this is the standard topology we have $B=(a,b)$ for some $a,b\in \mathbb{R}$ with $a<x<b$. We have  $(a+x)/2 \in O \cap A^c $ therefore $A\subset \partial A$. For any neighborhood $O$ containing $0$ that again as $O$ is open there exists a basis element $(a,b)$ with $a,b\in \mathbb{R}$ and $a<0<b$ and for some $n\in \mathbb{N}$ we have $1/n<b$ hence $1/n\in O\cap A\not = \emptyset$ additionally we have for some $c\in \mathbb{R}$ where $c\not \in A$ with $0<c<b$ that $c\in O\cap A^c\not = \emptyset$. Now assume $x\in \mathbb{R}$ with $x\not \in A\cup \{0\}$. Then if $x<0$ we immediately have an neighborhood $(x-1,0)$ with $(x-1,0)\cap A=\emptyset$ like wise if $x>0$ then we have the neighborhood $(1,x+1) $ with $(1,x+1)\cap A=\emptyset$. If $0<x<1$ we have for some $n\in \mathbb{N}$ that $1/(n+1)<x<1/n$ we have the neighborhood of x $(1/(n+1),1/n)$ with $A\cap (1/(n+1),1/n)=\emptyset$. This shows that $A\cup \{0\}=\partial A$.


We have that the limit point of $A$ is $\{0\}$. This is shown as let $O$ be an neighborhood of $0$
. Then as it is open we have a basis element $ (a,b)\subset O$ where $a,b\in \mathbb{R}$ with $a<0<b$. We have that there exists $n\in \mathbb{N}$ such that $0<1/n<b$ therefore $1/n\in O$ which shows that $0$ is a limit point. We have no element of $A$ is a limit point as for all $1/n\in A$ we have a neighborhood $(a,b)$ with $a,b\in \mathbb{R}$ where $1/n \in (a,b)$ but $1/(n+1),1/(n-1)\not \in (a,b)$ with the condition $n>1$. This shows that no element of $A$ is a limit point. Lastly assume $x\in \mathbb{R}\setminus(\{0\}\cup A)$ then if $x<0$ we have the neighborhood $(x-1,0)$ with $(x-1,0)\cap A=\emptyset$ and for $x>0$ we have the neighborhood $(1,x+1)$ with $(1,x+1)\cap A=\emptyset$. If $0<x<1$ then for some $n\in \mathbb{N}$ we have $(1/(n+1),1/n)$ with $x\in (1/(n+1),1/n)$ but $(1/(n+1),1/n)\cap A= \emptyset$. This implies that the only limit point of $A$ is $\{0\}$. 

\vspace{.1in}
{\bfseries b.} If $\mathbb R$ has the upper limit topology, what is the boundary of $A$, and what is the set of limit points of $A$. Show that your answers are correct. 

The boundary for $\mathbb R$ is $A$.


Let $1/n\in A$ then for any neighborhood $O$ of $1/n$ we have $1/n\in O \cap A$ therefore $O\cap A\not = \emptyset$. Now as $O$ is open we have a basis element with $(a,1/n]\subset O$ with $a\in \mathbb{R}$ and $a<1/n$. Then consider the intersection $O \cap A^c$ as $(a+\frac{1}{n})/2 \in (a,1/n)\subset O$ and $(a+\frac{1}{n})/2\in A^c$ that $O\cap A^c \not = \emptyset$. This implies that $A\subset \partial A$. Now let $x\in \mathbb{R}$ if $x\leq 0$ then we have the neighborhood $(x-1,0]$ and $(x-1,0]\cap A=\emptyset$. If $x>1$ then we have the neighborhood $(1,x+1]$ with $(1,x+1]\cap A=\emptyset$. If $0<x<1$ then we have the neighborhood $(a,x]$ where $a\in \mathbb R$ where $(a,x]\cap A=\emptyset$ this $a$ exists as for some $n\in \mathbb{N}$ we have $1/(n+1)<a<x<1/n$. 
This shows $A=\partial A$. 

We have that there are no limit points of $A$. 

Let $1/n\in A$ then we have the neighborhood $(1/(n+1),1/n]$ then as $(1/(n+1),1/n)\cap A=\{1/n\}$ we have no element of $A$ is a limit point. Now suppose $x\in \mathbb{R}$ if $x\leq 0$ then we have the neighborhood $(x-1,0]$ with $(x-1,0]\cap A=\emptyset$ if $x>1$ then $(1,x]$ is a neighborhood of $x$ and $(1,x]\cap A=\emptyset$. If $0<x<1$ then we have the neighborhood $(a,x]$ where $a\in \mathbb{R}$ and $(a,x]\cap A=\emptyset$ such an $a$ exists as for some $n\in \mathbb{N}$ the inequality is true $1/(n+1)<a<x<1/n$. This implies that there are no limit points of $A$ with the upper limit topology.


\vspace{.15in}

\noindent
\subsection*{2.} Let $A$ and $B$ be subsets of a topological space $X$. 

\vspace{.1in}
{\bfseries a.} If $A\subset B$, show that $\overline{A} \subset \overline{B}$. 

\begin{proof}
    Assume $A$ and $B$ are subsets of a topological space $X$ with $A\subset B$. Then we have $\bar A=A\cup A^\prime\subset B\cup A^\prime$. Let $x\in A^\prime$ then for all neighborhoods $O$ of $x$ we have $O\cap A\setminus 
    \{x\}\not = \emptyset$ as $A\subset B$ then $O\cap B\setminus \{x\}\not = \emptyset$ hence $x\in B^\prime$. This implies $A^\prime\subset B^\prime$ so we have $\bar A=A\cup A^\prime \subset B\cup B^\prime = \bar B$.
\end{proof}

\vspace{.1in}
{\bfseries b.} Show that $\overline{A\bigcup B} = \overline{A} \bigcup \overline{B}$. 

\begin{proof}
    As $\overline{ A\bigcup B}= A\cup B \cup (A\cup B)^\prime$ we just need to show $(A\cup B)^\prime = A^\prime \cup B^\prime$. Let $x\in (A\cup B)^\prime$ then for all neighborhoods $O$ of $x$ we have $O\cap (A\cup B)\setminus \{x\}\not = \emptyset$. Then \begin{equation}O\cap (A\cup B)\setminus \{x\}=(O\cap A)\cap \{x\}^c\cup (O\cap B)\cap \{x\}^c = (O\cap A\setminus \{x\})\cup (O\cap B\setminus \{x\})\end{equation} this shows that $x\in A^\prime \cup B^\prime$ hence $(A\cup B)^\prime \subset A^\prime \cup B^\prime $. Then for the other inclusion the equation $(1)$ also holds hence $A^\prime \cup B^\prime \subset (A\cup B)^\prime$ which implies $(A\cup B)^\prime=A^\prime \cup B^\prime$. Then we have $\overline{ A\bigcup B}=(A\cup B) \cup (A\cup B)^\prime=A \cup B \cup A^\prime \cup B^\prime= (A\cup A^\prime) \cup (B\cup B^\prime)=\overline{A} \bigcup \overline B$
\end{proof}


\vspace{.15in}

\noindent
\subsection*{3.} 

\vspace{.1in}
{\bfseries a.} Give an example of a topological space $X$ and subsets $A$ and $B$ for which $\overline{A\bigcap B} \ne \overline{A} \bigcap \overline{B}$. Show that your example has the asserted property. 

Let $\mathbb{R}$ be a topological space with the standard topology. Let the two sets be the open interval $(0,1)$ and $(1,2)$ then we have $(0,1)^\prime=[0,1]$ as for any $x\in (0,1)$ any neighborhood $O$ of $x$ we have a basis element $(a,b)$ where $a,b\in \mathbb{R}$ with $x\in (a,b)\subset U$ and as $(a,b)\cap (0,1)\setminus \{x\}\not = \emptyset$ then we have $x$ is a limit point. Now to show that $0\in (0,1)^\prime$ we have for any neighborhood $O$ of $0$ that there exists a basis element $(a,b)$ with $a,b\in \mathbb R$ and $a<0<b$ such that $(a,b)\subset U$ then $(a,b)\cap (0,1)\setminus \{0\}\not = \emptyset$ as $b/2\in (0,1)$. To show $1\in (0,1)^\prime$ for any neighborhood $O$ of $1$ we have a element of the basis $(a,b)$ where $a,b\in \mathbb{R}$ with $a<1<b$ and $(a,b)\subset O$. Then we have $(a,b)\cap (0,1)\not = \emptyset$ as there exists a real number $c$ with $a<c<1$. This shows that $1\in (0,1)^\prime$. For any real number $r\in \mathbb{R}\setminus [0,1]$ if $r<0$ we have the neighborhood $(r-1,0)$ and $(r-1,0)\cap (0,1)= \emptyset$ which shows that it is not a limit point. If $r>1$ then we have the neighborhood $(1,r+1)$ and $(1,r+1)\cap (0,1)=\emptyset$. This shows that the limit points of $(0,1)$ are $[0,1]$. From the definition of closure we have $\overline {(0,1)}= (0,1)\cup [0,1]=[0,1]$. A similar argument gives that $(1,2)^\prime=[1,2]$. Now as $(0,1)\cap (1,2)=\emptyset$. We have $\overline{(0,1)\bigcap (1,2)}=\emptyset$ and $\overline{(0,1)}\cap \overline{(1,2)}={1}$ hence they are not equal. n



\vspace{.1in}
{\bfseries b.} Give an example of a topological space $X$ and a collection of subsets $\{ A_j : j\in \mathbb N \}$ for which 
\[
\bigcup _{j\in \mathbb N } \overline{A_j}
\]
is not equal to 
\[
\overline{\bigcup _{j\in \mathbb N } A_j}.
\]
Show that your example has the asserted property. 

Claim that $A_j=\{1/j\}$ has this property in the Real numbers with the standard topology.

\begin{proof}
    From problem (1a) we have that the limit point of $\bigcup_{j\in \mathbb{N}}A_j$ is ${0}$. By the definition of complement $\overline {\bigcup_{j\in \mathbb{N}}A_j}=(\bigcup_{j\in \mathbb{N}}A_j) \cup (\bigcup_{j\in \mathbb{N}}A_j)^\prime=\{1/n:n\in \mathbb{N}\}\cup \{0\}$. But for each $\overline {A_j}=A_j\cup A_j^\prime$ we have $A^\prime=\emptyset$ which follows from (1a) therefore we have $\overline{A_j}=A_j$ which implies $\bigcup_{j\in \mathbb{N}}\overline A_j=\{1/n:n\in \mathbb{N}\}$ but this set does not contain $0$ hence $\overline {\bigcup_{j\in \mathbb{N}}A_j}\not = \bigcup_{j\in \mathbb{N}}\overline{A_j}$. 
\end{proof}



\vspace{.15in}

\noindent
\subsection*{4.}  Show that if $X$ is a Hausdorff topological space and $Y$ is a finite subset of $X$, then $Y$ has no limit points. 





\vspace{.15in}

\noindent
\subsection*{5.} 

{\bfseries a.} If $X$ is an infinite set with the finite complement topology and if $Y$ is a finite subset of $X$, what are the limit points of $Y$? Show that your answer is correct. 

Claim that the limit points of $Y$ is the emptyset. 
\begin{proof}
    Assume that $X$ is an infinite set with the finite complement topology and $Y$ is a finite subset of $X$. Now let $x\in X$ then we have that $(X\setminus Y)\cup \{x\}$ is a neighborhood as it it's complement is finite and contains $x$ but as $((X\setminus Y)\cup \{x\})\cap Y=\{x\}$ we have $x$ is not a limit point. As it was chosen arbitrarily we have $Y^\prime=\emptyset$  
\end{proof}


\vspace{.1in}
{\bfseries b.} If $X$ is an infinite set with the finite complement topology and if $Y$ is an infinite subset of $X$, what are the limit points of $Y$? Show that your answer is correct. 


Claim it is all of $X$ 

\begin{proof}
    Let $X$,$Y$ be as definined. Then let $x\in X$ we have that any neighborhood $O$ of $x$ contains an infinite number of elements of $Y$ as if it only contained a finite number of elements of $Y$ then $X\setminus O$ would not be countable because it would at least contain an infinite number of elements of $Y$. Therefore we have $O\cap Y\not = \emptyset$. Hence $x$ is a limit point and as $x$ was arbitrarily chosen from $X$ we have that $Y^\prime=X$.
\end{proof}

\vspace{.15in}

\noindent
\subsection*{6.} 

{\bfseries a.} Let $X$ be the product of countably many copies of $\mathbb R$, i.e. $X = \prod _{j\in \mathbb N } X_j$, where each $X_j$ is $\mathbb R$ with the standard topology. Let $Y$ be the subset of $X$ of elements $\vec{x} = ( x_1 , x_2 , . . . )$ for which at most finitely many of the entries $x_j$ are nonzero. If we give $X$ the box topology, what is the closure of $Y$? Show that your answer is correct. 

Claim is that $\bar Y=Y$

\begin{proof}
    Let $X,Y$ be as definition. From the definition of closure if we prove that $Y^\prime \subset Y$ then that proves the claim. Let  $\vec x\in X\setminus Y$ then consider the neighborhood $O=\prod_{j\in \mathbb{N}}u_j$ where $u_j=\begin{cases}
        (0,\pi_j(x)+1)& \text{ if } \pi_j(x)\not =0\\
        (-1,1) & \text{ if } \pi_j(x) = 0
    \end{cases}$
    we have $O$ does not contain zero an infinite number of times based on $x\not \in Y$. But as each element of $Y$ contains zero an infinite number of times we have $U\cap Y=\emptyset$. This implies that $Y^\prime \not = X\setminus Y$ which implies that $Y^\prime\subset Y$ so $\bar Y=Y\cup Y^\prime = Y$ which completes the proof. 



\end{proof}


\vspace{.1in}

{\bfseries b.} Let $X$ be the product of countably many copies of $\mathbb R$, i.e. $X = \prod _{j\in \mathbb N } X_j$, where each $X_j$ is $\mathbb R$ with the standard topology. Let $Y$ be the subset of $X$ of elements $\vec{x} = ( x_1 , x_2 , . . . )$ for which at most finitely many of the entries $x_j$ are nonzero. If we give $X$ the product topology, what is the closure of $Y$? Show that your answer is correct. 


 
\end{document}