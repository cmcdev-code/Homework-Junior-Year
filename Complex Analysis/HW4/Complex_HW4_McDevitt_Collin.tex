%===========================================================
% do not change this formatting please :-)
%===========================================================
\documentclass[letter,12pt,reqno]{article}
\usepackage[margin=1in]{geometry}
\usepackage{amssymb,amsmath,amsthm,bm,mathrsfs,colortbl,fancyhdr,tcolorbox,enumitem}
\def\honorcode{\textit{In accordance with the Hokie Honor Code, I affirm that I have neither given nor received unauthorized assistance on this assignment.}}
\fancyhead{}
\fancyhead[L]{Collin McDevitt} %Replace "NAME: " WITH YOUR NAME
\fancyhead[C]{MATH 5235 HOMEWORK 03}
\fancyhead[R]{PAGE \thepage}
\fancyfoot{}
\renewcommand{\footrulewidth}{0.4pt}
\fancyfoot[C]{\honorcode}
\date{\today}
%===========================================================
% convenient commands -- feel free to add more as you see fit
%===========================================================
\newcommand{\C}{\mathbb{C}}
\newcommand{\K}{\mathbb{K}}
\newcommand{\Poly}{\mathcal{P}}
\newcommand{\Q}{\mathbb{Q}}
\newcommand{\R}{\mathbb{R}}
\newcommand{\dotp}{\boldsymbol{\cdot}}
\newcommand{\Span}{\operatorname{Span}}

\begin{document}
\pagestyle{fancy}
%===========================================================
%===========================================================


%===========================================================
% PROBLEM 1
%===========================================================
\begin{tcolorbox}
  \textbf{Problem 1.}
    Compute the fractional linear transformation determined by the correspondence:
    \[
        (0,1,\infty)\mapsto (1,1+i,2)
    \]
  
\end{tcolorbox}

\[f(z)=\frac{z(2-2i)-2}{z(1-i)-2}\]

Then $f(0)=1$, $f(1)=\frac{2i}{1+i}=\frac{2i(1-i)}{|1+i|}=i(1-i)=i+1$, and $f(\infty)=2$.

\begin{tcolorbox}
    \textbf{Problem 2.}
    Show that the differential \[\frac{-ydx+xdy}{x^2+y^2},\;\;\;(x,y)\not = (0,0)\]
    is closed. Show that it is not independent of path on any annulus centered at $0$.
\end{tcolorbox}

First to show that the differential is closed. 

Computing $\frac{\partial P}{\partial y}$ we get \[\frac{\partial \left(\frac{-y}{x^2+y^2}\right)}{\partial y}=\frac{2y^2}{(x^2+y^2)}+\frac{-1}{x^2+y^2}=\frac{-x^2+y^2}{(x^2+y^2)^2}\]

Computing $\frac{\partial Q}{\partial x}$

\[
\frac{\partial \left(\frac{x}{x^2+y^2}\right)}{\partial x}=\frac{-2x^2}{(x^2+y^2)^2}+\frac{1}{x^2+y^2}=\frac{-x^2+y^2}{(x^2+y^2)^2}
\]
As $\frac{\partial P}{\partial y}=\frac{\partial Q}{\partial x}$ we have that the differential is closed.

Now to show that it is not independent of path on any annulus centered at $0$. 

Let $r>0$
\[
    \oint_{|z|=r}  \frac{-ydx+xdy}{x^2+y^2}
\]
Using the parametrization $\gamma(t)=(r\cos t,r \sin t)$ for $t\in [0,2\pi)$

\begin{align*}
    \oint_{|z|=r} \frac{-ydx+xdy}{x^2+y^2}&=\int_{0}^{2\pi}\frac{-r\sin t \cdot - r \sin t+r\cos t \cdot r c\cos t}{r^2\cos^2 t+r^2\sin^2t} dt\\
    &=\int_0^{2\pi}dt\\
    &= 2\pi
\end{align*}
Therefore as this is a closed curve it is not independent of path. 
\newpage
\begin{tcolorbox}
    \textbf{Problem 3.} Show that a complex valued function $h(z)$ on a simply connected domain is harmonic if and only if $h(z)=f(z)+\overline{g(z)}$, where $f(z),g(z)$ are analytic on $D$.
\end{tcolorbox}


\begin{proof}

    Assume that $h(z)$ is harmonic on a simply connected domain $D$. Then we have $h(z)=u(z)+iv(z)$ where $u(z),v(z)$ are both harmonic on $D$ as well. Then for $u(z)$ as this is a simply connected domain there exists a harmonic conjugate $\mu (z)$. Likewise for $v(z)$ there exists a harmonic conjugate $\phi(z)$. From this we get the two analytic equations $a(z)=u(z)+i\mu(z)$ and $b(z)=v(z)+i\phi(z)$. Solving for $u(z)$ and $v(z)$ we get $u(z)=\frac{a(z)+\overline{a(z)}}{2}$ and $v(z)=\frac{b(z)+\overline{b(z)}}{2}$. Then we have \begin{align*}
        h(z)&=u(z)+v(z)\\&=\frac{a(z)+\overline{a(z)}}{2}+i\frac{b(z)+\overline{b(z)}}{2}\\
        &=\frac{a(z)+ib(z)}{2}+\frac{\overline{ a(z)-ib (z)}}{2}
    \end{align*}
    Letting $f(z)=\frac{a(z)+ib(z)}{2}$ and $\overline{g(z)}=\frac{\overline{ a(z)-ib (z)}}{2}$. Both $f,g$ are analytic as $a,b$ are and the sum of two differentiable functions is differentiable with their derivatives still being continuous as well. This completes the forward direction.

    For the backwards direction assume that $h(z)$ is a complex valued function on the simply connected domain $D$ and $f(z),g(z)$ are analytic on $D$. With $h(z)=f(z)+\overline{g(z)}$

    Then we have $h=u+ iv$ with $u=\text{Re} f+\text{Re}g$ and $v=\text{Im}f-\text{Im}g$. Then as \[\frac{\partial^2  \text{Re} f+\text{Re}g}{\partial x^2}=\frac{\partial }{\partial x}\frac{\partial \text{Im}f +\text{Im}g}{\partial y}=\frac{\partial }{\partial y}\frac{\partial \text{Im}f +\text{Im}g}{\partial x}=\frac{\partial ^2 \text{Re}f+\text{Re}g}{\partial y^2}\]
    Which shows that $u$ is harmonic.

    Similarly with $v$.
    
    \[
        \frac{\partial ^2 \text{Im}f - \text{Im}g}{\partial x^2}=\frac{\partial}{\partial x} \frac{\partial \text{Re}f-\text{Re}g}{\partial y}=\frac{\partial}{\partial y} \frac{\partial \text{Re}f-\text{Re}g}{\partial x}=\frac{\partial \text{Im}f -\text{Im} g}{\partial y^2}
    \]
    Hence $v$ is harmonic as well which implies that $h=u+iv$  is harmonic. 


\end{proof}

\begin{tcolorbox}
    \textbf{Problem 4.} If $z_0\in D$ and $D_0$ is a disk centered at $z_0$ with area $A$ and contained in $D$, then $f(z_0)=\frac{1}{A}\int \int_{D_0}f(z)dxdy.$
\end{tcolorbox}

\begin{proof}
    Let $D_0$ have radius $\rho$. Then 
    \begin{align*}
    \frac{1}{A}\int_0^\rho\int_0^{2\pi}f(z_0+re^{i\theta})d\theta dr
    \end{align*}
    As $f(z)$ has the mean value property with respect to circles we get.
    \[
        \frac{1}{A}\int_0^\rho\int_0^{2\pi}f(z_0+re^{i\theta})rd\theta dr
        =\frac{2\pi}{A}f(z_0) \int_0^\rho r dr=\frac{\pi\rho^2}{A}f(z_0) =f(z_0)
    \]
    The last equality was satisfied due to $A=\rho^2 \pi$.
\end{proof}

\begin{tcolorbox}
    \textbf{Problem 4.}
    Use the maximum principle to prove the fundamental theorem of algebra, that any polynomial $p(z)$ of degree $n \geq 1$ has a zero, by applying the maximum principle to $1/p(z)$ on a disk of large radius.  
\end{tcolorbox}

\begin{proof}
    Assume that $p(z)$ is a polynomial of degree greater then $1$, and that $p(z)$ does not have any zeros. Then we have that the function $f(z)=\frac{1}{p(z)}$ is defined for all $z\in \mathbb{C}$. 
\end{proof}

\end{document}