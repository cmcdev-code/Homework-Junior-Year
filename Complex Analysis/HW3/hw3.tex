%===========================================================
% do not change this formatting please :-)
%===========================================================
\documentclass[letter,12pt,reqno]{article}
\usepackage[margin=1in]{geometry}
\usepackage{amssymb,amsmath,amsthm,bm,mathrsfs,colortbl,fancyhdr,tcolorbox,enumitem}
\def\honorcode{\textit{In accordance with the Hokie Honor Code, I affirm that I have neither given nor received unauthorized assistance on this assignment.}}
\fancyhead{}
\fancyhead[L]{Collin McDevitt} %Replace "NAME: " WITH YOUR NAME
\fancyhead[C]{MATH 5235 HOMEWORK 03}
\fancyhead[R]{PAGE \thepage}
\fancyfoot{}
\renewcommand{\footrulewidth}{0.4pt}
\fancyfoot[C]{\honorcode}
\date{\today}
%===========================================================
% convenient commands -- feel free to add more as you see fit
%===========================================================
\newcommand{\C}{\mathbb{C}}
\newcommand{\K}{\mathbb{K}}
\newcommand{\Poly}{\mathcal{P}}
\newcommand{\Q}{\mathbb{Q}}
\newcommand{\R}{\mathbb{R}}
\newcommand{\dotp}{\boldsymbol{\cdot}}
\newcommand{\Span}{\operatorname{Span}}

\begin{document}
\pagestyle{fancy}
%===========================================================
%===========================================================


%===========================================================
% PROBLEM 1
%===========================================================
\begin{tcolorbox}
  \textbf{Problem 1.}
  Show that if $f$ and $\bar f$ are analytic on a domain $D$, then $f$ is constant.
  \end{tcolorbox}
\vskip1em

\begin{proof}
    Assume that $f=u+iv$ and $\bar f= u- i v$ are both analytic on domain $D$. Then we have that both satisfy the Cauchy-Riemann equations: That is 
    \begin{align*}
        \frac{\partial u}{\partial x} &= \frac{\partial v}{\partial y}, & \frac{\partial u}{\partial y} &= -\frac{\partial v}{\partial x}, \\
        \frac{\partial u}{\partial x} &= -\frac{\partial v}{\partial y}, & \frac{\partial u}{\partial x} &= \frac{\partial v}{\partial x}.
    \end{align*}
    Then we have $\frac{\partial u}{\partial x}=- \frac{\partial u}{\partial x}\implies \frac{\partial u}{\partial x}=0$ using the same reasoning for the rest we get  $\frac{\partial v}{\partial x}=\frac{\partial u}{ \partial y}= \frac{\partial v}{\partial y}=0$. This implies that $u$ and $v$ are both constant functions therefore $f$ is constant.
    \end{proof}

    \begin{tcolorbox}
    \textbf{Problem 2.}
    Let $a$ be a complex number, $a \not = 0 $, and $f(z)$ be an analytic branch of $z^a$ on $\C \setminus (-\infty, 0]$. Show that $f^\prime(z)=a f(z)/z$. 
    \end{tcolorbox}

    \begin{proof}
        Let $f(z)=z^a$ where $a\neq 0$, and $f(z)$ be an analytic branch of $z^a$ on $\C \setminus (-\infty,0]$. Then we have $f(z)=e^{a \text{Log} z}$. Using the chain rule we get $f^\prime(z)=ae^{a\text{Log} z}\cdot \frac{1}{z}$. As $f(z)=e^{a\text{Log}}$ is the same branch we can do the substitution $f^\prime(z)=ae^{a\text{Log}z}/z=af(z)/z$ which completes the proof.
    \end{proof}

    \begin{tcolorbox}
    \textbf{Problem 3.}
    Show that if $h(z)$ is a complex valued harmonic function such that $zh(z)$ is also harmonic, then $h(z)$ is analytic.
    \end{tcolorbox}
    
    \begin{proof}
        Assume that $h(z)$ is a complex valued harmonic function such that $zh(z)$ is also harmonic. Then $h(z)=u(z)+iv(z)$ we get \begin{align*}
            \frac{\partial^2 u}{\partial x^2}+\frac{\partial^2 u}{\partial y^2 } & =0 \\ \frac{\partial^2 v}{\partial x^2}+\frac{\partial ^2 v}{\partial y^2} & =0
        \end{align*} 
        Additionally as we have $zh(z)$ is harmonic we get $$(x+iy)(u(z)+iv(z))=xu(z)-yv(z) +i(xv(z)+yu(z))$$
         
        We get \begin{align}
            \frac{\partial ^2(xu(z)-yv(z))}{\partial x^2}+\frac{\partial ^2 (xu(z)-yv(z))}{\partial y^2} &=0 \\ \frac{\partial ^2(xv(z)+yu(z))}{\partial x^2}+\frac{\partial ^2 (xv(z)+yu(z))}{\partial y^2} &=0
        \end{align}

        Applying the partial derivatives to (1) we get 
        \[
            x\frac{\partial ^2 u}{\partial x^2}+ \frac{\partial u}{\partial x}-y\frac{\partial^2 v}{\partial x^2}+ x\frac{\partial ^2 u}{\partial y^2}- y\frac{\partial ^2 v}{\partial y^2}- \frac{\partial v}{\partial y}=0
        \]
        \[
            x(\frac{\partial^2 u}{\partial x^2}+\frac{\partial ^2 u}{\partial y^2})-y(\frac{\partial ^2 v}{\partial x^2}+\frac{\partial ^2 v}{\partial y^2})+ \frac{\partial u}{\partial x}-\frac{\partial v}{\partial y}=0
        \]
        \[
            \frac{\partial u}{\partial x}=\frac{\partial v}{\partial y}
        \]
        
        Applying the partial derivatives to (2) we get.
        \[
            x\frac{\partial ^2 v}{\partial x^2}+ \frac{\partial v}{\partial x}+ y\frac{\partial ^2 u}{\partial x^2 } + x\frac{\partial ^2 v}{\partial y^2}+y\frac{\partial ^2u}{\partial y^2}+ \frac{\partial u}{\partial y}=0
        \] 
        \[
            x(\frac{\partial ^2 v}{\partial x^2}+\frac{\partial ^2 v}{\partial y^2})+y(\frac{\partial ^2 u}{\partial x^2}+\frac{\partial^2 u}{\partial y^2})+\frac{\partial v}{\partial x}+\frac{\partial u}{\partial y}=0
        \]
        \[
            \frac{\partial u}{\partial y}=-\frac{\partial v}{\partial x}
        \]
        Therefore we have that the Cauchy Riemann equations are satisfied which implies that $h(z)$ is analytic.
    \end{proof}
    \begin{tcolorbox}
    \textbf{Problem 4.}
    Let $f=u+iv$ be a continuously differentiable complex valued function on a domain $D$ such that the Jacobian matrix of $f$ does not vanish at any point of $D$. Show that if $f$ maps orthogonal curves to orthogonal curves, then either $f$ or $\bar f$ is analytic, with a nonvanishing derivative.
    \end{tcolorbox}
    \begin{proof}
        Assume that $f$ maps orthogonal curves to orthogonal We have that derivative of the tangent curves. Let $x_0+iy_0\in D$ and consider the four curves $c_1(t)=(x_0+t,y_0+t), c_2(t)=(x_0+t,y_0-t), c_3(t)=(x_0+t,y_0) ,c_4(t)=(x_0,y_0+t)$. We have that tangent vectors for $c_1$ and $c_2$ are orthogonal and the tangent vectors for $c_3$ and $c_4$ are orthogonal. Now for some curve $c(t)=(x(t),y(t))$ the following equation  $$\frac{ f( c(t))}{dt}=(\frac{\partial u}{\partial x}\frac{dx}{dt}+\frac{\partial u}{\partial y}\frac{dy}{dt},\;\;\frac{\partial v}{\partial x}\frac{dx}{dt}+\frac{\partial v}{\partial y}\frac{dy}{dt})$$
        which shows that the image of the tangent vector for a curve under a function is given by the Jacobian.

        Applying the equation for the curves $c_1,c_2$ we get 

        \begin{align*}
            f(c_1(t)) &= (\frac{\partial u}{\partial x}+\frac{\partial u}{\partial y}, \frac{\partial v}{\partial x}+\frac{\partial v}{\partial y}) \\
            f(c_2(t)) &= (\frac{\partial u}{\partial x}-\frac{\partial u}{\partial y}, \; \frac{\partial v}{\partial x}-\frac{\partial v}{\partial y} )
            \\
            f(c_3(t)) & = (\frac{\partial u}{\partial x},\frac{\partial v}{\partial x})
            \\
            f(c_4(t)) & = (\frac{\partial u}{\partial y},\frac{\partial v}{\partial y})
        \end{align*}

        As $f$ maps orthogonal curves to orthogonal curves we have get by the dot product of $f(c_1(t))$ and $f(c_2(t))$
        \begin{equation}(\frac{\partial u}{\partial x}+\frac{\partial u}{\partial y})(\frac{\partial u}{\partial x}-\frac{\partial u}{\partial y})+(\frac{\partial v}{\partial x}+\frac{\partial v}{\partial y})(\frac{\partial v}{\partial x}-\frac{\partial v}{\partial y})=0\end{equation}.

        By the dot product of $f(c_3(t))$ and $f(c_4(t))$ we get \begin{equation}
            \frac{\partial u}{\partial x}\frac{\partial u}{\partial y}+\frac{\partial v}{\partial x}\frac{\partial v}{\partial y}=0
        \end{equation}
        
        Expanding $(3)$ we get 
        \begin{align*}
        (\frac{\partial u}{\partial x})^2- (\frac{\partial u}{\partial y})^2 + (\frac{\partial v}{\partial x})^2-(\frac{\partial v}{\partial y})^2&=0\\
        \left(\frac{\partial u}{\partial x}\right)^2\left(        (\frac{\partial u}{\partial x})^2- (\frac{\partial u}{\partial y})^2 + (\frac{\partial v}{\partial x})^2-(\frac{\partial v}{\partial y})^2\right)&=0            
        \\
        (\frac{\partial u}{\partial x})^4- (\frac{\partial u}{\partial x})^2(\frac{\partial u}{\partial y})^2+(\frac{\partial u}{\partial x})^2(\frac{\partial v}{\partial x})^2-(\frac{\partial u}{\partial x})^2(\frac{\partial v}{\partial y})^2& =0
        \end{align*}

        Applying the substitution $(\frac{\partial u}{\partial x}\frac{\partial u}{\partial y})^2=(\frac{\partial v}{\partial x}\frac{\partial v}{\partial y})^2$ which comes from equation $(4)$.

        \begin{align*}
            (\frac{\partial u}{\partial x})^4- (\frac{\partial v}{\partial x})^2(\frac{\partial v}{\partial y})^2+(\frac{\partial u}{\partial x})^2(\frac{\partial v}{\partial x})^2-(\frac{\partial u}{\partial x})^2(\frac{\partial v}{\partial y})^2& =0\\
            (\frac{\partial u}{\partial x})^2\left((\frac{\partial u}{\partial x})^2 +(\frac{\partial v}{\partial x})^2\right)-(\frac{\partial v}{\partial y})^2\left((\frac{\partial u}{\partial x })^2+ (\frac{\partial v}{\partial x})^2 \right)&=0\\
            \left((\frac{\partial u}{\partial x})^2 -(\frac{\partial v}{\partial y})^2\right)\left((\frac{\partial u}{\partial x})^2 +(\frac{\partial v}{\partial x})^2\right) & = 0
        \end{align*}
        This implies either $\frac{\partial u}{\partial x}$
    

    \end{proof}

\end{document}