\documentclass[11pt,twoside]{article}

\usepackage{amssymb}
\usepackage{amsmath}
\usepackage{latexsym}
\usepackage{color}
\usepackage{graphics}
\usepackage{xspace}

% Commands for special characters
\newcommand{\mybackslash}{\char'134}
\newcommand{\myleftbracket}{\char'173}
\newcommand{\myrightbracket}{\char'175}
\newcommand{\mypercent}{\char'045}
\newcommand{\myunderscore}{\char'137}

% The 'ifthen' package supports Boolean flags
\usepackage{ifthen}
% The 'solutions' flag determines whether this is the original handout
%    or a solution
\newboolean{solutions}
\setboolean{solutions}{False}  % Default is original handout

% Uncomment the next line before starting on the solutions
\setboolean{solutions}{True}

\newcommand{\coursenumber}{CS 4124}
\newcommand{\coursetitle}{Theory of Computation}
\newcommand{\docdate}{February 15, 2024}
\newcommand{\duedate}{March 15, 2024}
\newcommand{\homeworknumber}{3}

% The document title depends on whether these are solutions
\ifthenelse{\boolean{solutions}}{% Then
\newcommand{\doctitle}{Solutions to Homework Assignment \homeworknumber}
}{% Else
\newcommand{\doctitle}{Homework Assignment \homeworknumber}
}

% The document date depends on whether these are solutions
\ifthenelse{\boolean{solutions}}{% Then
\renewcommand{\docdate}{\duedate}
}{% Else
}

% If you are the author, so put your name here
\renewcommand{\author}{Collin McDevitt}

\pagestyle{myheadings}
\markboth{\hfill\doctitle\hfill\docdate}{\docdate\hfill\doctitle\hfill}

\addtolength{\textwidth}{1.00in}
\addtolength{\textheight}{1.00in}
\addtolength{\evensidemargin}{-1.00in}
\addtolength{\oddsidemargin}{-0.00in}
\addtolength{\topmargin}{-.50in}
\setlength{\footskip}{0pt}

\newcommand{\polyreduce}{\leq_{\mathrm{P}}}

\newcounter{problem}
\setcounter{problem}{0}
\newcommand{\problem}[1]{%
\refstepcounter{problem}\noindent\textbf{[#1] \arabic{problem}.}}

\newcommand{\solution}{\bigskip\hrule\bigskip}
\newcommand{\problembreak}{\bigskip\hrule\bigskip}

\renewcommand{\theenumi}{\textbf{\Alph{enumi}}}
\renewcommand{\theenumii}{\textbf{\roman{enumii}}}

\newcommand{\emptysequence}{\Lambda}
\newcommand{\emptystring}{\epsilon}

% Pseudocode comment symbol
\newcommand{\comment}{\textbf{//}\ \ }

% Pseudocode line numbering
\newcounter{pseudocode}
\newcommand{\firstline}{\setcounter{pseudocode}{0}\linenumber}
\newcommand{\linenumber}{\refstepcounter{pseudocode}\thepseudocode}
\newcommand{\pseudoindent}{\hspace*{26pt}}

\newcommand{\nil}{\mbox{\textsc{nil}}}

% Mathematical symbols
\newcommand{\grid}{\mathcal{G}}  % Grid graph
\newcommand{\integers}{\mathbb{Z}}  % Integers
\newcommand{\Zplus}{\mathbb{Z}^+}  % Positive integers
\newcommand{\naturals}{\mathbb{N}}  % Natural numbers
\newcommand{\reals}{\mathbb{R}}  % Real numbers

% Probability
\newcommand{\expect}[1]{\mathbf{E}\left[#1\right]}
\newcommand{\prob}[1]{\mathrm{Pr}\left[#1\right]}
\newcommand{\var}[1]{\mathrm{Var}\left[#1\right]}

% Logic
\newcommand{\NOT}[1]{\neg #1}
\newcommand{\AND}{\wedge}
\newcommand{\OR}{\vee}
\newcommand{\clause}[1]{\left(#1\right)}

\newlength{\problemoffset}
\setlength{\problemoffset}{0.5in}

% Decision problem macro
% A command for formatting decision problems a la Garey and Johnson
\newcommand{\decision}[3]{%     \decision{NAME}{INSTANCE}{QUESTION}
\begin{list}{}{
\setlength{\leftmargin}{\problemoffset}
\setlength{\rightmargin}{\problemoffset}
\setlength{\parsep}{0pt}
\setlength{\itemsep}{2pt}
\setlength{\topsep}{\itemsep}
\setlength{\partopsep}{\itemsep}
}
\item
{\textsc{#1}}
\item
{INSTANCE: #2}
\item
{QUESTION: #3}
\end{list}
}

% Optimization problem macro
\newcommand{\optimization}[3]{%  \optimization{NAME}{INSTANCE}{SOLUTION}
\begin{list}{}{
\setlength{\leftmargin}{\problemoffset}
\setlength{\rightmargin}{\problemoffset}
\setlength{\parsep}{0pt}
\setlength{\itemsep}{2pt}
\setlength{\topsep}{\itemsep}
\setlength{\partopsep}{\itemsep}
}
\item
{\rule{0pt}{14pt}\textsc{#1}}
\item
{INSTANCE: #2}
\item
{SOLUTION: #3}
\end{list}
}

\newcommand{\reaches}{\leadsto}

% Table layout
\newcommand{\toprule}{\rule[11pt]{0pt}{2pt}}
\newcommand{\bottomrule}{\rule[-6pt]{0pt}{0pt}}
\newenvironment{raggedpars}[1][2.0in]{%
\begin{minipage}[t]{#1}\raggedright\toprule}%
{\bottomrule\end{minipage}}

% Dynamic programming macros
\newlength{\arrowwidth}
\setbox3=\hbox{$\nwarrow$}
\setlength{\arrowwidth}{\wd3}
\newcommand{\optnwarrow}[1]{\if1#1\nwarrow%
\else\rule{\arrowwidth}{0pt}\fi}
\newcommand{\optuparrow}[1]{\if1#1\uparrow%
\else\rule{\arrowwidth}{0pt}\fi}
\newcommand{\optleftarrow}[1]{\if1#1\leftarrow%
\else\rule{\arrowwidth}{0pt}\fi}
% Use \tablebox to specify any combination of arrows, plus the value
\newcommand{\tablebox}[4]{%
\setlength{\arraycolsep}{0.0pt}%
\begin{array}{cc}%
\optnwarrow{#1} & \optuparrow{#2} \\%
\optleftarrow{#3} & #4%
\end{array}%
}
% Use \tableboxred to specify any combination of arrows, plus the value
% The value will be red; arrows are made red if 2 used instead of 1
\newcommand{\optnwarrowred}[1]{\if1#1\nwarrow%
\else\if2#1\textcolor{red}{\nwarrow}\else\rule{\arrowwidth}{0pt}\fi\fi}
\newcommand{\optuparrowred}[1]{\if1#1\uparrow%
\else\if2#1\textcolor{red}{\uparrow}\else\rule{\arrowwidth}{0pt}\fi\fi}
\newcommand{\optleftarrowred}[1]{\if1#1\leftarrow%
\else\if2#1\textcolor{red}{\leftarrow}\else\rule{\arrowwidth}{0pt}\fi\fi}
\newcommand{\tableboxred}[4]{%
\setlength{\arraycolsep}{0.0pt}%
\begin{array}{cc}%
\optnwarrowred{#1} &%
\optuparrowred{#2} \\%
\optleftarrowred{#3} &%
\textcolor{red}{#4}%
\end{array}%
}

% Allow really sloppy paragraphs that do not generate overfull or
%    underfull hbox's
\newenvironment{SLOPPY}{\begin{sloppypar}\hbadness=10000}{\end{sloppypar}}

% Definitions for this homework
\newcommand{\extent}[1]{\mathrm{extent}(#1)}
\newcommand{\opt}[2]{\mbox{\textsc{opt}}(#1,#2)}
\newcommand{\gap}{\mbox{\texttt{-}}}
\newcommand{\rewriterule}[2]{#1\to #2}
\newcommand{\rewrites}[2][]{\mathop{\Longrightarrow}\limits_{#1}^{#2}}
\newcommand{\reduces}{\leq}
\newcommand{\polyreduces}{\leq_P}
\newcommand{\mathsc}[1]{\mbox{\normalfont\textsc{#1}}}
\newcommand{\NP}{\mathcal{NP}}
\renewcommand{\P}{\mathcal{P}}
\newcommand{\describes}{\vdash}
\newcommand{\powerset}[1]{\mathcal{P}(#1)}

% OTM related definitions
\newcommand{\Qpollacc}{Q^{\left(\mathrm{poll,acc}\right)}}
\newcommand{\Qpollrej}{Q^{\left(\mathrm{poll,rej}\right)}}
\newcommand{\Qaut}{Q^{\left(\mathrm{aut}\right)}}
\newcommand{\blank}{\ensuremath\fbox{\sc b}}

\begin{document}

\thispagestyle{empty}

\begin{center}
\begin{tabular}{lcr}
\multicolumn{3}{c}{\Large\textbf{\coursenumber}}
\\[0.04in]
\multicolumn{3}{c}{\Large\textbf{\doctitle}}
% If these are solutions, then include the author's (student's) name
\ifthenelse{\boolean{solutions}}{% Then
\\[0.04in]\multicolumn{3}{c}{\large\textbf{\author}}
}{} % Else omit
% If these are solutions, then the date is the due date
\ifthenelse{\boolean{solutions}}{% Then
\\[0.10in]\multicolumn{3}{c}{\duedate}
}{% Else, put given and due dates
\\[0.10in]
\textbf{Given:} \docdate
& \hspace*{1.0in} &
\textbf{Due:} \duedate
}
\end{tabular}
\end{center}

% If these are solutions, then we do not include the directions
\ifthenelse{\boolean{solutions}}{} % No directions
{
% Original document includes directions
\begingroup % This allows an argument that contains multiple paragraphs
\paragraph{General directions.}
The point value of each problem is shown in [ ].
Each solution must include all details and
an explanation of why the given solution is correct.
\textbf{\textcolor{red}{In particular,
write complete sentences.
A correct answer without an explanation is worth no credit.}}
The completed assignment must be submitted on Canvas as a PDF by 5:00 PM
on \duedate.
\textbf{No late homework will be accepted.}

\paragraph{Digital preparation of your solutions is mandatory.
This includes digital preparation of any drawings; see syllabus
concerning neat drawings included in \LaTeX\ solutions.}
\textbf{Use of \LaTeX\ is required.
Also,
please include your name.}

\paragraph{Use of \LaTeX\ (required).}
\begin{itemize}
\item Retrieve this \LaTeX\ source file,
named
\texttt{homework\homeworknumber.tex},
from the course web site.
\item Rename the file
\textit{$<$Your VT PID$>$}\texttt{{\myunderscore}solvehw\homeworknumber.tex},
For example,
for the instructor,
the file name would be
\texttt{heath{\myunderscore}solvehw\homeworknumber.tex}.

\item
Use a \textbf{text editor}
(such as \texttt{vi}, \texttt{emacs}, or \texttt{pico})
to accomplish the next three steps.
Alternately,
use Overleaf as your \LaTeX\ platform.

\item
Uncomment the line

\texttt{{\mypercent} 
{\mybackslash}setboolean{\myleftbracket}solutions{\myrightbracket}%
{\myleftbracket}True{\myrightbracket}}

in the document preamble by deleting the \texttt{\mypercent}.

\item
Find the line

\texttt{{\mybackslash}renewcommand%
{\myleftbracket}{\mybackslash}author{\myrightbracket}%
{\myleftbracket}Lenwood S. Heath{\myrightbracket}}

and replace the instructor's name with your name.

\item
Enter your solutions where you find
the \LaTeX\ comments

\texttt{{\mypercent} PUT YOUR SOLUTION HERE}

\item
Generate a PDF and turn it in on Canvas by 5:00 PM on \duedate.
\end{itemize}
\endgroup

\problembreak

\newpage

}

\problem{25}
Let
\begin{eqnarray*}
M_1
& = &
(\Qpollacc,\Qpollrej,\Qaut,\Sigma,\Gamma,\delta,q_0)
\end{eqnarray*}
be the OTM
with the following algebraic specification
\begin{eqnarray*}
\Qpollacc
& = &
\{q_0\}
\\
\Qpollrej
& = &
\{q_1\}
\\
\Qaut
& = &
\emptyset
\\
\Sigma
& = &
\{0,1\}
\\
\Gamma
& = &
\{0,1,\blank\,\},
\end{eqnarray*}
where the specification of $\delta$ is given in a transition table;
see Table~\ref{table:transition1}.

\begin{table}[bp]
\begin{displaymath}
% \begin{array}{c|cc}
% q & \delta((q,0),\gamma) & \delta((q,1),\gamma)
% \\[2pt]\hline\hline
% q_0 & (q_1,\gamma,R) & (q_1,\gamma,R)
% \\\hline
% q_1 & (q_0,\gamma,R) & (q_0,\gamma,R)
% \end{array}
\begin{array}{c|cccccc}
q
& \delta((q,0),0) & \delta((q,1),0)
& \delta((q,0),1) & \delta((q,1),1)
& \delta((q,0),\blank\,) & \delta((q,1),\blank\,)
\\[2pt]\hline\hline
q_0
& (q_1,0,R) & (q_1,1,R)
& (q_1,0,R) & (q_1,1,R)
& (q_1,0,R) & (q_1,1,R)
\\\hline
q_1
& (q_0,0,R) & (q_0,1,R)
& (q_0,0,R) & (q_0,1,R)
& (q_0,0,R) & (q_0,1,R)
\end{array}
\end{displaymath}
\caption{Transition table for Problem 1.}
\label{table:transition1}
\end{table}

{\bfseries
\begin{enumerate}

\item
On input $w=01100$,
give the computation
(sequence of configurations)
that $M_1$ goes through.

\item
Let $L_1$ be the language accepted by $M_1$.
What is $L_1$?
Explain.

\end{enumerate}
}



\solution


The sequence of configurations is 

 \[C_{0}^{M_1}(01100)=\langle \varepsilon, \blank q_0 \blank \blank\rangle \]
\[C_{1}^{M_1}(01100)=\langle 0, \blank 0 q_1\blank  \rangle\]
\[C_{2}^{M_1}(01100)=\langle 01,\blank 0 1 q_0 \blank \rangle\]
\[C_3^{M_1}(01100)=\langle 011 , \blank 01 1 q_1\blank\rangle \]
\[C_4^{M_1}(01100)=\langle 0110,\blank 01 1 0 q_0\blank\rangle\]
\[C_5^{M_1}(01100)=\langle 01100,\blank 01100 q_1 \blank \rangle \]


The language that $M_1$ accepts is given by $$L_1=\{w\in \{0,1\}^\star: w \text{ has even length} \}$$.
We have no autonomous states and all the state transition functions only ever move right hence we only have to examine when the final state is $q_0$. Looking at the transition table we see that on any input and any tape symbol the state goes from $q_0$ to $q_1$ or from $q_1$ to $q_0$ informally the state flips at each step in the configuration. As the initial state is $q_0$ then for us to end up in $q_0$ again we need to have flipped an even number of times which implies that the word was even length.

\problembreak

\clearpage

\problem{25}
Let $L_2\subseteq\{a,b\}^\star$ be the language
\begin{eqnarray*}
L_2
& = &
\{(aab)^i\mid i\geq0\}.
\end{eqnarray*}
Three example strings in $L_2$ are
$\emptystring$,
$aabaab$, and $aabaabaabaabaab$.
Three example strings not in $L_2$ are
$baa$, $aaaab$, and $aabaa$.

{\bfseries
Design an OTM $M_2$
that accepts $L_2$.
Give the complete algebraic specification for $M_2$:
\begin{eqnarray*}
M_2
& = &
(\Qpollacc,\Qpollrej,\Qaut,\Sigma,\Gamma,\delta,q_0).
\end{eqnarray*}
You may want to put $\delta$ in a transition table;
otherwise,
you can specify all the transitions
as equations
in an \verb,eqnarray*, or a \verb,displaymath, environment.

{\normalfont\textsc{Hint:}}
You will need a dead state that is in $\Qpollrej$.

}

\solution

Let $M_2=(\Qpollacc,\Qpollrej,\Qaut,\Sigma,\Gamma,\delta)$ be an OTM with the following algebraic specification. 

\[
    Q^{(\text{pull,acc})}=\{q_0,q_3\}
\]
\[
    Q^{(\text{pull,rej})}=\{q_1,q_2,q_4\}    
\]
\[
\Qaut=\emptyset    
\]
\[
    \Sigma=\{a,b\}
\]
\[
    \Gamma=\{a,b,\blank\}
\]
Where $\delta$ is given by the state transition Table 2. Note each of the transition functions will be doing the same thing regardless of what is on the tape so for the sake of readability I will construct the table with an arbitrary tape element $d\in \Gamma$.

\begin{table}[bp]
    \begin{displaymath}
    \begin{array}{c|cccccc}
    q
    & \delta((q,a),d) & \delta((q,b),d)
    
    \\[2pt]\hline\hline
    q_0
    & (q_1,a,R) & (q_4,b,R)
    \\\hline
    q_1
    & (q_2,a,R) & (q_4,b,R)
    \\\hline
    q_2 & (q_4,a,R) & (q_3,b,R)
    \\\hline
    q_3 &
    (q_1,a,R) & (q_4,b,R)\\\hline
    q_4 & 
    (q_4,a,R) & (q_4,b,R)
    \end{array}
    \end{displaymath}
    \caption{Transition table for Problem 2 with  $d\in \Gamma$}
    \label{table:transition1}
    \end{table}
    

\problembreak

\newpage

\problem{20}
Let $D$ be a set of cards from a standard deck,
52 cards in all.
Let $C$ be a set of $50$ pennies,
each with a different year so they are easily distinguished.

{\bfseries
Prove by contradiction
that there is no injection
\begin{displaymath}
h:D\to C.
\end{displaymath}

{\normalfont\textsc{Hint:}}
Assume that such a injection $h$ exists
and derive a contradiction.
Think about the Pigeonhole Principle.
}

\solution

% PUT YOUR SOLUTION HERE

\problembreak

\problem{30}
Let $A$ be any nonempty set.
Let $\powerset{A}$ be the power set of $A$,
the set of all subsets of $A$;
see Page 492.

{\bfseries
Prove by contradiction (diagonalization)
that there is no bijection
\begin{displaymath}
h:A\to\powerset{A}.
\end{displaymath}

{\normalfont\textsc{Hint:}}
Assume that such a bijection $h$ exists
and derive a contradiction.
Be careful with your argument.
}

\solution

% PUT YOUR SOLUTION HERE

\problembreak

\end{document}