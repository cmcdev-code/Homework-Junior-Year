\documentclass[11pt,twoside]{article}

\usepackage{amssymb}
\usepackage{amsmath}
\usepackage{latexsym}
\usepackage{color}
\usepackage{graphics}
\usepackage{xspace}

% Commands for special characters
\newcommand{\mybackslash}{\char'134}
\newcommand{\myleftbracket}{\char'173}
\newcommand{\myrightbracket}{\char'175}
\newcommand{\mypercent}{\char'045}
\newcommand{\myunderscore}{\char'137}

% The 'ifthen' package supports Boolean flags
\usepackage{ifthen}
% The 'solutions' flag determines whether this is the original handout
%    or a solution
\newboolean{solutions}
\setboolean{solutions}{False}  % Default is original handout

% Uncomment the next line before starting on the solutions
% \setboolean{solutions}{True}

\newcommand{\coursenumber}{CS 4124}
\newcommand{\coursetitle}{Theory of Computation}
\newcommand{\docdate}{April 10, 2024}
\newcommand{\duedate}{April 26, 2024}
\newcommand{\homeworknumber}{6}

% The document title depends on whether these are solutions
\ifthenelse{\boolean{solutions}}{% Then
\newcommand{\doctitle}{Solutions to Homework Assignment \homeworknumber}
}{% Else
\newcommand{\doctitle}{Homework Assignment \homeworknumber}
}

% The document date depends on whether these are solutions
\ifthenelse{\boolean{solutions}}{% Then
\renewcommand{\docdate}{\duedate}
}{% Else
}

% If you are the author, so put your name here
\renewcommand{\author}{Collin McDevitt}

\pagestyle{myheadings}
\markboth{\hfill\doctitle\hfill\docdate}{\docdate\hfill\doctitle\hfill}

\addtolength{\textwidth}{1.00in}
\addtolength{\textheight}{1.00in}
\addtolength{\evensidemargin}{-1.00in}
\addtolength{\oddsidemargin}{-0.00in}
\addtolength{\topmargin}{-.50in}
\setlength{\footskip}{0pt}

\newcommand{\polyreduce}{\leq_{\mathrm{P}}}

\newcounter{problem}
\setcounter{problem}{0}
\newcommand{\problem}[1]{%
\refstepcounter{problem}\noindent\textbf{[#1] \arabic{problem}.}}

\newcommand{\solution}{\bigskip\hrule\bigskip}
\newcommand{\problembreak}{\bigskip\hrule\bigskip}

\renewcommand{\theenumi}{\textbf{\Alph{enumi}}}
\renewcommand{\theenumii}{\textbf{\roman{enumii}}}

\newcommand{\emptysequence}{\Lambda}
\newcommand{\emptystring}{\varepsilon}

% Pseudocode comment symbol
\newcommand{\comment}{\textbf{//}\ \ }

% Pseudocode line numbering
\newcounter{pseudocode}
\newcommand{\firstline}{\setcounter{pseudocode}{0}\linenumber}
\newcommand{\linenumber}{\refstepcounter{pseudocode}\thepseudocode}
\newcommand{\pseudoindent}{\hspace*{26pt}}

\newcommand{\nil}{\mbox{\textsc{nil}}}

% Mathematical symbols
\newcommand{\grid}{\mathcal{G}}  % Grid graph
\newcommand{\integers}{\mathbb{Z}}  % Integers
\newcommand{\Zplus}{\mathbb{Z}^+}  % Positive integers
\newcommand{\naturals}{\mathbb{N}}  % Natural numbers
\newcommand{\reals}{\mathbb{R}}  % Real numbers

% Probability
\newcommand{\expect}[1]{\mathbf{E}\left[#1\right]}
\newcommand{\prob}[1]{\mathrm{Pr}\left[#1\right]}
\newcommand{\var}[1]{\mathrm{Var}\left[#1\right]}

% Logic
\newcommand{\NOT}[1]{\neg #1}
\newcommand{\AND}{\wedge}
\newcommand{\OR}{\vee}
\newcommand{\clause}[1]{\left(#1\right)}

\newlength{\problemoffset}
\setlength{\problemoffset}{0.5in}

% Decision problem macro
% A command for formatting decision problems a la Garey and Johnson
\newcommand{\decision}[3]{%     \decision{NAME}{INSTANCE}{QUESTION}
\begin{list}{}{
\setlength{\leftmargin}{\problemoffset}
\setlength{\rightmargin}{\problemoffset}
\setlength{\parsep}{0pt}
\setlength{\itemsep}{2pt}
\setlength{\topsep}{\itemsep}
\setlength{\partopsep}{\itemsep}
}
\item
{\textsc{#1}}
\item
{INSTANCE: #2}
\item
{QUESTION: #3}
\end{list}
}

% Optimization problem macro
\newcommand{\optimization}[3]{%  \optimization{NAME}{INSTANCE}{SOLUTION}
\begin{list}{}{
\setlength{\leftmargin}{\problemoffset}
\setlength{\rightmargin}{\problemoffset}
\setlength{\parsep}{0pt}
\setlength{\itemsep}{2pt}
\setlength{\topsep}{\itemsep}
\setlength{\partopsep}{\itemsep}
}
\item
{\rule{0pt}{14pt}\textsc{#1}}
\item
{INSTANCE: #2}
\item
{SOLUTION: #3}
\end{list}
}

\newcommand{\reaches}{\leadsto}

% Table layout
\newcommand{\toprule}{\rule[11pt]{0pt}{2pt}}
\newcommand{\bottomrule}{\rule[-6pt]{0pt}{0pt}}
\newenvironment{raggedpars}[1][2.0in]{%
\begin{minipage}[t]{#1}\raggedright\toprule}%
{\bottomrule\end{minipage}}

% Dynamic programming macros
\newlength{\arrowwidth}
\setbox3=\hbox{$\nwarrow$}
\setlength{\arrowwidth}{\wd3}
\newcommand{\optnwarrow}[1]{\if1#1\nwarrow%
\else\rule{\arrowwidth}{0pt}\fi}
\newcommand{\optuparrow}[1]{\if1#1\uparrow%
\else\rule{\arrowwidth}{0pt}\fi}
\newcommand{\optleftarrow}[1]{\if1#1\leftarrow%
\else\rule{\arrowwidth}{0pt}\fi}
% Use \tablebox to specify any combination of arrows, plus the value
\newcommand{\tablebox}[4]{%
\setlength{\arraycolsep}{0.0pt}%
\begin{array}{cc}%
\optnwarrow{#1} & \optuparrow{#2} \\%
\optleftarrow{#3} & #4%
\end{array}%
}
% Use \tableboxred to specify any combination of arrows, plus the value
% The value will be red; arrows are made red if 2 used instead of 1
\newcommand{\optnwarrowred}[1]{\if1#1\nwarrow%
\else\if2#1\textcolor{red}{\nwarrow}\else\rule{\arrowwidth}{0pt}\fi\fi}
\newcommand{\optuparrowred}[1]{\if1#1\uparrow%
\else\if2#1\textcolor{red}{\uparrow}\else\rule{\arrowwidth}{0pt}\fi\fi}
\newcommand{\optleftarrowred}[1]{\if1#1\leftarrow%
\else\if2#1\textcolor{red}{\leftarrow}\else\rule{\arrowwidth}{0pt}\fi\fi}
\newcommand{\tableboxred}[4]{%
\setlength{\arraycolsep}{0.0pt}%
\begin{array}{cc}%
\optnwarrowred{#1} &%
\optuparrowred{#2} \\%
\optleftarrowred{#3} &%
\textcolor{red}{#4}%
\end{array}%
}

% Allow really sloppy paragraphs that do not generate overfull or
%    underfull hbox's
\newenvironment{SLOPPY}{\begin{sloppypar}\hbadness=10000}{\end{sloppypar}}

% Definitions for this homework
\newcommand{\extent}[1]{\mathrm{extent}(#1)}
\newcommand{\opt}[2]{\mbox{\textsc{opt}}(#1,#2)}
\newcommand{\gap}{\mbox{\texttt{-}}}
\newcommand{\rewriterule}[2]{#1\to #2}
\newcommand{\rewrites}[2][]{\mathop{\Longrightarrow}\limits_{#1}^{#2}}
\newcommand{\reduces}{\leq}
\newcommand{\polyreduces}{\leq_P}
\newcommand{\mathsc}[1]{\mbox{\normalfont\textsc{#1}}}
\newcommand{\NP}{\mathbf{NP}}
\renewcommand{\P}{\mathbf{P}}
\newcommand{\describes}{\vdash}
\newcommand{\powerset}[1]{\mathcal{P}(#1)}

% OTM related definitions
\newcommand{\Qpollacc}{Q^{\left(\mathrm{poll,acc}\right)}}
\newcommand{\Qpollrej}{Q^{\left(\mathrm{poll,rej}\right)}}
\newcommand{\Qaut}{Q^{\left(\mathrm{aut}\right)}}
\newcommand{\blank}{\ensuremath\fbox{\sc b}}

\begin{document}

\thispagestyle{empty}

\begin{center}
\begin{tabular}{lcr}
\multicolumn{3}{c}{\Large\textbf{\coursenumber}}
\\[0.04in]
\multicolumn{3}{c}{\Large\textbf{\doctitle}}
% If these are solutions, then include the author's (student's) name
\ifthenelse{\boolean{solutions}}{% Then
\\[0.04in]\multicolumn{3}{c}{\large\textbf{\author}}
}{} % Else omit
% If these are solutions, then the date is the due date
\ifthenelse{\boolean{solutions}}{% Then
\\[0.10in]\multicolumn{3}{c}{\duedate}
}{% Else, put given and due dates
\\[0.10in]
\textbf{Given:} \docdate
& \hspace*{1.0in} &
\textbf{Due:} \duedate
}
\end{tabular}
\end{center}

% If these are solutions, then we do not include the directions
\ifthenelse{\boolean{solutions}}{} % No directions
{
% Original document includes directions
\begingroup % This allows an argument that contains multiple paragraphs
\paragraph{General directions.}
The point value of each problem is shown in [ ].
Each solution must include all details and
an explanation of why the given solution is correct.
\textbf{\textcolor{red}{In particular,
write complete sentences.
A correct answer without an explanation is worth no credit.}}
The completed assignment must be submitted on Canvas as a PDF by 5:00 PM
on \duedate.
\textbf{No late homework will be accepted.}

\paragraph{Digital preparation of your solutions is mandatory.
This includes digital preparation of any drawings; see syllabus
concerning neat drawings included in \LaTeX\ solutions.}
\textbf{Use of \LaTeX\ is required.
Also,
please include your name.}

\paragraph{Use of \LaTeX\ (required).}
\begin{itemize}
\item Retrieve this \LaTeX\ source file,
named
\texttt{homework\homeworknumber.tex},
from the course web site.
\item Rename the file
\textit{$<$Your VT PID$>$}\texttt{{\myunderscore}solvehw\homeworknumber.tex},
For example,
for the instructor,
the file name would be
\texttt{heath{\myunderscore}solvehw\homeworknumber.tex}.

\item
Use a \textbf{text editor}
(such as \texttt{vi}, \texttt{emacs}, or \texttt{pico})
to accomplish the next three steps.
Alternately,
use Overleaf as your \LaTeX\ platform.

\item
Uncomment the line

\texttt{{\mypercent} 
{\mybackslash}setboolean{\myleftbracket}solutions{\myrightbracket}%
{\myleftbracket}True{\myrightbracket}}

in the document preamble by deleting the \texttt{\mypercent}.

\item
Find the line

\texttt{{\mybackslash}renewcommand%
{\myleftbracket}{\mybackslash}author{\myrightbracket}%
{\myleftbracket}Lenwood S. Heath{\myrightbracket}}

and replace the instructor's name with your name.

\item
Enter your solutions where you find
the \LaTeX\ comments

\texttt{{\mypercent} PUT YOUR SOLUTION HERE}

\item
Generate a PDF and turn it in on Canvas by 5:00 PM on \duedate.
\end{itemize}
\endgroup

\problembreak

\newpage

}

\newcommand{\SSP}{\textbf{SSP}\xspace}

\problem{20}
{\bfseries
Textbook Problem 5(a) in B.15
on Page 538.

Prove that the following set (language)
belongs to the class $\NP$:

a.
The set of binary representations
of composite (i.e., nonprime) integers.

}

\solution

This belongs to $\P$ as given an integer $n$ we can just check if it is divisible by any number greater then $1$ and less then $n$. This can be done in polynomial time by checking $\{(i,j): 2\leq i,j\leq n-1\}$ just multiplying each of the entries and checking if it equals $n$. Will tell weather it is composite or not.

\problembreak

\problem{30}
{\bfseries
Textbook Problem 12(a) in B.15
on Page 538.

Provide a rigorously analyzed answer to the following question:

a. What is the ``size''
(in the complexity-theoretic sense of the term)
of a 3-CNF formula that has $m$ clauses and $n$ variables?

{\normalfont\textsc{Hint:}}
How would you design a TM that could process
\textit{any} 3-CNF formula?
The only ``tough'' issue is how to represent ``names''
(in this case, of the variables).
You should use the most compact
(to within constant factors)
fixed-length encoding of ``names''.
Use the binary alphabet $\Sigma=\{0,1\}$
for the encoding.
}

\solution

We need to a fixed-length encoding of the names. This can done by if we know that there is $n$ variables then we can use $\lceil \log_2(n)\rceil $ bits to encode the name of the variable by using the binary representation of the variable number. As there is $m$ clauses and each clause has $3$ variables we can encode the entire formula in $3m\lceil \log_2(n)\rceil$ bits. Hence we can say that the size of the 3-CNF formula is $O(m\log(n))$.


\problembreak

\problem{50}
For each integer $n\geq1$,
let $S_n=\{1,2,3,\ldots,n\}$.
A \textit{family} of $S_n$
is a set $F$ of nonempty subsets of $S_n$;
note that the empty set is not allowed to be an element of $F$.
If $f\in F$ and $i\in f$,
then $i$ is a \textit{representative} of $f$.
A \textit{representative set} $R$ for $F$
is a subset $R\subseteq S_n$ such that $R$
contains at least one representative of each $f\in F$.
The \textsc{Minimum Family Representation} problem is
to find a minimum-size representative set $R$ for $F$.

A formal statement
of the \textsc{Minimum Family Representation} problem follows:
\optimization{Minimum Family Representation}%
{Integer $n\geq1$;
$F$, a family of $S_n$.}%
{A subset $R\subseteq S_n$
that is a representative set of $F$
and such that $|R|$ is minimum.}

To obtain a decision problem,
we add the parameter $K$ to the instance
and
formulate the maximization problem
in terms of a question involving $K$,
as follows:
\decision{Decision Family Representative}%
{Integer $n\geq1$;
$F$, a family of $S_n$; integer $K\geq0$.}%
{Is there representative set $R$ of $F$
such that $|R|\leq K$?}

The resulting language is just
\begin{eqnarray*}
\mathrm{DFR}
& = &
\{
\rho(n,F,K)
\mid
F \mathrm{\ is\ a\ family\ of\ } S_n
\mathrm{\ with\ a\ representative\ set\ of\ size\ } \leq K
\}.
\end{eqnarray*}

\begin{SLOPPY}

{\bfseries
Prove that {\normalfont{\textsc{DFR}}} is NP-complete.
Follow the NP-completeness Proof Paradigm from class.
Hint: Use a polynomial-time reduction from {\normalfont{\textsc{VERTEX-COVER}}}.
}

\end{SLOPPY}

\solution

For a graph $G=(V,E)$ denote the neighbors of a vertex $v\in V$ by the set $$\mathcal N(v):=\{a\in V: a\text{ is neighbors with }v\}\cup \{v\}$$. Then we have the following reduction from ${\normalfont{\textsc{VERTEX-COVER}}}\leq_{poly}{\normalfont{\textsc{DFR}}}$ which is given by the function $f:\Sigma^{\star}\to \Sigma^\star$. Given a graph $G=(V,E)$ with $|V|=n$ create an enumeration of the vertexes from $1,...,n$.  Then create the set of subsets of vertexes $$U:=\bigcup_{i=1}^n\{\mathcal{N}(v_i)\}$$. Then let $f$ be given by the equation $f(G,K)\to (|V|,U,K)$. 

Now to show that this function can be computed in polynomial time. We can enumerate the neighbors of each vertex in $O(n^2)$ time hence we can create $U$ in $O(n^2)$ time. 

Now there are two cases if $(G,K)\in{\normalfont{\textsc{VERTEX-COVER}}} $ then there exists a vertex cover of size $K$. Find one such vertex cover of size $K$ (this can be done in nondeterministic polynomial time) and look at of the edges in the vertex cover. For each edge in the vertex cover choose a single vertex from the edge and add it to the set $R$. We have that $|R|=K$ and by the construction of $U$ we have that $R$ is a representative set of $U$. Hence we have that $f(G,K)\in{\normalfont{\textsc{DFR}}}$.

Now if $(G,K)\not \in {\normalfont{\textsc{VERTEX-COVER}}}$ then there does not exist a vertex cover of size $K$. By the construction of $U$ we have that there does not exist a representative $|R|<K$. Hence we have that $f(G,K)\not \in {\normalfont{\textsc{DFR}}}$. This shows that $f$ is a polynomial time reduction from ${\normalfont{\textsc{VERTEX-COVER}}}$ to ${\normalfont{\textsc{DFR}}}$. Hence we have that ${\normalfont{\textsc{DFR}}}$ is $\NP$-complete.
\problembreak

\end{document}
