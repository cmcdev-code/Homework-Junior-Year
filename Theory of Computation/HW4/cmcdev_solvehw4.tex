\documentclass[11pt,twoside]{article}

\usepackage{amssymb}
\usepackage{amsmath}
\usepackage{latexsym}
\usepackage{color}
\usepackage{graphics}
\usepackage{xspace}

% Commands for special characters
\newcommand{\mybackslash}{\char'134}
\newcommand{\myleftbracket}{\char'173}
\newcommand{\myrightbracket}{\char'175}
\newcommand{\mypercent}{\char'045}
\newcommand{\myunderscore}{\char'137}

% The 'ifthen' package supports Boolean flags
\usepackage{ifthen}
% The 'solutions' flag determines whether this is the original handout
%    or a solution
\newboolean{solutions}
\setboolean{solutions}{False}  % Default is original handout

% Uncomment the next line before starting on the solutions
\setboolean{solutions}{True}

\newcommand{\coursenumber}{CS 4124}
\newcommand{\coursetitle}{Theory of Computation}
\newcommand{\docdate}{March 14, 2024}
\newcommand{\duedate}{March 29, 2024}
\newcommand{\homeworknumber}{4}

% The document title depends on whether these are solutions
\ifthenelse{\boolean{solutions}}{% Then
\newcommand{\doctitle}{Solutions to Homework Assignment \homeworknumber}
}{% Else
\newcommand{\doctitle}{Homework Assignment \homeworknumber}
}

% The document date depends on whether these are solutions
\ifthenelse{\boolean{solutions}}{% Then
\renewcommand{\docdate}{\duedate}
}{% 
}

% If you are the author, so put your name here
\renewcommand{\author}{Collin McDevitt}

\pagestyle{myheadings}
\markboth{\hfill\doctitle\hfill\docdate}{\docdate\hfill\doctitle\hfill}

\addtolength{\textwidth}{1.00in}
\addtolength{\textheight}{1.00in}
\addtolength{\evensidemargin}{-1.00in}
\addtolength{\oddsidemargin}{-0.00in}
\addtolength{\topmargin}{-.50in}
\setlength{\footskip}{0pt}

\newcommand{\polyreduce}{\leq_{\mathrm{P}}}

\newcounter{problem}
\setcounter{problem}{0}
\newcommand{\problem}[1]{%
\refstepcounter{problem}\noindent\textbf{[#1] \arabic{problem}.}}

\newcommand{\solution}{\bigskip\hrule\bigskip}
\newcommand{\problembreak}{\bigskip\hrule\bigskip}

\renewcommand{\theenumi}{\textbf{\Alph{enumi}}}
\renewcommand{\theenumii}{\textbf{\roman{enumii}}}

\newcommand{\emptysequence}{\Lambda}
\newcommand{\emptystring}{\epsilon}

% Pseudocode comment symbol
\newcommand{\comment}{\textbf{//}\ \ }

% Pseudocode line numbering
\newcounter{pseudocode}
\newcommand{\firstline}{\setcounter{pseudocode}{0}\linenumber}
\newcommand{\linenumber}{\refstepcounter{pseudocode}\thepseudocode}
\newcommand{\pseudoindent}{\hspace*{26pt}}

\newcommand{\nil}{\mbox{\textsc{nil}}}

% Mathematical symbols
\newcommand{\grid}{\mathcal{G}}  % Grid graph
\newcommand{\integers}{\mathbb{Z}}  % Integers
\newcommand{\Zplus}{\mathbb{Z}^+}  % Positive integers
\newcommand{\naturals}{\mathbb{N}}  % Natural numbers
\newcommand{\reals}{\mathbb{R}}  % Real numbers

% Probability
\newcommand{\expect}[1]{\mathbf{E}\left[#1\right]}
\newcommand{\prob}[1]{\mathrm{Pr}\left[#1\right]}
\newcommand{\var}[1]{\mathrm{Var}\left[#1\right]}

% Logic
\newcommand{\NOT}[1]{\neg #1}
\newcommand{\AND}{\wedge}
\newcommand{\OR}{\vee}
\newcommand{\clause}[1]{\left(#1\right)}

\newlength{\problemoffset}
\setlength{\problemoffset}{0.5in}

% Decision problem macro
% A command for formatting decision problems a la Garey and Johnson
\newcommand{\decision}[3]{%     \decision{NAME}{INSTANCE}{QUESTION}
\begin{list}{}{
\setlength{\leftmargin}{\problemoffset}
\setlength{\rightmargin}{\problemoffset}
\setlength{\parsep}{0pt}
\setlength{\itemsep}{2pt}
\setlength{\topsep}{\itemsep}
\setlength{\partopsep}{\itemsep}
}
\item
{\textsc{#1}}
\item
{INSTANCE: #2}
\item
{QUESTION: #3}
\end{list}
}

% Optimization problem macro
\newcommand{\optimization}[3]{%  \optimization{NAME}{INSTANCE}{SOLUTION}
\begin{list}{}{
\setlength{\leftmargin}{\problemoffset}
\setlength{\rightmargin}{\problemoffset}
\setlength{\parsep}{0pt}
\setlength{\itemsep}{2pt}
\setlength{\topsep}{\itemsep}
\setlength{\partopsep}{\itemsep}
}
\item
{\rule{0pt}{14pt}\textsc{#1}}
\item
{INSTANCE: #2}
\item
{SOLUTION: #3}
\end{list}
}

\newcommand{\reaches}{\leadsto}

% Table layout
\newcommand{\toprule}{\rule[11pt]{0pt}{2pt}}
\newcommand{\bottomrule}{\rule[-6pt]{0pt}{0pt}}
\newenvironment{raggedpars}[1][2.0in]{%
\begin{minipage}[t]{#1}\raggedright\toprule}%
{\bottomrule\end{minipage}}

% Dynamic programming macros
\newlength{\arrowwidth}
\setbox3=\hbox{$\nwarrow$}
\setlength{\arrowwidth}{\wd3}
\newcommand{\optnwarrow}[1]{\if1#1\nwarrow%
\else\rule{\arrowwidth}{0pt}\fi}
\newcommand{\optuparrow}[1]{\if1#1\uparrow%
\else\rule{\arrowwidth}{0pt}\fi}
\newcommand{\optleftarrow}[1]{\if1#1\leftarrow%
\else\rule{\arrowwidth}{0pt}\fi}
% Use \tablebox to specify any combination of arrows, plus the value
\newcommand{\tablebox}[4]{%
\setlength{\arraycolsep}{0.0pt}%
\begin{array}{cc}%
\optnwarrow{#1} & \optuparrow{#2} \\%
\optleftarrow{#3} & #4%
\end{array}%
}
% Use \tableboxred to specify any combination of arrows, plus the value
% The value will be red; arrows are made red if 2 used instead of 1
\newcommand{\optnwarrowred}[1]{\if1#1\nwarrow%
\else\if2#1\textcolor{red}{\nwarrow}\else\rule{\arrowwidth}{0pt}\fi\fi}
\newcommand{\optuparrowred}[1]{\if1#1\uparrow%
\else\if2#1\textcolor{red}{\uparrow}\else\rule{\arrowwidth}{0pt}\fi\fi}
\newcommand{\optleftarrowred}[1]{\if1#1\leftarrow%
\else\if2#1\textcolor{red}{\leftarrow}\else\rule{\arrowwidth}{0pt}\fi\fi}
\newcommand{\tableboxred}[4]{%
\setlength{\arraycolsep}{0.0pt}%
\begin{array}{cc}%
\optnwarrowred{#1} &%
\optuparrowred{#2} \\%
\optleftarrowred{#3} &%
\textcolor{red}{#4}%
\end{array}%
}

% Allow really sloppy paragraphs that do not generate overfull or
%    underfull hbox's
\newenvironment{SLOPPY}{\begin{sloppypar}\hbadness=10000}{\end{sloppypar}}

% Definitions for this homework
\newcommand{\extent}[1]{\mathrm{extent}(#1)}
\newcommand{\opt}[2]{\mbox{\textsc{opt}}(#1,#2)}
\newcommand{\gap}{\mbox{\texttt{-}}}
\newcommand{\rewriterule}[2]{#1\to #2}
\newcommand{\rewrites}[2][]{\mathop{\Longrightarrow}\limits_{#1}^{#2}}
\newcommand{\reduces}{\leq}
\newcommand{\polyreduces}{\leq_P}
\newcommand{\mathsc}[1]{\mbox{\normalfont\textsc{#1}}}
\newcommand{\NP}{\mathcal{NP}}
\renewcommand{\P}{\mathcal{P}}
\newcommand{\describes}{\vdash}
\newcommand{\powerset}[1]{\mathcal{P}(#1)}

% OTM related definitions
\newcommand{\Qpollacc}{Q^{\left(\mathrm{poll,acc}\right)}}
\newcommand{\Qpollrej}{Q^{\left(\mathrm{poll,rej}\right)}}
\newcommand{\Qaut}{Q^{\left(\mathrm{aut}\right)}}
\newcommand{\blank}{\ensuremath\fbox{\sc b}}

\begin{document}

\thispagestyle{empty}

\begin{center}
\begin{tabular}{lcr}
\multicolumn{3}{c}{\Large\textbf{\coursenumber}}
\\[0.04in]
\multicolumn{3}{c}{\Large\textbf{\doctitle}}
% If these are solutions, then include the author's (student's) name
\ifthenelse{\boolean{solutions}}{% Then
\\[0.04in]\multicolumn{3}{c}{\large\textbf{\author}}
}{} % Else omit
% If these are solutions, then the date is the due date
\ifthenelse{\boolean{solutions}}{% Then
\\[0.10in]\multicolumn{3}{c}{\duedate}
}{% Else, put given and due dates
\\[0.10in]
\textbf{Given:} \docdate
& \hspace*{1.0in} &
\textbf{Due:} \duedate
}
\end{tabular}
\end{center}

% If these are solutions, then we do not include the directions
\ifthenelse{\boolean{solutions}}{} % No directions
{
% Original document includes directions
\begingroup % This allows an argument that contains multiple paragraphs
\paragraph{General directions.}
The point value of each problem is shown in [ ].
Each solution must include all details and
an explanation of why the given solution is correct.
\textbf{\textcolor{red}{In particular,
write complete sentences.
A correct answer without an explanation is worth no credit.}}
The completed assignment must be submitted on Canvas as a PDF by 5:00 PM
on \duedate.
\textbf{No late homework will be accepted.}

\paragraph{Digital preparation of your solutions is mandatory.
This includes digital preparation of any drawings; see syllabus
concerning neat drawings included in \LaTeX\ solutions.}
\textbf{Use of \LaTeX\ is required.
Also,
please include your name.}

\paragraph{Use of \LaTeX\ (required).}
\begin{itemize}
\item Retrieve this \LaTeX\ source file,
named
\texttt{homework\homeworknumber.tex},
from the course web site.
\item Rename the file
\textit{$<$Your VT PID$>$}\texttt{{\myunderscore}solvehw\homeworknumber.tex},
For example,
for the instructor,
the file name would be
\texttt{heath{\myunderscore}solvehw\homeworknumber.tex}.

\item
Use a \textbf{text editor}
(such as \texttt{vi}, \texttt{emacs}, or \texttt{pico})
to accomplish the next three steps.
Alternately,
use Overleaf as your \LaTeX\ platform.

\item
Uncomment the line

\texttt{{\mypercent} 
{\mybackslash}setboolean{\myleftbracket}solutions{\myrightbracket}%
{\myleftbracket}True{\myrightbracket}}

in the document preamble by deleting the \texttt{\mypercent}.

\item
Find the line

\texttt{{\mybackslash}renewcommand%
{\myleftbracket}{\mybackslash}author{\myrightbracket}%
{\myleftbracket}Lenwood S. Heath{\myrightbracket}}

and replace the instructor's name with your name.

\item
Enter your solutions where you find
the \LaTeX\ comments

\texttt{{\mypercent} PUT YOUR SOLUTION HERE}

\item
Generate a PDF and turn it in on Canvas by 5:00 PM on \duedate.
\end{itemize}
\endgroup

\problembreak

\newpage

}

\problem{30}
Let $\Sigma=\{0,1\}$,
the binary alphabet.
Let
\begin{eqnarray*}
F
& = &
\{h \mid h:\Sigma^\star\to\Sigma^\star\},
\end{eqnarray*}
the set of functions from strings to strings.

{\bfseries
Prove that $F$ is uncountable.

{\normalfont\textsc{Hint:}}
Use the fact that $\Sigma^\star$ is countably infinite
and
employ a proof by diagonalization.
}

\solution

My proof strategy:


I will prove that the set of functions $F^\prime=\{h: h :\Sigma^\star \to\{0,1\} \}$ is uncountable and as this set is a subset of $F$ that would imply that $F$ is uncountable.
I will do this by constructing a bijection between $H:F^\prime\to \mathcal P (\Sigma^\star)$ (the power set of $\Sigma^\star $) in the last homework we proved that no set can have a bijection with its power set and as $\Sigma^\star$ is countable we get that $\mathcal{P}(\Sigma^\star)$ is uncountable hence if a bijection $H$ exists we would have that $F^\prime$ is uncountable.


Let $H: F^\prime\to \mathcal P(\Sigma^\star)$ where $H(f)=\{w\in \Sigma^\star: f(w)=1\}$ (informally this is the set of elements from $\Sigma^\star$ who get mapped to $1$). 

Now to show that this is a bijection I will come up with an inverse. We create the function $$H^{-1}: \mathcal P(\Sigma^\star)\to F^\prime$$ where for $A \in \mathcal P(\Sigma^\star)$ we have $H^{-1}(A)=f$ where for all $w\in \Sigma^\star$ we have $f(w)=1$ if and only if $w\in A$. The fact that $H^{-1}$ is not a multivalued function can be proven by assuming that for some $A\in \mathcal{P}(\Sigma^\star)$ that there exists two $f_1,f_2\in F^\prime$ such that $H^{-1}(A)=f_1$ and $H^{-1}(A)=f_2$ then we have that for all $w\in \Sigma^\star$ that $f_1(w)=1$ and $f_2(w)=1$ if and only if $w\in A$ which directly implies that for any $\omega \in \Sigma^\star$ where $\omega \not \in A$ that $f_1(\omega)=0$ and $f_2(\omega)=0$. As $f_1,f_2$ are equal at every input we have that $f_1=f_2$. 

Now to show that let $f\in F^\prime$ then we have $H(f)=\{w\in \Sigma^\star: f(w)=1\}$. Then take $H^{-1}(\{w\in \Sigma^\star: f(w)=1\})=f_0$ for some $f_0\in F^\prime$. But from the definition of $H^{-1}$ we get that $f_0(\omega)=1$ if and only if $\omega \in \{w\in \Sigma^\star: f(w)=1\}$ which is the same as $f(\omega)=1$ so we have that $f_0=f$. This implies that $H^{-1}\circ H=\text{Id}_{F^\prime}$. Now let $A\in \mathcal P(\Sigma^\star)$ then we have that $H^{-1}(A)=f$ for some $f\in F^\prime$. Then taking $H(f)=\{w\in \Sigma ^\star: f(w)=1\}$ but from the definition of $f$ we have that for all $\omega \in \Sigma^\star$ that $f(\omega)=1$ if and only if $\omega \in A$. This implies that $H(f)=A$ so we have $H\circ H^{-1}(A)=A$ so we have that $H$ is the inverse of $H^{-1}$ so we have that $H$ is a bijection. 

Then as a bijection between $F^\prime\subset F$ to $\mathcal{P}(\Sigma^\star)$ exists and $\mathcal{P}(\Sigma^\star)$ is uncountable we get $F^\prime$ is uncountable and so $F$ is uncountable as well.



\problembreak

\problem{30}
Let
\begin{eqnarray*}
E
& = &
\{w\in\{0,1\}^\star\mid \ell(w) \mathrm{\ is\ even}\},
\end{eqnarray*}
the language of binary strings of even length.
See Figure~\ref{figure:sampleprogram}
for a sample program that decides $E$,
just for \LaTeX\ formatting purposes.
Let
\begin{eqnarray*}
L
& = &
\{x\mid \mathrm{program\ } x \mathrm{\ decides\ } E\}
\end{eqnarray*}
be the language of programs that decide $E$.

{\bfseries
Prove that $L$ is undecidable by using m-reducibility.
}

\solution

We have that $DHP$ is undecidable hence I will create the $m$-reduction $DHP\leq_M L$. We do this by taking any program $x\to P_x$ where $P_x$ is the program above. Note that $E$ is regular as a simple FTA that has two states could recognize it. Hence I am justified in asking weather $y$ has even length in $P$.

\begin{figure}[tp]
    \begin{tabular}{|l|}
    \hline
    \textbf{Program} $P_x$ \\
    \textbf{Input:} $y$ \\
    \textbf{if} $x$ halts on input $x$ \\
    $\;\;$ \textbf{       if} $y$ has even length 
     \\
    $\;\;\;\;$ \textbf{       then} accept \\
     $\;\;\;\;$ \textbf{       else} reject \\
    \textbf{else} loop forever \\
    \hline
    \end{tabular}
    \label{figure:sampleprogram}
\end{figure}



Now let $x\in \Sigma^{\star}$ and suppose that $x\in DHP$ then we have that from the definition of $P_x$ that for any input $y$ that $P_x(y)$ will accept if $y$ has even length and reject if $y$ has odd length. This implies that $P_x$ decides $E$ so we have that $P_x \in L$. Now for any $x\in \Sigma^\star$ consider $P_x\in L$ then we have that $P_x$ would have decided $E$ hence it halts so we have that $x$ halts on input $x$ hence $x\in DHP$. This implies that $x\in DHP$ if and only if $P(x)\in L$ so we have that $DHP\leq_M L$ so $L$ is undecidable as $DHP$ is undecidable.

\problembreak




\problem{40}
Let
\begin{eqnarray*}
\mathrm{DOUBLE}
& = &
\{ww\mid w\in\{a,b\}^\star\}.
\end{eqnarray*}
Let
\begin{eqnarray*}
W
& = &
\{x\mid \mathrm{program\ } x \mathrm{\ decides\ DOUBLE}\}
\end{eqnarray*}
be the language of programs that decide $\mathrm{DOUBLE}$.

{\bfseries
\begin{enumerate}
\item
Is $\mathrm{DOUBLE}$ decidable?
Justify your answer.
\item
Is $W$ decidable?
Prove your answer.
\end{enumerate}
}



\solution

A: Yes $\mathrm{DOUBLE}$ is decidable. We could create the program that takes in an input and checks if the input has even length and checks if the input is a palindrome. If both of these are true we accept otherwise we reject. This program would be able to decide $\mathrm{DOUBLE}$ as the only strings that are in $\mathrm{DOUBLE}$ are palindromes of even length.

\problembreak

\begin{figure}[tp]
    \begin{tabular}{|l|}
    \hline
    \textbf{Program} $P_x$ \\
    \textbf{Input:} $y$ \\
    \textbf{if} $x$ halts on input $x$ \\
    $\;\;$ \textbf{       if}  $y\in \mathrm{DOUBLE}$ 
     \\
    $\;\;\;\;$ \textbf{       then} accept \\
     $\;\;\;\;$ \textbf{       else} reject \\
    \textbf{else} loop forever \\
    \hline
    \end{tabular}
    \label{figure:sampleprogra11m}
\end{figure}



B: We have that $W$ is not decidable. I will show this by creating the $m$-reduction $DHP\leq_M W$. We do this by taking any program $x\to P_x$ where $P_x$ is the program above. Note that $\mathrm{DOUBLE}$ is decidable using the same reasoning as in part A. Hence I am justified in asking weather $y\in \mathrm{DOUBLE}$ in $P_x$.

Now to show that this is a valid $m$-reduction. Suppose that $x\in DHP$ then we have from the definition of $P_x$ that for any input $y$ that it either accepts or rejects.



\end{document}
